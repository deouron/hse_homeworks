\documentclass[a4paper, 16pt]{article}

\usepackage[utf8]{inputenc}

\usepackage[russian, english]{babel}
\usepackage{subfiles}
\usepackage[utf8]{inputenc}
\usepackage[T2A]{fontenc}
\usepackage{ucs}
\usepackage{textcomp}
\usepackage{array}
\usepackage{indentfirst}
\usepackage{amsmath}
\usepackage{amssymb}
\usepackage{enumerate}
\usepackage[margin=1.2cm]{geometry}
\usepackage{authblk}
\usepackage{tikz}
\usepackage{icomma}
\usepackage{gensymb}
\usepackage{graphicx}

\DeclareGraphicsExtensions{,.png,.jpg}

\graphicspath{{pictures/}}

\renewcommand{\baselinestretch}{1.4}

\renewcommand{\C}{\mathbb{C}}
\newcommand{\N} {\mathbb{N}}
\newcommand{\Q} {\mathbb{Q}}
\newcommand{\Z} {\mathbb{Z}}
\newcommand{\R} {\mathbb{R}}
\newcommand{\ord} {\mathop{\rm ord}}
\newcommand{\Ima}{\mathop{\rm Im}}
\newcommand{\rk}{\mathop{\rm rk}}

\renewcommand{\r}{\right}
\renewcommand{\l}{\left}
\renewcommand{\inf}{\infty}
\newcommand{\Sum}[2]{\overset{#2}{\underset{#1}{\sum}}}
\newcommand{\Lim}[2]{\lim\limits_{#1 \rightarrow #2}}
\newcommand\tab[1][1cm]{\hspace*{#1}}

\newcommand{\task}[1] {\noindent \textbf{Задача #1.} \hfill}
\newcommand{\note}[1] {\noindent \textbf{Примечание #1.} \hfill}
\newenvironment{proof}[1][Доказательство]{%
	\begin{trivlist}
		\item[\hskip \labelsep {\bfseries #1:}]
		\item \hspace{14pt}
	}{
		$ \hfill\blacksquare $
	\end{trivlist}
	\hfill\break
}
\newenvironment{solution}[1][Решение]{%
	\begin{trivlist}
		\item[\hskip \labelsep {\bfseries #1:}]
		\item \hspace{15pt}
	}{
	\end{trivlist}
}

\newenvironment{answer}[1][Ответ]{%
	\begin{trivlist}
		\item[\hskip \labelsep {\bfseries #1:}] \hskip \labelsep
	}{
	\end{trivlist}
	\hfill
}

\title{Дискретная математика} 
\date{\today}
\author{Сидоров Дмитрий}
\affil{Группа БПМИ 219}


\begin{document}
	\maketitle
	
	\section*{№1}
		
		Найдите коэффициент при в разложении многочлена на
		мономы.
		
		\begin{solution}
			Многочлен имеет степень 12, те состоит из 12 скобок. Заметим, что коэффициенты при каждом $x_1, x_2, \dots, x_6$ в каждой скобке $(x_1 + x_2 + \dots + x_6)$ равны 1. Тогда, выбирая по две скобки (т к степень каждого $x_i$ равна 2) для каждого x получим, что коэффициент равен $\frac{12!}{(2!)^6}$ (тк $\binom{12}{2} \cdot \binom{10}{2} \cdot \binom{8}{2} \cdot \binom{6}{2} \cdot \binom{4}{2} \cdot \binom{2}{2}= \frac{12!}{(2!)^6})$. Заметим, что 2! = 2, т е $\frac{12!}{(2!)^6} = \frac{12!}{2^6}$.
		\end{solution}
		
		\begin{answer}
			$ \frac{12!}{2^6}$
		\end{answer}

	\section*{№2}
	
		Есть 3 гвоздики, 4 розы и 5 тюльпанов. Сколькими способами можно составить букет из 7 цветов,
		используя имеющиеся цветы? (Цветы одного сорта считаем одинаковыми.)
		Ответом должно быть число в десятичной записи.
		
		\begin{solution}
			Рассмотрим букет из 7 цветов. Пусть в нём $x$ гвоздик, $y$ роз и $z$ тюльпанов. Тогда ответ на задачу - количество решений уравнения $x  + y + z = 7$, причём $0 \leq x \leq 3$, $0 \leq y \leq 4$, $0 \leq z \leq 5$ и одновременно $x, y, z$ не равны 0. Тогда рассмотрим количество букетов в зависимости от количества гвоздик в нём.
			
			1) $x = 0 \Rightarrow y + z = 7 \Rightarrow$ возможные варианты $(y, z$): (2, 5), (3, 4), (4, 3) - Итого 3 пары.
			
			2) $x = 1 \Rightarrow y + z = 6 \Rightarrow$ возможные варианты $(y, z$:) (1, 5), (2, 4), (3, 3), (4, 2) - Итого 4 пары.
			
			3) $x = 2 \Rightarrow y + z = 5 \Rightarrow$ возможные варианты $(y, z$): (0, 5), (1, 4), (2, 3), (3, 2), (4, 1) - Итого 5 пар.
			
			4) $x = 3 \Rightarrow y + z = 4 \Rightarrow$ возможные варианты $(y, z$): (0, 4), (1, 3), (2, 2), (3, 1), (4, 0) - Итого 5 пар.
			
			Таким образом, всго 3 + 4 + 5 + 5 = 17 решений $\Rightarrow$ 17 пособами можно составить букет из 7 цветов.
		\end{solution}
	
		\begin{answer}
			17
		\end{answer}
	
	\section*{№3}
	
		Сколько есть 6-элементных подмножеств множества чисел [15] = $\{1, 2, \dots, 15\}$, в которых любая
		пара чисел различается хотя бы на 2?
		Ответом должно быть число в десятичной записи.
		
		\begin{solution}
			Заметим, что если любая
			пара чисел различается хотя бы на 2, то в 6-элементном подмножестве нет пар чисел, которые  различаются на 0 (те одинаковых) или на 1 (те подряд идущих). Первое условие всегда выполняется, тк в 6-элементном подмножестве множества чисел [15] все числа различные. Посчитаем количество 6-элементных подмножеств множества чисел [15] = $\{1, 2, \dots, 15\}$, в которых нет подряд идущих чисел. Заметим, что каждое 6-элементное подмножество состот из 6 элементов множества чисел [15], те берётся ровно 6 чисел из множества чисел [15], а значит каждое такое подмножество задаётся как последовательность длины 15 из 0 и 1, где на $i$-ом месте стоит 0, если $i$ не входит в подмножество, и 1 иначе. Таким образом, каждое подмножество задаётся как последовательность из 6 единиц и 15 - 6 = 9 нулей, при этом, тк в подмножестве нет подряд идущих чисел, в последовательности нет подряд идущих 1. Таким образом, искомое количество подмножеств равно количеству таких последовательностей. Заметим, что можно расставить 6 единиц произвольным способом в последовательности длины 15 - 5 = 10 (на остальных 4 местах нули) и добавить ещё 5 нулей так, что никакие две единицы не будут стоять рядом, значит количество удовлетворяющих последовательностей равно $C_{10}^6 = \frac{10!}{6! \cdot 4!} = \frac{10 \cdot 9 \cdot 8 \cdot 7}{2 \cdot 3 \cdot 4} = \frac{10 \cdot 9 \cdot 7}{3} = 70 \cdot 3 = 210$
		\end{solution}
	
		\begin{answer}
			210
		\end{answer}
	
	\section*{№4}
	
		Найдите количество монотонных тотальных функций из [8] = $\{1, 2, \dots , 8\}$ в [12] = $\{1, 2, \dots, 12\}$.
		Функция f называется монотонной, если из x $\leq$ y следует f(x) $\leq$ f(y).
		
		\begin{solution}
			Если функция тотальна, то каждому из 8 элементов из [8] необходимо поставить в соответствие 1 элемент из [12].
			Заметим, что 8 элементов можно расположить в порядке неубывания единственным образом. Значит количество монотонных тотальных функций равно количеству способов выбрать 8 элементов из [12], при этом элементы из [12] могут повторяться, а значит количество способов равно $C_{8 + 12 - 1}^8 = C_{19}^8 = \frac{19!}{11! \cdot 8!}$
		\end{solution}
	
		\begin{answer}
			$\frac{19!}{11! \cdot 8!}$
		\end{answer}
		
	\section*{№5}
	
		Сколько различных слов (не обязательно осмысленных) можно получить, переставляя буквы в слове
		
		
		«ОБОРОНОСПОСОБНОСТЬ» так, чтобы никакие две буквы О не стояли рядом?
		
		\begin{solution}
			В слове ОБОРОНОСПОСОБНОСТЬ 18 букв, в нём 7 букв О. По условию никакие две буквы О не стоят рядом. Посчитаем количество способов расставить букву О. Заметим, что можно расставить буквы О произвольным способом в слове длины 18 - 6 = 12 и потом добавить в слово 6 мест так, чтобы никакие две буквы О не стояли рядом. Значит количество способов расставить буквы О $C_{12}^7$. После расстановки букв О останется ещё 11 букв, которые надо расставить, из них повторяются буквы Б (2 раза), Н (2 раза) и С (3 раза), остальные не повторяются, а значит количество способов расставить оставшиеся 11 букв $\frac{11!}{2!2!3!} = \frac{11!}{24}$, таким образом, всего различных слов $C_{12}^7 \cdot \frac{11!}{24} = \frac{12! \cdot 11!}{5! \cdot 7! \cdot 24}$
		\end{solution}
	
		\begin{answer}
			$\frac{12! \cdot 11!}{5! \cdot 7! \cdot 24}$
		\end{answer}
	
	\section*{№6}
	
		Сколько есть способов разместить 20 различных книг на 5 полках, если каждая полка может вместить все 20 книг? Размещения, различающиеся порядком книг на полках, считаются различными.
		
		\begin{solution}
			Количество искомых размещений - это количество решений уравнения $x_1 + x_2 + x_3 + x_4 + x_5 = 20$, где $x_i$ - количество книг на $i$ - ой полке, те это количество последовательностей из 20 шариков и 4 перегородок, те $\frac{24!}{4!20!} \cdot 20! = \frac{24!}{4!}$ (умножаем на 20!, тк важен порядок книг).
		\end{solution}
	
		\begin{answer}
			$\frac{24!}{4!}$
		\end{answer}
	
	\section*{№7}
	
		Найдите количество таких всюду определённых функций $f$ из 7-элементного множества A в себя,
		что $f\circ f = id_A$.
		Ответом должно быть число в десятичной записи.
		
		\begin{solution}
			Заметим, что если $f$ - отображение A в себя и $f\circ f = id_A$, то если $f(x) = a$, то $f(a) = f(f(x)) = x$. Заметим, что если $f(a) = b$ и $a \ne b$, то $f(b) = a$, причём $b \ne a$, те для каждого элемента, который не переходит в себя, существует "парный элемент", а значит количество элементов, которые не переходят в себя, чётно, а значит количество элементов, которые переходят в себя нечётно (тк всего 7 элементов). Таким образом, элементов, которые переходят в себя, может быть 1, 3, 5 или 7.
			Рассмотри 4 случая:
			
			1 элемент) Есть 7 способов выбрать элемент, который переходит в себя, остаётся 6 элементов. Посчитаем количесвто способов разбить 6 элементов на пары: Первую пару можно выбрать $C_6^2$ способами, вторую $C_4^2$, третью $C_2^2$, значит способов  $\frac{C_6^2 \cdot C_4^2 \cdot C_2^2}{3!}$ (делим на 3!, тк порядок не важен) = $\frac{15 \cdot 6}{6} = 15$, итого $7 \cdot 15 = 105$ функций.
			
			3 элемента) Есть $C_7^3 = \frac{ 5 \cdot 6 \cdot 7}{6} = 35$ способа выбрать 3 элемента, которые переходят в себя, остальные 4 разбиваются на пары 3 способами, итого $35 \cdot 3 = 105$ функций.
			
			5 элементов) Выбираем 5 элементов $C_7^5 = \frac{6 \cdot 7}{2} = 21$ способом, остальые 2 разбиваются на пары единственным способом, итого 21 функция.
			
			7 элементов) Все элементы множества А переходят в себя $\Rightarrow$ одна функция.
			
			Таким образом, всего 105 + 105 + 21 + 1 = 232 функций.
		\end{solution}
	
		\begin{answer}
			232
		\end{answer}
	
	\section*{№8}
	
		Найдите максимальное количество простых путей из заданной вершины s в заданную вершину t в
		ориентированном ациклическом графе на n вершинах. (Максимум берётся по всем ориентированным
		ациклическим графам с n вершинами и всем парам вершин s, t в каждом графе.)
		
		\begin{solution}
			Будем считать, что $n \geq 2$, иначе существует только один путь (из вершины в себя). Зафиксируем вершины $s$ и $t$ и рассмотрим оставшиеся $n - 2$ вершины (тк "максимум берётся по всем ориентированным
			ациклическим графам с n вершинами и всем парам вершин s, t в каждом графе"). Заметим, что если существует простой путь из $s$ в $t$, то он имеет вид $s v_1 v_2 \dots v_k t$. По условию граф ациклический, те в нём нет циклов длины более 1. Тогда докажем, что если существует путь $s v_1 v_2 \dots v_k t$, то не существует другой путь из $s$ в $t$, состоящий в точности из верших, входящих в путь $s v_1 v_2 \dots v_k t$, но в другом порядке. Пусть такой путь существует, путь имеет вид $s v_1 \dots v_i v_{i+1} \dots v_k t$ и существует путь $s v_1 \dots v_{i+1} v_k \dots v_m v_i \dots v_k t$ (обязательно найдутся такие вершины $v_i$ и $v_{i+1}$ такие, что $v_{i + 1}$ стоит левее $v_i$, иначе пути $s v_1 \dots v_i v_{i+1} \dots v_k t$ и $s v_1 \dots v_{i+1} v_k \dots v_m v_i \dots v_k t$ совпадают). Но тогда в исходном графе существует цикл $v_{i + 1} v_k \dots v_m v_i v_{i + 1}$, в котором более чем одна вершина, что невозможно по условию. Значит каждый путь из $s$ в $t$ определяется входящими в него верщинами, а всего таких путей столько, сколько подмножеств из $n -2$ вершин, те $2^{n-2}$ (каждую из $n - 2$ вершин либо "берём" в путь, либо "не берём").
		\end{solution}
	
		\begin{answer}
			$2^{n-2}$
		\end{answer}
	
	
	
	
	
	
	
	
	
	
	
	
\end{document}