\documentclass[a4paper, 16pt]{article}

\usepackage[utf8]{inputenc}

\usepackage[russian, english]{babel}
\usepackage{subfiles}
\usepackage[utf8]{inputenc}
\usepackage[T2A]{fontenc}
\usepackage{ucs}
\usepackage{textcomp}
\usepackage{array}
\usepackage{indentfirst}
\usepackage{amsmath}
\usepackage{amssymb}
\usepackage{enumerate}
\usepackage[margin=1.2cm]{geometry}
\usepackage{authblk}
\usepackage{tikz}
\usepackage{icomma}
\usepackage{gensymb}
\usepackage{graphicx}
\usepackage{mathtools} 
\usepackage[makeroom]{cancel}

\DeclareGraphicsExtensions{,.png,.jpg}

\graphicspath{{pictures/}}

\renewcommand{\baselinestretch}{1.4}

\renewcommand{\C}{\mathbb{C}}
\newcommand{\N} {\mathbb{N}}
\newcommand{\Q} {\mathbb{Q}}
\newcommand{\Z} {\mathbb{Z}}
\newcommand{\R} {\mathbb{R}}
\newcommand{\ord} {\mathop{\rm ord}}
\newcommand{\Ima}{\mathop{\rm Im}}
\newcommand{\rk}{\mathop{\rm rk}}

\renewcommand{\r}{\right}
\renewcommand{\l}{\left}
\renewcommand{\inf}{\infty}
\newcommand{\Sum}[2]{\overset{#2}{\underset{#1}{\sum}}}
\newcommand{\Lim}[2]{\lim\limits_{#1 \rightarrow #2}}
\newcommand\tab[1][1cm]{\hspace*{#1}}
\newcommand{\Mod}[1]{\ (\mathrm{mod}\ #1)}

\newcommand{\task}[1] {\noindent \textbf{Задача #1.} \hfill}
\newcommand{\note}[1] {\noindent \textbf{Примечание #1.} \hfill}
\newenvironment{proof}[1][Доказательство]{%
	\begin{trivlist}
		\item[\hskip \labelsep {\bfseries #1:}]
		\item \hspace{14pt}
	}{
		$ \hfill\blacksquare $
	\end{trivlist}
	\hfill\break
}
\newenvironment{solution}[1][Решение]{%
	\begin{trivlist}
		\item[\hskip \labelsep {\bfseries #1:}]
		\item \hspace{15pt}
	}{
	\end{trivlist}
}

\newenvironment{answer}[1][Ответ]{%
	\begin{trivlist}
		\item[\hskip \labelsep {\bfseries #1:}] \hskip \labelsep
	}{
	\end{trivlist}
	\hfill
}

\title{Дискретная математика} 
\date{\today}
\author{Сидоров Дмитрий}
\affil{Группа БПМИ 219}


\begin{document}
	\maketitle
	
	\section*{№1}
	
		Положительное целое число
		a
		чётно, но не делится на
		4
		. Покажите, что количество (положительных)
		чётных делителей
		a
		равно количеству (положительных) нечётных делителей
		a
		
		\begin{proof}
			Если $a$ - положительное целое чётное число, то $a = 2x$, и тк $a$ не делится на 4, то $x$ нечет. Тогда у $x$ нет чётных делителей (тк $x$ нечет), и при этом все делители $x$ являются делителями $a$. Тогда, тк $a = 2x$, все нечётные делители $a$ - это делители $x$. Тогда для каждого нечётного делителя $a$ $d$ существует число $2d$, которое является чётным делителем $a$, а значит количество (положительных)
			чётных делителей
			$a$
			равно количеству (положительных) нечётных делителей
			$a$.
		\end{proof}
	
	\section*{№2}
		Пусть
		$p$
		— простое число, большее 3. Докажите, что
		$p^2-1$ делится на 24.
		
		\begin{proof}
			$24 = 8 \cdot 3 = 2^3 \cdot 3$, $p^2-1 = (p-1)(p+1)$. Тк $p$
			— простое число, большее 3, то $p$ не делится на 2, те нечет, а значит $p-1$ и $p+1$ - чёт. Пусть ни $p-1$, ни $p+1$ не делится на 4. Тогда, тк $p$ - нечет, на 4 делится либо $p-2$, либо $p + 2$, но, тк $p$ - нечет, то и $p-2$, и $p + 2$ - нечет, противоречие, и значит либо $p-1$, либо $p+1$ делится на 4. Таким образом, $p^2-1 = (p-1)(p+1)$ делится на 8 (либо $p-1$, либо $p+1$ делится на 4, и оба числа чётные). Тк $p-1, p, p+1$  - 3 последовательных числа, то одно из них делится на 3. Тк $p$
			— простое число, большее 3, то оно не делится на 3, и значит либо $p-1$, либо $p+1$ делится на 3. Таким образом, $p^2-1 = (p-1)(p+1)$ делится на 3.
			
			 Таким образом,  $p^2-1 = (p-1)(p+1)$ делится на 3 и на 8, те делится на 24.
		\end{proof}
	
	\section*{№3}
	
		Докажите, что десятичная запись
		$3^{4000}$
		заканчивается на $\dots 0001$.
		
		\begin{proof}
			Если $3^{4000} - 1$ заканчивается на 4 нуля, те делится на 10000, то $3^{4000}$ заканчивается на $\dots 0001$. Заметим, что $10000 = 10^4  =2^4 \cdot 5^4$. Среди чисел от 0 до 9999 делятся на 2 $\frac{10000}{2} = 5000$ чисел, делятся на 5 $\frac{10000}{5} = 2000$ чисел, делятся на 2 и на 5 $\frac{10000}{10} = 1000$ чисел. Значит чисел от 0 до 9999, которые не делятся ни на 2, ни на 5, те взаимно простых с 10000, $10000 - 5000 - 2000 + 1000 = 4000$. Тогда по теореме Эйлера, тк 3 взаимно просто с 10000, $3^{4000} - 1$ делится на 10000, а значит $3^{4000}$ заканчивается на $\dots 0001$.
		\end{proof}
	
	\section*{№4}
	
		Докажите, что при любом нечетном положительном
		$n$
		число
		$2^{n!}$
		-
		1
		делится на
		$n$.
		
		\begin{proof}
			Тк $\varphi (n) \leq n$ и $ \varphi (n)  \in \Z$, то $\frac{n!}{\varphi (n) } = k \in Z$, значит $2^{n!} = 2 ^{k\cdot \varphi(n)} = (2 ^{k})^ {\varphi(n)}$. Тогда по теореме Эйлера, тк $n$ положительно и нечетно по условию и тогда $n$ и $2^k$ взаимно просты, то $(2 ^{k})^ {\varphi(n)} \equiv 1 \Mod{n}$, а значит число $(2 ^{k})^ {\varphi(n)} - 1 = $ 
			$2^{n!}$
			-
			1
			делится на
			$n$.
		\end{proof}
	
	\section*{№5}
	
		Если от некоторого трёхзначного числа отнять 6, то оно разделится на 7, если отнять 7, то оно
		разделится на 8, а если отнять 8, то оно разделится на 9. Найдите это число.
		
		\begin{solution}
			Из условия следует, что остаток от деления этого трёхзначного числа $x$ на 7 равен 6, на 8 равен 7, на 9 равен 8. Значит $x + 1$ делится на 7, 8, 9. Тогда $x + 1 = 7 \cdot 8 \cdot 9 k = 504 k, k \in Z$. Тк $x$ - трёхзначное число, то $x + 1 = 504, x = 503$
		\end{solution}
		
		\begin{answer}
			503
		\end{answer}
	
	\section*{№6}
	
		Найдите остаток при делении
		а)
		$19^{10}$
		на 66;
		б)
		$2^{2022}$
		на 105.
		
		\subsection*{a)}
			
			\begin{solution}
				$66 = 11 \cdot 6$. $19^{10} \equiv 1 ^ {10} = 1\Mod{6}$ (тк $1 = 19 - 6 \cdot 3$). При этом $19 ^ {10} = 19 ^ {11-1}\equiv 1 \Mod{11}$ по малой теореме Ферма, тк 11 - простое число и 19 не кратно 11. Таким образом, необходимо найти число на отрезке $[0; 65]$, которое даёт остаток 1 при делении на 6, и даёт остаток 1 при делении на 11, значит это число - 1. Значит остаток при делении $19^{10}$
				на 66 равен 1.
			\end{solution}
		
			\begin{answer}
				1
			\end{answer}
		
		\subsection*{b)}
		
			\begin{solution}
				$105 = 3 \cdot 5 \cdot 7$. $2^{2022} \equiv (-1)^{2022} = 1 \Mod{3}$, $2^{2022} \equiv (4)^{1011} \equiv(-1)^{1011} = -1 \equiv4 \Mod{5}$, $2^{2022} \equiv (8)^{674} \equiv (1)^{674} = 1 \Mod{7}$. Таким образом, необходимо найти число на отрезке $[0; 105]$, которое даёт остаток 1 при делении на 3 и 7, и даёт остаток 4 при делении на 5. Переберём целые положительные числа большие 0, дающие остаток 4 при делении на 5. 9, 14, 19, 24, 29, 34, 39, 44, 49, 54, 59, 64 - уд. Значит число 64 даёт остаток 1 при делении на 3 и 7, и даёт остаток 4 при делении на 5, и тогда остаток при делении
				$2^{2022}$
				на 105 равен 64.
			\end{solution}
		
			\begin{answer}
				64
			\end{answer}
	
	
	\section*{№7}
	
		При каких целых
		$n$
		число
		$a_n$
		=
		$n^2$
		+ $3n$
		+ 1
		делится на 55?
		
		\begin{solution}
			$a$ делится на 55, если делится на 5 и на 11
			
			1) $a = n^2 + 3n + 1\ \vdots\ 5 \Leftrightarrow n^2 + 3n + 1 - 5n \ \vdots\ 5 \Leftrightarrow  n^2  -2n + 1 = (n-1)^2\ \vdots\ 5$
			
			2) $a = n^2 + 3n + 1\ \vdots\ 11 \Leftrightarrow n^2 + 3n + 1 - 11n \ \vdots\ 11 \Leftrightarrow  n^2  -8n + 1 + 11\ \vdots\ 11 \Leftrightarrow   n^2  -8n + 12\ \vdots\ 11 \Leftrightarrow  (n-6)(n-2)\ \vdots\ 11$
			
			Таким образом, если $a$ делится на 55, то $n$ при делении на 5 должно давать остаток 1, а при делении на 11 2 или 6. Тогда среди чисел вида $11m + 2$ и $11m + 6\ m\in Z$ необходимо найти те, которые дают остаток 1 при делении на 5. По китайской теореме об остатках, тк 5 и 11 взаимно простые, то в промежутке от 0 до 54 существует единственное число $x$, для которого $x \equiv 2  \Mod {11}$ и $x \equiv 1 \Mod 5$ $(x \equiv 6  \Mod {11}$ и $x \equiv 1 \Mod 5$). Тогда в первом случае $x$ - одно из чисел 2, 13, 24, 35, 46, значит $x = 46$. Во втором случае $x$ - одно из чисел 6, 17, 28, 39, 50, значит $x = 6$. Тогда $a$ делится на 55 при $n = 55k + 6$, $n = 55k + 46, k \in \Z$.
		\end{solution}
	
		\begin{answer}
			$n = 55k + 6$, $n = 55k + 46, k\in \Z$
		\end{answer}
	
	\section*{№8}
	
		Докажите, что числитель несократимой дроби, равной
		$\frac{1}{1} + \frac{1}{2} + \dots + \frac{1}{p-1}$, делится на
		$p$
		для любого
		простого
		$p >2$.
		
		\begin{proof}
			Тк $p > 2$ и является простым, то $p$ нечет, а значит в сумме чётное количество дробей ($p - 1$ - чёт), тогда 
			$\frac{1}{1} + \frac{1}{2} + \dots + \frac{1}{p-1} = \frac{1}{1} + \frac{1}{p-1} + \frac{1}{2} + \frac{1}{p-2} + \dots = \frac{p}{p-1} + \frac{p}{2(p-2)} + \dots = p \left(\frac{1}{p-1} + \frac{1}{2(p-2)} + \dots\right)$. Тк $p > 2$ и является простым и все числа в знаменателе меньше $p$, то после приведения $p\left(\frac{1}{p-1} + \frac{1}{2(p-2)} + \dots\right)$ к общему знаменателю $p$ в числителе не сократится, и числитель несократимой дроби, полученной при приведении $p \left(\frac{1}{p-1} + \frac{1}{2(p-2)} + \dots\right)$ к общему знаменателю, делится на $p$, а значит числитель несократимой дроби, равной
			$\frac{1}{1} + \frac{1}{2} + \dots + \frac{1}{p-1}$, делится на
			$p$
			для любого
			простого
			$p >2$.
		\end{proof}
		
		
	
	
	
	
	
	
	
\end{document}