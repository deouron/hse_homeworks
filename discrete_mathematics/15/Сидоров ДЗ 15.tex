\documentclass[a4paper, 16pt]{article}

\usepackage[utf8]{inputenc}

\usepackage[russian, english]{babel}
\usepackage{subfiles}
\usepackage[utf8]{inputenc}
\usepackage[T2A]{fontenc}
\usepackage{ucs}
\usepackage{textcomp}
\usepackage{array}
\usepackage{indentfirst}
\usepackage{amsmath}
\usepackage{amssymb}
\usepackage{enumerate}
\usepackage[margin=1.2cm]{geometry}
\usepackage{authblk}
\usepackage{tikz}
\usepackage{icomma}
\usepackage{gensymb}
\usepackage{graphicx}

\DeclareGraphicsExtensions{,.png,.jpg}

\graphicspath{{pictures/}}

\renewcommand{\baselinestretch}{1.4}

\renewcommand{\C}{\mathbb{C}}
\newcommand{\N} {\mathbb{N}}
\newcommand{\Q} {\mathbb{Q}}
\newcommand{\Z} {\mathbb{Z}}
\newcommand{\R} {\mathbb{R}}
\newcommand{\ord} {\mathop{\rm ord}}
\newcommand{\Ima}{\mathop{\rm Im}}
\newcommand{\rk}{\mathop{\rm rk}}

\renewcommand{\r}{\right}
\renewcommand{\l}{\left}
\renewcommand{\inf}{\infty}
\newcommand{\Sum}[2]{\overset{#2}{\underset{#1}{\sum}}}
\newcommand{\Lim}[2]{\lim\limits_{#1 \rightarrow #2}}
\newcommand\tab[1][1cm]{\hspace*{#1}}

\newcommand{\task}[1] {\noindent \textbf{Задача #1.} \hfill}
\newcommand{\note}[1] {\noindent \textbf{Примечание #1.} \hfill}
\newenvironment{proof}[1][Доказательство]{%
	\begin{trivlist}
		\item[\hskip \labelsep {\bfseries #1:}]
		\item \hspace{14pt}
	}{
		$ \hfill\blacksquare $
	\end{trivlist}
	\hfill\break
}
\newenvironment{solution}[1][Решение]{%
	\begin{trivlist}
		\item[\hskip \labelsep {\bfseries #1:}]
		\item \hspace{15pt}
	}{
	\end{trivlist}
}

\newenvironment{answer}[1][Ответ]{%
	\begin{trivlist}
		\item[\hskip \labelsep {\bfseries #1:}] \hskip \labelsep
	}{
	\end{trivlist}
	\hfill
}

\title{Дискретная математика} 
\date{\today}
\author{Сидоров Дмитрий}
\affil{Группа БПМИ 219}


\begin{document}
	\maketitle
	
	\section*{№1}
	
		Вероятностное пространство: последовательности (x1, x2) длины 2, состоящие из целых чисел в
		диапазоне от 0 до 9. Все исходы равновозможны. Найдите вероятность события «$x_1 \ne x_2$». Ответ
		привести в виде числа (обыкновенная дробь, числитель и знаменатель записаны в десятичной системе).
		
		\begin{solution}
			Найдём вероятность события $x_1 = x_2$, тогда вероятность события $x_1 \ne x_2$ равна 1 - вероятность события $x_1 = x_2$. Заметим, что всего существует 10 $\cdot$ 10 = 100 различных последовательностей длины 2, состоящих из целых чисел в
			диапазоне от 0 до 9. При этом благоприятных исходов (те исходов, в которых $x_1 = x_2$) 10 штук ((0,0), (1,1), $\dots$, (9,9)). А значит вероятность события $x_1 = x_2$ равна $\frac{10}{100} = \frac{1}{10}$, а вероятность события $x_1 \ne x_2$ равна 1 - $\frac{1}{10} = \frac{9}{10}$.
		\end{solution}
	
		\begin{answer}
			$\frac{9}{10}$
		\end{answer}
	
	\section*{№2}
	
		Вероятностное пространство: последовательности (x1, x2, x3) длины 3, состоящие из целых чисел в
		диапазоне от 1 до 6. Все исходы равновозможны. Какова вероятность события «все числа в последовательности разные»? Ответ привести в виде числа (обыкновенная дробь, числитель и знаменатель
		записаны в десятичной системе).
		
		\begin{solution}
			Всего возможных исходов $6^3$. Посчитаем число благоприятных исходов. $x_1$ может быть любым числом от 1 до 6, те 6 вариантов. Тогда $x_2$ может быть любым из 5 чисел, и тогда $x_3$ любым из 4. Итого вероятность события «все числа в последовательности разные» равна $\frac{6 \cdot 5 \cdot 4}{6^3} = \frac{20}{36} = \frac{5}{9}$.
		\end{solution}
	
		\begin{answer}
			$\frac{5}{9}$
		\end{answer}
	
	\section*{№3}
	
		Вероятностное пространство: целые числа в диапазоне от 1000 до 9999. Все исходы равновозможны.
		Найдите вероятность события «сумма цифр числа равна 8». Больше эта вероятность 1/100 или меньше?
		
		\begin{solution}
			Всего целых чисел в диапазоне от 1000 до 9999 9000 штук. Обозначим цифры числа как $a \geq 1, b, c, d \geq 0; \ a, b, c, d \in Z$ (число имеет вид $\overline{abcd}$). Если сумма чисел равна 8, то выполняется $a + b + c + d = 8$. Если $a \geq 1$, то можно перейти к уравнению $a + b + c + d = 7$ и при этом $a, b, c, d \geq 0; \ a, b, c, d \in Z$. Тк дополнительно известно, что $a, b, c, d \leq 9$ и 9 > 7, количество решений уравнения равно $C_{7 + 4 - 1}^7 = C_{10}^7 = \frac{10 \cdot 9 \cdot 8}{6} = 120$. Значит существует 120 целых чисел в диапазоне от 1000 до 9999, сумма цифр которых равны 8. Тогда искомая вероятность равна $\frac{120}{9000} = \frac{1}{75} > \frac{1}{100}$
		\end{solution}
	
		\begin{answer}
			$\frac{1}{75} > \frac{1}{100}$
		\end{answer}
	
	\section*{№4}
	
		Вероятностное пространство: последовательности (x1, x2, x3, x4) длины 4, состоящие из целых чисел в
		диапазоне от 0 до 2. Все исходы равновозможны. Найдите вероятность события «в последовательности
		встречаются и 0, и 1, и 2». Ответ привести в виде числа (обыкновенная дробь, числитель и знаменатель
		записаны в десятичной системе).
		
		\begin{solution}
			Всего возможных исходов $3^4$. Заметим, что если каждый $x_i$ может принимать одно из 3 значений (0, 1 или 2), а длина последовательности 4, то по приницпу Дирихле найдется такое число (0, 1 или 2), которое повторяется (в последовательности, в которой встречаются и 0, и 1, и 2). Тогда для повторяющегося числа можно выбрать 2 из 4 мест, те $C_4^2 = \frac{3 \cdot 4}{2} = 6$ вариантов, повторяться может одно из 3 чисел, и при этом остаётся ещё 2 места для 2 чисел, эти числа можно расставить 2 способами. Итого всего $6 \cdot 3 \cdot 2 = 36 = 3^2 \cdot 4$ благоприятных исхода. А значит вероятность события «в последовательности
			встречаются и 0, и 1, и 2» равна $\frac{3^2 \cdot 4}{3^4} = \frac{4}{9}$
		\end{solution}
	
		\begin{answer}
			$\frac{4}{9}$
		\end{answer}
	
	\section*{№5}
	
		 Докажите, что случайный граф на n вершинах связен.
		 
		Точная формулировка: исходы — все неориентированные графы без кратных ребер с одним и тем же
		множеством вершин, в котором n элементов. Все исходы равновозможны. Нужно доказать, что вероятность события «граф несвязный» стремится к нулю при n → $\inf$
		
		\begin{proof}
			%Пусть есть $n$ вершин. Тогда, тк исходы равновозможные, то для каждого графа на $n$ вершинах между любой парой вершин ребро проводится (или не проводится) с вероятностью $\frac{1}{2}$. Если граф несвязный, то в нё есть хотя бы 2 две компоненты связности. Заметим, что всегда найдётся компонента связности, в которой $x \leq \frac{n}{2}$ вершин (тк иначе в каждой компоненте хотя бы $\frac{n}{2} + 1$ вершина, компонент не меньше 2, тогда в графе хотя бы $n + 2$ вершины - противоречие). Рассмотрим такую компоненту с $x \leq \frac{n}{2}$ вершиной. Для этой компоненты можно выбрать вершины $C_n^x = \frac{n!}{x! (n - x)!}$ способами. При этом каждая из $x$ вершин соединена только с вершинами из своей компоненты связности и не соединена с вершинами из другой компоненты. Вероятность того, что ни одна из вершин этой компоненты не соединена с вершиной из других компонент (а таких вершин $n - x$), равна $\frac{1}{2} ^ {(n-x) \cdot x}$ (тк вероятность наличия ребра между вершинами равна $\frac{1}{2}$, а всего можно провести $(n-x) \cdot x$ ребро). При этом можно выбрать вершины $\frac{n!}{x! (n - x)!}$ способами, и $x \leq \frac{n}{2}$, а значит вероятность, что граф не связен, не больше $\frac{n!}{x! (n - x)!} \cdot$ $\frac{1}{2} ^ {(n-x) \cdot x}$. Тогда искомая вероятность не больше $\frac{n!}{\frac{n}{2}! \cdot \frac{числит}{знамен}}$
			
			Граф точно связен, если у каждой пары вершин есть общий сосед, либо они соединены ребром. Найдем вероятность события «граф несвязный». Пусть в графе $n$ вершин. Тогда, тк исходы равновозможные, то для каждого графа на $n$ вершинах между любой парой вершин ребро проводится (или не проводится) с вероятностью $\frac{1}{2}$. Если граф несвязный, то найдутся хотя бы две вершины, которые не будут соединены ребром, и для которых не найдётся третья вершина такая, что обе вершины соединены с этой третьей. Зафиксируем такие две вершины. Тогда для любой третьей вершины вероятность, что эта вершина не соединена одновременно с двумя, равна $1 - \frac{1}{2} \cdot \frac{1}{2} = \frac{3}{4}$. Вершин, за исключением двух зафиксированных, $n - 2$. Заметим, что способов выбрать две вершины $C_n^2 = \frac{n(n-1)}{2}$. Значит вероятность того, что граф несвязный не больше $\frac{n(n-1)}{2} \cdot (\frac{3}{4})^{n-2} = \frac{n(n-1)3^n}{4^n} \cdot \frac{8}{9} \to 0$ при $n \to \inf$ (тк $\lim\limits_{n \to \inf} \frac{n}{4^n} = \lim\limits_{n \to \inf} \frac{n - 1}{4^n} = \lim\limits_{n \to \inf} \frac{3^n}{4^n} = 0$). Таким образом, вероятность события «граф несвязный» стремится к 0 при $n \to \inf$, а значит случайный граф на $n$ вершинах связен.
		\end{proof}
	
	\section*{№6}
	
		Вероятностное пространство: перестановки чисел от 1 до 36. Все исходы равновозможны. Найдите
		вероятность события «наибольшее среди первых 10 чисел в перестановке больше наибольшего среди
		последних 10 чисел». Ответ привести в виде числа (обыкновенная дробь, числитель и знаменатель
		записаны в десятичной системе).
		
		\begin{solution}
			Всего возможно 36! перестановок. Посчитаем количество перестановок, удовлетворяющих условию. Пронумеруем все числа в перестановке от 1 до 36. Заметим, что числа на позициях от 11 до 26 вкл (16 чисел) можно расставить $36 \cdot 35 \cdot \ \dots \ \cdot 21$ способами. Теперь среди оставшихся 20 чисел есть ровно одно наибольшее, и если оно будет среди первых 10 чисел, то будет выполняться условие «наибольшее среди первых 10 чисел в перестановке больше наибольшего среди
			последних 10 чисел». Обозначим это число за $x$. Для $x$ есть 10 мест среди первых 10 чисел. При этом остальные 9 среди первых 10 и все 10, среди последних 10, могут быть любыми. Тогда условию «наибольшее среди первых 10 чисел в перестановке больше наибольшего среди
			последних 10 чисел» будет удовлетворять $(36 \cdot 35 \cdot \ \dots \ \cdot 21) \cdot 10 \cdot (19 \cdot 18 \cdot \ \dots \ 2 \cdot 1)$ перестановок. Тогда искомая вероятность равна $\frac{(36 \cdot 35 \cdot \ \dots \ \cdot 21) \cdot 10 \cdot (19 \cdot 18 \cdot \ \dots \ 2 \cdot 1)}{36!} = \frac{10}{20} = \frac{1}{2}$
		\end{solution}
	
		\begin{answer}
			$\frac{1}{2}$
		\end{answer}
	
	\section*{№7}
	
		Вероятностное пространство: двоичные слова длины 21. Все исходы равновозможны. Найдите вероятность события «на первых 10 позициях стоит меньше единиц, чем на последних 11». Ответ привести
		в виде числа (обыкновенная дробь, числитель и знаменатель записаны в десятичной системе).
		
		\begin{solution}
			Рассмотрим два случая: количество единиц на первых 10 позициях равно количеству единиц на последних 10 позициях, либо не равно.
			
			1) Равно
			
			Заметим, что в этом случае вероятность события «на первых 10 позициях стоит меньше единиц, чем на последних 11» равна вероятности события «на 11-ом месте  стоит единица», те $\frac{1}{2}$ (тк слово двоичное).
			
			2) Не равно
			
			В этом случае количество единиц на первых 10 позициях либо меньше количества единиц на последних 10 позициях, либо больше. Заметим, что эти случаи симметричны, те их вероятности равны. Заметим, что если на первых 10 позициях количество единиц меньше количества единиц на последних 10 позициях, то на первых 10 позициях стоит меньше единиц, чем на последних 11 (аналогично, если больше на первых 10, то не меньше, чем на последних 11). Таким образом, вероятность события «на первых 10 позициях стоит меньше единиц, чем на последних 11» в этом случае равна $\frac{1}{2}$
			
			В обоих случаях вероятность равна $\frac{1}{2}$, а значит искомая вероятность равна $\frac{1}{2}$
		\end{solution}
	
		\begin{answer}
			$\frac{1}{2}$
		\end{answer}
		

	
	
	
	
	
	
	
	
	
	
	
\end{document}