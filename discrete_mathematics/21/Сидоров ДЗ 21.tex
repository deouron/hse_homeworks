\documentclass[a4paper, 16pt]{article}

\usepackage[utf8]{inputenc}

\usepackage[russian, english]{babel}
\usepackage{subfiles}
\usepackage[utf8]{inputenc}
\usepackage[T2A]{fontenc}
\usepackage{ucs}
\usepackage{textcomp}
\usepackage{array}
\usepackage{indentfirst}
\usepackage{amsmath}
\usepackage{amssymb}
\usepackage{enumerate}
\usepackage[margin=1.2cm]{geometry}
\usepackage{authblk}
\usepackage{tikz}
\usepackage{icomma}
\usepackage{gensymb}
\usepackage{graphicx}
\usepackage{mathtools} 
\usepackage[makeroom]{cancel}

\DeclareGraphicsExtensions{,.png,.jpg}

\graphicspath{{pictures/}}

\renewcommand{\baselinestretch}{1.4}

\renewcommand{\C}{\mathbb{C}}
\newcommand{\N} {\mathbb{N}}
\newcommand{\Q} {\mathbb{Q}}
\newcommand{\Z} {\mathbb{Z}}
\newcommand{\R} {\mathbb{R}}
\newcommand{\ord} {\mathop{\rm ord}}
\newcommand{\Ima}{\mathop{\rm Im}}
\newcommand{\rk}{\mathop{\rm rk}}

\renewcommand{\r}{\right}
\renewcommand{\l}{\left}
\renewcommand{\inf}{\infty}
\newcommand{\Sum}[2]{\overset{#2}{\underset{#1}{\sum}}}
\newcommand{\Lim}[2]{\lim\limits_{#1 \rightarrow #2}}
\newcommand\tab[1][1cm]{\hspace*{#1}}
\newcommand{\Mod}[1]{\ (\mathrm{mod}\ #1)}

\newcommand{\task}[1] {\noindent \textbf{Задача #1.} \hfill}
\newcommand{\note}[1] {\noindent \textbf{Примечание #1.} \hfill}
\newenvironment{proof}[1][Доказательство]{%
	\begin{trivlist}
		\item[\hskip \labelsep {\bfseries #1:}]
		\item \hspace{14pt}
	}{
		$ \hfill\blacksquare $
	\end{trivlist}
	\hfill\break
}
\newenvironment{solution}[1][Решение]{%
	\begin{trivlist}
		\item[\hskip \labelsep {\bfseries #1:}]
		\item \hspace{15pt}
	}{
	\end{trivlist}
}

\newenvironment{answer}[1][Ответ]{%
	\begin{trivlist}
		\item[\hskip \labelsep {\bfseries #1:}] \hskip \labelsep
	}{
	\end{trivlist}
	\hfill
}

\title{Дискретная математика} 
\date{\today}
\author{Сидоров Дмитрий}
\affil{Группа БПМИ 219}


\begin{document}
	\maketitle
	
	\section*{№1}
	
		Числа Фибоначчи задаются правилами
		$F_0 = 1; F_1 = 1; F_{n+2} = F_{n+1} + F_n$ для всех $n \geq 2$. Докажите,
		что для любого
		$n$
		числа
		$F_n$
		и
		$F_{n+1} $
		взаимно просты.
		
		\begin{proof}
			Докажем, что 		$F_n$
			и
			$F_{n+1} $ взаимно просты, используя принцип мат индукции.
			
			1) База. Для $n = 0:$ $F_0 = 1, \  F_1 = 1 \Rightarrow$ НОД$(F_0; F_{1}) = 1 \Rightarrow F_0$ и $F_{1}$ взаимно просты.
			
			2) Пусть $F_k$ и $F_{k + 1}$ взаимно просты. Тогда НОД($F_k$; $F_{k + 1}$) = 1.
			
			3) Пусть $n = k + 1$. Тогда НОД($F_{k+1}$; $F_{k + 2}$) = НОД($F_{k  +1}$; $F_k + F_{k + 1}$) = НОД($F_{k  +1}$; $F_k$) = 1 по предположению индукции $\Rightarrow$ $F_n$
			и
			$F_{n+1} $ взаимно просты по принципу мат индукции.
		\end{proof}
	
	\section*{№2}
	
		Решите систему сравнений
		$ \\ x \equiv 3 \Mod{15} \ (1) \\ x \equiv 4 \Mod{21} \ (2)\\  x \equiv 5 \Mod{35} \ (3)$
		
		\begin{solution}
			Из (1) следует, что $x = 3 + 15k, \ k \in \Z$.	Из (2) следует, что $x = 4 + 21d, \ d \in \Z$. Тогда $3 + 15k = 4 + 21d \Rightarrow 15k = 1 + 21d$. Заметим, что левая часть уравнения делится на 3, а правая при делении на 3 даёт остаток 1, а значит уравнение не умеет решений $\Rightarrow$ система сравнений не имеет решений.
		\end{solution}
	
		\begin{answer}
			решений нет
		\end{answer}
	
	\section*{№3}
	
		Известно, что
		$a^{12} + b^{12} + c^{12} + d^{12} + e^{12} + f^{12}$
		делится на 13. Докажите, что
		$abcdef$
		делится на
		$13^6$
		.
		Здесь
		$a,b,c,d,e,f$
		— целые числа.
	
		\begin{proof}
			По малой теореме Ферма, тк 13 - простое число, $x^{12} \equiv 1 \Mod{13} \ \forall x$, не делящемся на $p$ (и $x^{12} \equiv 0 \Mod{13} \ \forall x$,  делящемся на $p$). Таким образом, остаток при делении $a^{12} + b^{12} + c^{12} + d^{12} + e^{12} + f^{12}$ на 13 - это целое число от 0 до 6, и равен 0 только в том случае, когда остаток при делении каждого слагамого на 13 равен 0. Значит, тк $a^{12} + b^{12} + c^{12} + d^{12} + e^{12} + f^{12}$
			делится на 13 по условию, то  статок при делении каждого слагамого на 13 равен 0, значит $a, b, c, d, e, f$ делятся на 13, и тогда $abcdef$
			делится на
			$13^6$, тк каждый множитель делится на 13.
		\end{proof}
	
	\section*{№4}
	
		Вычислите остаток от деления
		$1^5+ 2^5+ \dots+ 2022^5$
		на 11.
		
		\begin{solution}
			Каждое число $x$ от 1 до 2022 можно представить в виде $x = 11k + r, \ k, r \in \Z$, где $k$ - целая часть при делении $x$ на 11, $r$ - остаток. Тогда каждое слагаемое можно представить в виде $x^5 = (11k + r)^5$ и по биному Ньютона $x^5 = (11k + r)^5 = (11k)^5 + 5(11k^4)r + 10(11k)^3r^2+10(11k)^2r^3+5(11k)r^4+r^5$, те остаток при делении $x^5$ на 11 равен остатку при делении $r^5$ на 11, и остаток не зависит от $k$ (1). При этом, тк $1 \equiv -10, 2 \equiv -9,  3 \equiv -8,  4 \equiv -7,  5 \equiv -6 \Mod{11}$ и числа вида $(11d)^5$ делятся на 11, то сумма $1^5 + 2^5 + \dots + 10^5 + 11^5$ делится на 11 (тк остаток будет 0), и с учётом (1) сумма $1^5 + 2^5 + \dots + 2013^5$ делится на 11 (2013 делится на 11). Значит остаток от деления
			$1^5+ 2^5+ \dots+ 2022^5$
			на 11 совпадает с остатком от деления
			$2014^5+ 2015^5+ \dots+ 2022^5$
			на 11. Заметим, что остаток от деления
			$2014^5+ 2015^5+ \dots+ 2022^5 + 2023^5 + 2024^5$
			на 11 равен 0 (тк 2024 делится на 11), а значит остаток от деления
			$2014^5+ 2015^5+ \dots+ 2022^5$
			на 11 равен 0 - (остаток от деления $2023^5 + 2024^5$ на 11). $2024^5$ делится на 11, а $2023^5 = (183 \cdot 11 + 10)^5 \Rightarrow 2023^5 \equiv 10^5 \equiv (-1)^5 = -1 \Mod{11} \Rightarrow$ остаток от деления
			$2014^5+ 2015^5+ \dots+ 2022^5$
			на 11 равен 0 - (-1) = 1 $\Rightarrow$ искомый остаток равен 1.
		\end{solution}
	
		\begin{answer}
			1
		\end{answer}
	
	\section*{№5}
	
		Решите уравнение $\varphi(x) = x / 4$
		
		\begin{solution}
			 Пусть $x = p_1^{\alpha_1} \cdot p_2^{\alpha_2} \cdot \ \dots \ \cdot p_n^{\alpha_n}, \ \alpha_i > 0, p_i$ - различные простые. Тогда $\varphi(x) = x \prod \limits_{i = 1}^{n} \left(1 - \frac{1}{p_i}\right) = \frac{x}{4}$. Тогда $\prod \limits_{i = 1}^{n} \left(1 - \frac{1}{p_i}\right) = \frac{1}{4} = \frac{(p_1 - 1)(p_2 - 1)\dots(p_n-1)}{p_1p_2\dots p_n}$. Заметим, что знаменатель этой дроби либо нечет, либо степень 2 в знаменателе этой дроби не больше 1 (тк 2 - единственное чётное простое число, может входить, может не входить, может входить и сократиться). Таким образом, $\frac{(p_1 - 1)(p_2 - 1)\dots(p_n-1)}{p_1p_2\dots p_n}$ не может равняться $\frac{1}{4}$ при любых $p$, а значит  уравнение $\varphi(x) = x / 4$ не имеет решений.
		\end{solution}
	
		\begin{answer}
			решений нет
		\end{answer}
	
	\section*{№6}
	
		Функция
		$f$
		из множества целых чисел в множество целых чисел сопоставляет числу
		$x$
		наименьшее
		простое число, которое больше
		$x^2$
		. Докажите, что если множество целых чисел
		$X$
		конечное, то и полный
		прообраз этого множества
		$f^{-1}
		(
		X
		)$
		конечен.
		
		\begin{proof}
			Пусть функция сопоставляет числу
			$x$
			наименьшее
			простое число $p$, которое больше
			$x^2$. Рассмотрим прообраз $p$. Прообраз $p$ состоит из таких $x$, для которых $f(x) = x^2 < p$. Значит прообраз $p$ состоит из таких $x$, для которых выполняется $|x| < \sqrt{p}$. Тк $p$ - простое число, то неравенству $|x| < \sqrt{p}$ удовлетворяет конечное число целых чисел, а значит прообраз $p$ конечен. Таким образом, для любого простого $p$ количество элементов в прообразове $p$ конечно, а значит полный
			прообраз множества
			$f^{-1}
			(
			X
			)$
			конечен.
		\end{proof}
	
	\section*{№7}
	
		Докажите, что для всякого
		$n$
		существует такая арифметическая прогрессия
		$a_i = a_0 + id, 0 \leq i < n$,
		что числа
		$a_0, \dots, a_{n-1}$
		попарно взаимно просты.
		
		\begin{proof}
			Будем считать, что $n > 2$, тк для $n = 1$ и $n = 2$ условие выполняется (в первом случае последовательность состоит из одного числа, т е, например, последовательность {1} удовлетворяет условию, во втором случае, например, подходит последовательность {1, 2}). Пусть $a_0 = 1$, а $d$ равняется произведению $n$ различных простых чисел $2 \cdot 3 \cdot 5 \cdot 7 \cdot \dots$ . Докажем, что в последовательности $1, 1 + d, \dots, 1 + (n-1)d$ числа
			$a_0, \dots, a_{n-1}$
			попарно взаимно просты. Заметим, что все члены прогрессии при делении на простое число, которое является множителем $d$, дают остаток 1, те не делятся на него, значит нет пары членов последовательности, в которой оба члена делились бы на простое число из разложения $d$. При этом каждое $0 \leq i < n$ либо является простым, либо раскладывает на простые множители, и тк $d$ равняется произведению $n$ различных простых чисел, то в любом из двух случаях все простые множители $i$ входят в разложение $d$ (наибольший множитель-простое число в $d$ не меньше $n - 1$), а значит все члены последовательности являются простыми (кроме $a_0 = 1$), те все члены последовательности попарно взаимно просты.
		\end{proof}
		
	
	
	
	
	
	
\end{document}