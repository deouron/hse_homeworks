\documentclass[a4paper, 16pt]{article}

\usepackage[utf8]{inputenc}

\usepackage[russian, english]{babel}
\usepackage{subfiles}
\usepackage[utf8]{inputenc}
\usepackage[T2A]{fontenc}
\usepackage{ucs}
\usepackage{textcomp}
\usepackage{array}
\usepackage{indentfirst}
\usepackage{amsmath}
\usepackage{amssymb}
\usepackage{enumerate}
\usepackage[margin=1.2cm]{geometry}
\usepackage{authblk}
\usepackage{tikz}
\usepackage{icomma}
\usepackage{gensymb}
\usepackage{graphicx}

\DeclareGraphicsExtensions{,.png,.jpg}

\graphicspath{{pictures/}}

\renewcommand{\baselinestretch}{1.4}

\renewcommand{\C}{\mathbb{C}}
\newcommand{\N} {\mathbb{N}}
\newcommand{\Q} {\mathbb{Q}}
\newcommand{\Z} {\mathbb{Z}}
\newcommand{\R} {\mathbb{R}}
\newcommand{\ord} {\mathop{\rm ord}}
\newcommand{\Ima}{\mathop{\rm Im}}
\newcommand{\rk}{\mathop{\rm rk}}

\renewcommand{\r}{\right}
\renewcommand{\l}{\left}
\renewcommand{\inf}{\infty}
\newcommand{\Sum}[2]{\overset{#2}{\underset{#1}{\sum}}}
\newcommand{\Lim}[2]{\lim\limits_{#1 \rightarrow #2}}
\newcommand\tab[1][1cm]{\hspace*{#1}}

\newcommand{\task}[1] {\noindent \textbf{Задача #1.} \hfill}
\newcommand{\note}[1] {\noindent \textbf{Примечание #1.} \hfill}
\newenvironment{proof}[1][Доказательство]{%
	\begin{trivlist}
		\item[\hskip \labelsep {\bfseries #1:}]
		\item \hspace{14pt}
	}{
		$ \hfill\blacksquare $
	\end{trivlist}
	\hfill\break
}
\newenvironment{solution}[1][Решение]{%
	\begin{trivlist}
		\item[\hskip \labelsep {\bfseries #1:}]
		\item \hspace{15pt}
	}{
	\end{trivlist}
}

\newenvironment{answer}[1][Ответ]{%
	\begin{trivlist}
		\item[\hskip \labelsep {\bfseries #1:}] \hskip \labelsep
	}{
	\end{trivlist}
	\hfill
}

\title{Дискретная математика} 
\date{\today}
\author{Сидоров Дмитрий}
\affil{Группа БПМИ 219}


\begin{document}
	\maketitle
	
	\section*{№1}
	
	Рассмотрим на множестве
	$\R$
	бинарное отношение
	$R(x, y)$
	, означающее, что $\frac{x}{y}>0$
	. Чему равно
	$R$
	◦
	$R$?
	
	\begin{solution}
		По определению композиции $x$($R$◦$R$) $y \Leftrightarrow \exists z: (x, z) \in R, (z, y) \in R$. Значит $\frac{x}{z} > 0$ и $\frac{z}{y} > 0$. Заметим, что $\frac{x}{z} > 0$ тогда и только тогда, когда $x$ и $z$ одного знака. Аналогично для $\frac{z}{y} > 0$ $y$ и $z$ одного знака, а значит $R$
		◦
		$R$ равно $\frac{x}{y}>0$.
	\end{solution}
	\begin{answer}
		$\frac{x}{y}>0$
	\end{answer}

	\section*{№2}
	
		Бинарное отношению
		R
		$\subset$ $\{
			a, b, c, d, e, f, g, h
		\} \times \{
			1
			,
			2
			,
			3
			,
			4
			,
			5
			,
			6
			,
			7
			,
			8
		\}$
		состоит из пар
		$\{
			(
			a,
			1)
			,
			(
			b,
			2)
			,
			(
			c,
			4)
			,
			(
			d,
			8)
			,
			(
			e,
			8)
			,
			$ $(
			f,
			8)
			,
			(
			g,
			8)
			,
			(
			h,
			8)
		\}$
		.
		Найдите количество элементов в отношениях
		$R^T$
		◦
		$R$
		и
		$R$
		◦
		$R^T$.
		
		\begin{solution}
			Отношение $R^T$ состоит из пар $\{(1, a), (2, b), (4, c), (8, d), (8, e), (8, f), (8, g), (8, h)\}$
			
			Тогда композиция $R^T$◦$R$ равна $\{(1, 1), (2, 2), (4, 4), (8, 8)\}$. а композиция $R$◦$R^T$ равна $\{(a, a), (b, b), (c, c), (d, d), (d, e),$ 
			
			$ (d, f), (d, g), (d, h), (e, d), (e, f), (e, g), (e, h), (e, e), (f, d), (f, e), (f, f), (f, g), (f, h), (g, d), (g, e), (g, f), (g, g), (g, h),$ 
			
			$ (h, d), (h, e), (h, f), (h, g), (h, h)\}$
			
			В композиция $R^T$◦$R$ 4 элемента, а в композиции  композиция $R$◦$R^T$ 3 + 5 $\cdot$ 5 = 28.
		\end{solution}
	
		\begin{answer}
			4 и 28
		\end{answer}
	
	\section*{№3}
	
		Пусть
		$R_1$
		,
		$R_2$
		— такие отношения на множествах
		A
		и
		B
		, что
		$R_1$
		$\cup$
		$R_2$
		является функцией. Докажите,
		что тогда и
		$R_1$, и
		$R_2$
		также являются функциями.
		
		\begin{proof}
			Докажем от противного: пусть $R_1$
			$\cup$
			$R_2$
			является функцией, но хотя бы одно из отношений $R_1$ и
			$R_2$
			не является функцией. Это значит, что в отношении, которое не является функцией, найдутся пары $(a, x), (a, y)$, где $x \ne y$ ($a \in A,\ x, y \in B$, либо $a \in B,\ x, y \in A$). Тогда в объединение 	$R_1$
			$\cup$
			$R_2$ входят пары $(a, x), (a, y)$, а значит это объединение не является функцией $\Rightarrow$ противоречие, а значит и
			$R_1$, и
			$R_2$
		    являются функциями.
		\end{proof}
	
	\section*{№4}
	
		Всегда ли композиция отношений эквивалентности является отношением эквивалентности?
		
		\begin{solution}
			Нет. Приведем пример: пусть на множестве $X = \{1, 2, 3\}$ заданы отношения  $A = \{(1, 1), (2, 2), (2, 3), (3, 2), (3, 3)\}$ и $B = \{(1, 1), (1, 2), (2, 2), (2, 1), (3, 3)\}$. Отношение $A$ является отношением эквивалентности, тк оно рефлексивно ($xAx \ \forall x\in X$), симметрично (если $xAy$, то $yAx$ $\forall x, y, \in X$) и транзитивно (если $xAy$ и $yAz$, то $xAz$ $\forall x, y, z \in X$). Аналогично $B$ является отношением эквивалентности. При этом $A \circ B = \{(1, 1), (1, 2), (1, 3), (2, 2), (2, 1), (2, 3), (3, 3), (3, 2)\}$. Заметим, что в композиции есть пара $(1, 3)$, но нет пары $(3, 1)$, а значит не соблюдается симметричность, те $A \circ B$ не является отношением эквивалентности.
		\end{solution}
	
		\begin{answer}
			Нет
		\end{answer}
	\section*{№5}
	
		\subsection*{a)}
		
			В связном графе степени восьми вершин равны 3, а степени остальных вершин равны 4. Докажите,
			что нельзя удалить ребро так, чтобы граф распался на две изоморфные компоненты связности.
			
			\begin{proof}
				Пусть в графе было $x$ вершин. Степени восьми равны 3, а степени $x - 8$ равны 4. Пусть в графе можно удалить ребро так, чтобы граф распался на две изоморфные компоненты связности. Тогда возможны 3 варианта: ребро, которое удалило соединяло вершины с степенью 3 и 3, 4 и 4, либо 3 и 4.
				
				1) Пусть ребро соединяло вершины степени 3 и 3. Тогда в графе стало 6 вершин с нечет степенью и $x - 6$ с чётной, и в каждой компоненте 3 вершины с нечет степенью, остальные с чёт. Тогда сумма степеней вершин в каждой компоненте - нечет, что невозможно (тк сумма степеней вершин графа чёт, тк равна удовенному числу рёбер).
				
				2) Пусть ребро соединяло вершины степени 4 и 4. Тогда в графе стало 10 вершин с нечет степенью и $x - 10$ с чётной, и в каждой компоненте 5 вершины с нечет степенью, остальные с чёт. Тогда сумма степеней вершин в каждой компоненте - нечет, что невозможно (тк сумма степеней вершин графа чёт, тк равна удовенному числу рёбер).
				
				3) Пусть ребро соединяло вершины степени 3 и 4. Тогда в графе будет ровно одна вершина степен 2, значит ровно в одной компоненте будет вершина степени 2, а в другой её не будет, а значит компоненты связности неизоморфные.
				
				Таким образом, нельзя удалить ребро так, чтобы граф распался на две изоморфные компоненты связности.
			\end{proof}
		
		\subsection*{б)}
			Верно ли аналогичное утверждение для графов с 10 вершинами степени 3 (и произвольным количеством вершин степени 4)?
			
			\begin{solution}
				Неверно. На рисунке ниже изображен граф с 10 вершинами степени 3 и 0 вершинами степени 4, в котором можно удалить ребро так, чтобы граф распался на две изоморфные конпоненты связности.
				
				\includegraphics{граф5_1}
				\includegraphics{граф5_2}
			\end{solution}
		
		\section*{№6}
		
			Найдите нестрогий порядок на четырёх элементах, в котором есть ровно три пары несравнимых элементов.
			
			\begin{solution}
				Пусть у нас есть множество, состоящее из 4 элементов-векторов: $\{(1, 2), (3, 4), (5, 6), (0, 7)\}$. Будем считать, что $(x_1, x_2) \leq (y_1, y_2) \Leftrightarrow x_i \leq y_i \forall i$. Тогда $(1, 2), (3, 4), (5, 6)$ несравнимы с $(0, 7)$ и сравнимы друг с другом, а значит в порядке есть ровно три пары несравнимых элементов.
			\end{solution}
		
		\section*{№7}
		
			Приведите пример порядка на 6 элементах, в котором есть 9 соседних пар. (Определение см. на
			предыдущей странице.) (Определения.
			Элементы
			x
			,
			y
			порядка
			(
			X, <
			)
			соседние
			(синонимы: элемент
			x
			непосредственно предше-
			ствует
			y
			, элемент
			y
			непосредственно следует за
			x
			), если
			x < y
			и нет такого
			z
			, что
			x < z < y.)
			
			\begin{solution}
				Пусть у нас есть множество, состоящее из 6 элементов-векторов: $\{(-5, -5), (1, 1), (2, 0), (3, -1), (5, 5), (4, 6)\}$. Будем считать, что $(x_1, x_2) < (y_1, y_2) \Leftrightarrow x_i < y_i \forall i$. Тогда $(-5, -5) < (1, 1); (-5, -5) < (2, 0); (-5, -5) < (3, -1); (1, 1) < (5, 5); (1, 1) < (4, 6); (2, 0) < (5, 5); (2, 0) < (4, 6); (3, -1) < (5, 5); (3, -1) < (4, 6)$, а $(1, 1); ,(2, 0), (3, -1)$ несравнимы и $(5, 5); (4, 6)$ несравнимы. Таким образом в таком порядке на 6 элементах есть 9 соседних пар.
			\end{solution}
	
	
	
	
	
	
	
	
	
		
		
	
	
	
	
	
	
	
	
	
	
	
	
	
	
\end{document}