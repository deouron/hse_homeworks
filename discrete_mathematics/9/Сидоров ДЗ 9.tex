\documentclass[a4paper, 16pt]{article}

\usepackage[utf8]{inputenc}

\usepackage[russian, english]{babel}
\usepackage{subfiles}
\usepackage[utf8]{inputenc}
\usepackage[T2A]{fontenc}
\usepackage{ucs}
\usepackage{textcomp}
\usepackage{array}
\usepackage{indentfirst}
\usepackage{amsmath}
\usepackage{amssymb}
\usepackage{enumerate}
\usepackage[margin=1.2cm]{geometry}
\usepackage{authblk}
\usepackage{tikz}
\usepackage{icomma}
\usepackage{gensymb}
\usepackage{graphicx}

\DeclareGraphicsExtensions{,.png,.jpg}

\graphicspath{{pictures/}}

\renewcommand{\baselinestretch}{1.4}

\renewcommand{\C}{\mathbb{C}}
\newcommand{\N} {\mathbb{N}}
\newcommand{\Q} {\mathbb{Q}}
\newcommand{\Z} {\mathbb{Z}}
\newcommand{\R} {\mathbb{R}}
\newcommand{\ord} {\mathop{\rm ord}}
\newcommand{\Ima}{\mathop{\rm Im}}
\newcommand{\rk}{\mathop{\rm rk}}

\renewcommand{\r}{\right}
\renewcommand{\l}{\left}
\renewcommand{\inf}{\infty}
\newcommand{\Sum}[2]{\overset{#2}{\underset{#1}{\sum}}}
\newcommand{\Lim}[2]{\lim\limits_{#1 \rightarrow #2}}
\newcommand\tab[1][1cm]{\hspace*{#1}}

\newcommand{\task}[1] {\noindent \textbf{Задача #1.} \hfill}
\newcommand{\note}[1] {\noindent \textbf{Примечание #1.} \hfill}
\newenvironment{proof}[1][Доказательство]{%
	\begin{trivlist}
		\item[\hskip \labelsep {\bfseries #1:}]
		\item \hspace{14pt}
	}{
		$ \hfill\blacksquare $
	\end{trivlist}
	\hfill\break
}
\newenvironment{solution}[1][Решение]{%
	\begin{trivlist}
		\item[\hskip \labelsep {\bfseries #1:}]
		\item \hspace{15pt}
	}{
	\end{trivlist}
}

\newenvironment{answer}[1][Ответ]{%
	\begin{trivlist}
		\item[\hskip \labelsep {\bfseries #1:}] \hskip \labelsep
	}{
	\end{trivlist}
	\hfill
}

\title{Дискретная математика} 
\date{\today}
\author{Сидоров Дмитрий}
\affil{Группа БПМИ 219}


\begin{document}
	\maketitle
	
	\section*{№1}
	
		Верно ли, что множество прямых на плоскости имеет мощность континуум?
	
		\begin{solution}
			Докажем, что верно, для этого докажем, что множество прямых на плоскости имеет мощность не меньше и не больше континуума.
			
			1) Не меньше:
			
			Любая прямая задаётся парой чисел $(k, b)$ (тк $y = kx + b$), те парой действительных чисел. По определению множество бесконечных двоичных последовательностей имеет мощность континуум. Рассмотрим прямые, которые параллельны оси $Ox$, те имеют $k = 0$:  такие прямые пересекают ось $Oy$ в точке $b$, а значит таких прямых столько же, сколько действительных точек на прямой, а значит их не меньше континуума, тогда множество прямых на плоскости имеет мощность не меньше континуума.
			
			2) Не больше:
			
			Любая прямая задаётся парой чисел $(k, b)$. Заметим, что $R^2$ равномощно $R$, те множество пар равномощно множеству $R$, а значит множество прямых на плоскости имеет мощность не больше континуума.
			
			Таким образом, множество прямых на плоскости имеет мощность континуум.
		\end{solution}
	
		\begin{answer}
			Верно
		\end{answer}
	
	\section*{№2}
		
		Докажите, что множество неубывающих бесконечных последовательностей натуральных чисел имеет
		мощность континуум.
		
		\begin{proof}
			Докажем, что это множество имеет мощность не меньше и не больше континуума.
			
			1) Не больше:
			
			Докажем, что таких  последовательностей не больше континуума. Множество натуральных чисел счётно. Покажем, что существует однозначное соответствие между последовательностью неатуральных чисел (не обязательно неубывающих) и двоичной последовательностью. Каждое натуральное число будем представлять в виде $n$ 1, где $n$ - представляемое натуральное число. После числа пишем 0 (отделяем от соседей). Таким образом, последовательностей натуральных чисел не больше континуума, а значит множество неубывающих бесконечных последовательностей натуральных чисел имеет мощность не больше континуума.
			
			2) Не меньше:
			
			 По определению множество бесконечных двоичных последовательностей имеет мощность континуум. Рассмотрим двоичную последовательность и построим соответствующую этой последовательности последовательность неубывающих натуральных чисел: пусть такая последовательность начинается на $n_1$ = 1, независимо от того, какую двоичную последовательность рассматриваем. Далее $n_k$ строим так, что $n_k = n_{k-1}$, если текущий член двоичной последовательности равен 0, и $n_k = n_{k-1} + 1$, если текущий член двоичной последовательности равен 1. Таким образом, мы построили неубывающую бесконечную последовательность натуральных чисел , и каждой построенной последовательности соответствует ровно одна двоичная последовательность, а значит множество неубывающих бесконечных последовательностей натуральных чисел имеет мощность не меньше континуума.
			 
			 Таким образом, множество неубывающих бесконечных последовательностей натуральных чисел имеет
			 мощность континуум.
		\end{proof}
		
	\section*{№3}
	
		Верно ли, что множество невозрастающих бесконечных последовательностей натуральных чисел имеет мощность континуум.
		
		 \begin{solution}
		 	Заметим, что если последовательность невозрастающих натуральных чисел бесконечна, то с какого-то места $k$ все её члены будут попарно равны (т е $a_{k-1} > a_k = a_{k+1} = a_{k+2}, \dots$). Тогда можно отбросить члены последовательности, начиная с $a_{k+1}$ и рассматривать конечные невозрастающие последовательности. Множество невозрастающих бесконечных последовательностей натуральных чисел имеет мощность не больше, чем континуум (из 2.1 последовательностей натуральных чисел не больше континуума), и так же множество натуральных чисел счётно. Покажем, что множество невозрастающих конечных последовательностей натуральных чисел счётно. Для этого покажем нумерацию таких последовательностей: сначала пересчитаем все последовательности с суммой членов 1, потом с суммой 2 итд. Для каждого натурального числа $n$ существует лишь конечное число конечных последовательностей натуральных чисел с суммой членов равной $n$, а значит таким образом можно пересчитать все конечные последовательности натуральных чисел, а значит множество невозрастающих бесконечных последовательностей натуральных чисел счётно.
		 \end{solution}
	 
	 	\begin{answer}
	 		Неверно
	 	\end{answer}
	 \section*{№4}
	 
	 	Счётно ли множество бесконечных двоичных последовательностей
	 	$b_0, b_1, \dots, b_n, \dots$, в которых каждый отрезок нечётной длины
	 	$b_i, b_{i+1}, \dots, b_{i+2k}$
	 	содержит почти поровну нулей и единиц (модуль разности равен 1)?
	 	
	 	\begin{solution}
	 		Докажем, что такое множество несчётно. Для этого покажем, что подмножество этого множества несчётно (а всякое подмножество счётного множества конечно или счётно). Рассмотрим последовательности, в которых каждый член равен либо 01, либо 10: такие последовательности удовлетворяют условию, тк сумма цифр каждого члена равна 1, членов нечётное количество, а значит модуль разности равен 1. Заметим, что если член 01 закодировать как 0, а 10 как 1, то каждая такая последовательность соответсвует своя последовательности из 0 и 1, а занчит таких последовательностей не меньше, чем континуум, а значит мощность множества таких последовательностей несчётна.
	 		
	 		Таким образом, множество бесконечных двоичных последовательностей
	 		$b_0, b_1, \dots, b_n, \dots$, в которых каждый отрезок нечётной длины
	 		$b_i, b_{i+1}, \dots, b_{i+2k}$
	 		содержит почти поровну нулей и единиц (модуль разности равен 1) имеет несчётное подмножество, а значит несчётно.
	 	\end{solution}
 	
 		\begin{answer}
 			Нет
 		\end{answer}
	
	\section*{№5}
	
		Углом на плоскости называется фигура, состоящая из точки и двух исходящих из неё лучей. Можно
		ли расположить на плоскости континуум непересекающихся углов, таких чтобы никакие два из них не
		имели одинаковую градусную меру?
		
		\begin{solution}
			Покажем, как можно построить континуум непересекающихся углов, таких чтобы никакие два из них не имели одинаковую градусную меру. Заметим, что рациональных точек на любом отрезке континуум. Выберем отрезок $[0; 1]$ на оси $Oy$, на нём будем отмечать вершины углов, их будет континуум. Из каждой вершины проведем луч паралелльно оси $Oy$ влево (ни один из этих лучей не пересекается, тк они все паралелльны друг другу). Для вершины в точке $(0; 1)$ проведём второй луч вдоль оси $Oy$, те получим угол с  градусной мерой  $90^\circ$, этот угол не пересекает остальные углы, тк его вершина находится выше вершины  каждого другого угла. Из угла с вершиной в точке (0; 0) проведём второй луч так, чтобы этот угол составлял $1^\circ$ с осью $Ox$ (те угол равен $179^\circ$). Далее для каждого угла с вершиной в точке $(0;y)$ будем проводить второй луч так, чтобы градусная мера угла была $(1 + 89y)^\circ$ с осью $Ox$ (или другими словами, градусная мера угла была $(180 - 1 - 89y)^\circ = $$(179 -  89y)^\circ$).  Заметим, что тогда все углы будут иметь разную градусную меру (чем выше будет вершина угла, тем меньше будет его градусная мера), и никакие два угла не пересекутся (лучи, параллельные оси $Ox$ лежат в 2-ой координатной четверти, непараллельные в 1-ой).
		\end{solution}
	
		\begin{answer}
			Можно
		\end{answer}
	
	\section*{№6 }
	
		Функция называется периодической, если для некоторого числа
		$T$
		и любого
		$x$
		выполняется
		$f
		(
		x
		+
		T
		) =
		f
		(
		x
		)$
		. Счётно ли множество множество периодических функций
		$f
		:
		Q
		\rightarrow
		Q$
		?
		
		\begin{solution}
			%Докажем, что множество периодических функций
			%$f
			%:
			%Q
			%\rightarrow
			%$ несчётно, и тогда множество периодических функций
			%$f
		%	:
			%Q
		%	\rightarrow
			%Q$ тоже несчётно.
			Докажем, что множество таких функций $f: Q \rightarrow Q$ несчётно. По определению континуум - мощность множества бесконечных двоичных последрвательностей $\{0, 1\}^N$. Рассмотрим множество из всех тотальных функций из рациональных чисел на отрезке $[0, 1)$ в $\{0, 1\}$. Каждая такая функция отображает счётное множество (тк этот полуинтервал - подмножество счётного множества $Q$) в множество $\{0, 1\}$, те состоящее из 2 элементов, а значит множество таких функций имеет мощность $2^N$, те континуум. По определению периодической функции выполняется
			$f
			(
			x
			+
			T
			) =
			f
			(
			x
			)$ для некоторого числа $T$. Рассмотрим $T \in Z$. Тогда каждое отображении можно представить как отображение рационального числа $x \in [0, 1)$ и $T$, как $x + T \in Q$. Заметим, что можно выбирать $T$ так, что получится отображение всего множества $Q$ в $\{0, 1\}$, а значит  множество периодических функций
			$f
			:
			Q
			\rightarrow
			Q$ будет иметь мощность континуум, те несчётно.
		\end{solution}
	
		\begin{solution}
			Несчётно
		\end{solution}
	
	\section*{№7}
	
		Докажите, что если
		A
		$\cup$
		B
		имеет мощность континуум, то
		A
		или
		B
		имеет мощность континуум.
		(
		Замечание.
		Никому неизвестно, существуют ли множества промежуточной мощности между счетными
		и континуальными.)
		
		\begin{proof}
			%Докажем от противного: пусть ни $A$, ни $B$ не имеют мощность континуум. 
			Заметим, что если и $A$, и $B$ конечны или счётны, то их объединение тоже конечно или счётно, те не континуум, а значит хотя бы одно из множеств $A$ и $B$ несчётно, те не меньше мощности континуум. Также заметим, что мощность объединения множеств $A, B$ совпадает с наибольшей из мощностей $A, B$ (те $|A\cup B| = max(|A|, |B|)$ ), а значит мощность  $A, B$ не больше мощности континуум. Тогда $A$
			или
			$B$
			имеет мощность континуум.
		\end{proof}
	
	
	
	
	
	
	
	
	
	
	
	
	
	
\end{document}