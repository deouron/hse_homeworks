\documentclass[a4paper, 16pt]{article}

\usepackage[utf8]{inputenc}

\usepackage[russian, english]{babel}
\usepackage{subfiles}
\usepackage[utf8]{inputenc}
\usepackage[T2A]{fontenc}
\usepackage{ucs}
\usepackage{textcomp}
\usepackage{array}
\usepackage{indentfirst}
\usepackage{amsmath}
\usepackage{amssymb}
\usepackage{enumerate}
\usepackage[margin=2cm]{geometry}
\usepackage{authblk}
\usepackage{tikz}
\usepackage{icomma}
\usepackage{gensymb}
\usepackage{graphicx}

\DeclareGraphicsExtensions{,.png,.jpg}

\graphicspath{{pictures/}}

\renewcommand{\baselinestretch}{1.5}

\renewcommand{\C}{\mathbb{C}}
\newcommand{\N} {\mathbb{N}}
\newcommand{\Q} {\mathbb{Q}}
\newcommand{\Z} {\mathbb{Z}}
\newcommand{\R} {\mathbb{R}}
\newcommand{\ord} {\mathop{\rm ord}}
\newcommand{\Ima}{\mathop{\rm Im}}
\newcommand{\rk}{\mathop{\rm rk}}

\renewcommand{\r}{\right}
\renewcommand{\l}{\left}
\renewcommand{\inf}{\infty}
\newcommand{\Sum}[2]{\overset{#2}{\underset{#1}{\sum}}}
\newcommand{\Lim}[2]{\lim\limits_{#1 \rightarrow #2}}

\newcommand{\task}[1] {\noindent \textbf{Задача #1.} \hfill}
\newcommand{\note}[1] {\noindent \textbf{Примечание #1.} \hfill}

\newenvironment{proof}[1][Доказательство]{%
	\begin{trivlist}
		\item[\hskip \labelsep {\bfseries #1:}]
		\item \hspace{15pt}
	}{
		$ \hfill\blacksquare $
	\end{trivlist}
	\hfill\break
}
\newenvironment{solution}[1][Решение]{%
	\begin{trivlist}
		\item[\hskip \labelsep {\bfseries #1:}]
		\item \hspace{15pt}
	}{
	\end{trivlist}
}

\newenvironment{answer}[1][Ответ]{%
	\begin{trivlist}
		\item[\hskip \labelsep {\bfseries #1:}] \hskip \labelsep
	}{
	\end{trivlist}
	\hfill
}

\title{Дискретная математика} 
\date{\today}
\author{Сидоров Дмитрий}
\affil{Группа БПМИ 219}


\begin{document}
	\maketitle
	
	\section*{№1}
	
		Найдите наименьшее количество вершин в графе, сумма степеней вершин в котором равна 24.
	
		\begin{solution}
			
			Сумма степеней вершин в графе равна удвоенному количеству рёбер в графе, значит, в этом графе $\frac{24}{2} = 12$ ребро. Граф с $n$ вершинами имеет максимум $\frac{n(n-1)}{2}$ рёбер. Значит 12 рёбер не может иметь граф с менее чем 6 вершинами. Значит в графе, сумма степеней вершин которого равна 24, как минимум есть 6 вершин. Пример такого графа см ниже.
			
			\includegraphics{граф1}
		\end{solution}
	
		\begin{answer}
			6
		\end{answer}
		
	\section*{№2}
	
		Существует ли граф на 9 вершинах, степени которых равны 1, 1, 1, 1, 1, 2, 4, 5, 6?
		
		\begin{solution}
			
			Пронумеруем вершины, степени которых 1, 1, 1, 1, 1, 2, 4, 5, 6 как 1, 2, 3, 4, 5, 6, 7, 8, 9 вершины соответственно. 
			%Заметим, что вершины 1-5 могут быть соединены только с одной вершиной, значит, вершина 9 должна быть соединена не более чем с 3 вершинами с степенью 1 (иначе вершина 8 не сможет иметь степень 5, тк если вершина 9 соединена с вершиной 8, то из вершин 1-5 должно остаться хотя бы 2 несоединеные вершины, а если 9 не соединена с вершиной 8, то из вершин 1-5 должно остаться хотя бы 3 несоединеные вершины). 
			Рассмотрим 2 случая: 9 и 8 имеют общее ребро или не имеют.
			
			1) 9 соединена с 8: 
			
			Тогда 9 должна быть соединена еще с 5 вершинами. Заметим, что если граф существует, то в нем есть только 4 вершины, у которых степень >1 (без 9 и 8 таких вершин 2), значит, если граф существует и 9 соединена с 8, то 9 соединена как минимум с 3 вершинами с степенью 1. Если 9 соединена более чем с 3 вершинами с степенью 1, то такой граф не существует, тк 8 должна быть соединена еще с 4 вершинами, а вершин, с которыми по условию можно быть соединеным осталось менее 4. Значит, 9 может быть соединена только с 3 вершинами с степенью 1. Таким образом, если рассматривать часть графа без ребер, которые инциденты вершине 9, и вершин, которые принадлежат этим ребрам, то вершины этой части графа будут иметь степени 1, 1, 1, 3, 4. Заметим, что эта часть графа не образует нужный граф, тк вершина с степенью 4 должна быть соединена с каждой вершиной, но тогда вершина с степенью 3 может быть соединена только с вершиной с степенью 4, и её степень будет не 3, а 1. Значит такой граф не существует.
			
			2) 9 не соединена с 8:
			
			Тогда 9 должна быть соединена с 6 вершинами, как минимум 4 из них имеют степень 1. Вершина 8 должна быть соединена с 5 вершинами, но вершин, с которыми она может быть соединена осталось не более 3 (одна с степенью 1 и вершины 6, 7). Значит такой граф не существует.
			
			Таким образом, граф не существует, если 9 соединена с 8 и если 9 не соединена с 8, значит такой граф не существует.
		\end{solution}
	
		\begin{answer}
			нет
		\end{answer}

	\section*{№3}
	
		В стране Радуга есть 7 городов с официальными названиями Красный, Оранжевый, Жёлтый, Зелёный, Голубой, Синий, Фиолетовый. Путешественник обнаружил, что два города соединены авиалинией в том и только в том случае, если количество общих букв в названиях городов не меньше 3. (Количество вхождений букв несущественно, прописные и строчные не различаются, используются только официальные названия.) Можно ли добраться из города Красный в город Фиолетовый, используя эти авиалинии (возможно, с пересадками)?
	
		\begin{solution}
			Можно, например, из города Красный добраться в Оранжевый (совпадают буквы р, а, н...), а из города Оранжевый добраться в город Фиолетовый (совпадают буквы о, в, ы...).
		\end{solution}
	
		\begin{answer}
			Да
		\end{answer}
		
	\section*{№4}
	
		\begin{solution}
			Заметим, что в графе $2n$ вершин, и каждая вершина имеет степень $n$. Обозначим подграф отрезков $[(0,i); (n+1, i)]$ как $Y$, а подграф отрезков
			$[(i, 0); (i, n+1)]$ как $X$. Заметим, что подграфы $X$ и $Y$ состоят только из вершин, причём ни одна пара которых не связана ребром (т. к. парал отрезки не пересекаются), но при этом каждая вершина $X$ соединена с каждой вершиной $Y$, а каждая вершина $Y$ соединена с каждой вершиной $X$. Таким образом, если вершина $x \in X$ входит в независимое множества в графе $L_n$, то в это множество не может входить ни одна вершина $y \in Y$ (аналогично если вершина $y \in Y$ входит в независимое множество в графе $L_n$, то в это множество не может входить ни одна вершина $x \in X$). Таким образом, размер независимого множества в графе $L_n$ не более чем половина вершин в графе $L_n$, т. е. $\frac{2n}{2} = n$. Значит, можно выбрать все элементы $x \in X$
			(или $y \in Y$) как элементы независимого множества в графе $L_n$, и это будет 
			независимое множество в графе $L_n$ максимального размера, и этот размер будет равен $n$.
		\end{solution}
	
		\begin{answer}
			n
		\end{answer}
	
	\section*{№5}
	
		Докажите, что при n $\geq$ 1 связен любой граф на 2n + 1 вершине, степень каждой из которых не меньше n.
	
		\begin{proof}
			Пусть такой граф не связен. Тогда найдутся 2 вершины, которые не соединены путём. Каждая из этих двух вершин соединена не менее чем с $n$ вершинами (по условию степень каждой вершины не меньше $n$, а эти вершины не имеют общий путь $\Rightarrow$ вершины, с которыми соединены эти верны, не совпадают), значит, в графе не менее $1 + 1 + 2n = 2 + 2n$ вершин, что противоречит условию, что в графе $2n + 1$ вершина, значит предположение не верно, и такой граф связен. 
		\end{proof}
	
	\section*{№6}
	
		В графе 2n + 1 вершина, степень каждой равна n. Докажите, что после удаления любого подмножества из менее чем n рёбер получается связный граф.
		
		\begin{proof}
			Пусть после удаления любого подмножества из менее чем n рёбер получился не связный граф. Тогда получаются как минимум 2 компоненты связности. Обозначим эти компоненты как $A$ и $B$. Обозначим за $A$ компоненту связности, в которой наименьшее число элементов (т. о. в $A$ не более $n$ элементов, а в $B$ не менее $n+1$ элемент). Пусть в  $A \ a$ элементов ($a$ $\leq n$). Степень вершины в $A$ не больше чем $a - 1$. Тк в исходном графе степень вершины была равна $n$, то из $A$ удалили минимум $n - a + 1$ ребро, тк эти ребра были удалены, то эти рёбра соединяли вершины из $A$ с вершинами из $B$, значит всего удалили не менее $a(n - a + 1)$ ребра (по условия удалили не более чем $n - 1$ ребро, т. е. $a(n - a + 1) < n$). Тогда $an - a^2 + a - n <0$, $an - a^2 + a - n =a(n-a)-(n-a)=
			(n-a)(a-1)<0$, что является ложью, тк $a$ $\leq n$ и $a \geq 1$. Значит предположение не верно, и после удаления любого подмножества из менее чем n рёбер получается связный граф.
		\end{proof}
	
	\section*{№7}
	
		\begin{solution}
			%Заметим, что $\frac{2021}{47} = 43$	
			Докажем, что любое слово длины 2021 последовательными инвертированиями битов в 47-ми позициях можно превратить в нулевое. Если в слове число единич $p$ кратно 47, то, инвертируя какие-нибудь 47 единиц, спустя $\frac{p}{47}$ таких операций получим нулевое слово. Иначе инвертируем 47 единиц, пока в слове не останется менее 47 единиц. Далее возможно 2 варианта: в слове осталось чётное или нечётное количество единиц. 
			
			1) Если в слове осталось нечётное число единиц. Покажем, что в любом слове за 2 оперции можно увеличить количество единиц на 2. Пусть в слове $k$ единиц (причем $k \ne 47$, т. к. иначе они были бы инвертированы до этого шага). Тогда изменим любые 47 нулевые позициии (а такие точно найдутся, т. к. в слове осталось менее 47 единиц), получим слово, в котором 47 + $k$ единиц. Далее
			инвертируем 46 единиц и 1 ноль. Получим слово, в котором $k$ + 1 (оставшаяся единица) + 1 (единица, которая на прошлом шаге была нулём) = $k + 2$ единиц. Таким образом, тк в слове было нечётное число единиц, а нечет + 2 = нечет, спустя несколько таких операций получим слово, в котором 47 единиц, т. е. из него можно получить нулевое слово.
			
			2) Если в слове осталось чётное число единиц (причём не 0, т. к. иначе слово уже нулевое). Тогда покажем, что в любом слове за 2 оперции можно уменьшить количество единиц на 2. Инвертируем одну из единиц и 46 нулей. Следующим шагом инвертируем 46 единиц, которые на прошлом шаге были нулями, и еще одну единицу. Тогда в исходном слове количество единиц уменьшится на 2. Т. к. чет - 2 = чет, то спустя несколько таких операций получится слово, в котором 0 единиц, т. е. оно нулевое.
			
			Таким образом, из любого слова длины 2021 последовательными инвертированиями битов в 47-ми позициях можно превратить в нулевое, значит граф $Q_{2021,47}$ связен.
			
		\end{solution}
	
		\begin{answer}
			да
		\end{answer}
	
	\section*{№8}
	
		Докажите, что в любом графе на 2$n$ вершинах с $n^2$ + 1 ребром, $n \geq 2$, найдётся треугольник: три попарно смежные вершины.
	
		\begin{proof}
			Докажем по индукции.
			
			1) При $n=2$ в графе 4 вершины и 5 рёбер. В полном графе с 4 вершинами $\frac{4 \cdot 3}{2} = 6$ рёбер, значит в графе, в котором 4 вершины и 5 рёбер, попарно соединены все верщины, кроме 2. Значит есть три попарно смежные вершины (например, 2 другие вершины и одна из 2 попарно не соединеных). 
			
			2) Пусть верно при $n=k$ (т. е. выполняется для $2k$ вершин и $k^2+1$ ребра). Докажем, что условие выполняется при $n = k + 1$.
			Тогда необходимо доказать, что в графе с $2k+2$ вершинами и с $(k+1)^2+1=k^2+2k+2$ ребром найдется треугольник. Рассмотрим 2 вершины $A$ и $B$. Они образуют треугольник, если нашлась такая вершина $C$, что $C$ соединена ребром и с $A$, и с $B$. Пусть такой вершины нет (если есть, то треугольник нашелся). Тогда степень $A$ и степень $B$ не более $2k$ (т. к. и $A$, и $B$ не соединены сами с собой и с $C$). Рассмотрим граф без $A$ и $B$. Тогда в графе останется $2k+2-2=2k$ вершины и не менее чем $(k+1)^2 + 1 - 2k = k^2+2$ ребро. Но по индукционному предположению в графе с $2k$ вершинами и $k^2+1$ ребром найдётся треугольник, значит треугольник найдется в графе с $2k$ вершинами и $k^2+2$ ребром.  Значит по методу мат индукции в любом графе на 2$n$ вершинах с $n^2$ + 1 ребром, $n \geq 2$, найдётся треугольник.
		\end{proof}
	
\end{document}
