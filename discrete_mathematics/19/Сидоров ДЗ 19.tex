\documentclass[a4paper, 16pt]{article}

\usepackage[utf8]{inputenc}

\usepackage[russian, english]{babel}
\usepackage{subfiles}
\usepackage[utf8]{inputenc}
\usepackage[T2A]{fontenc}
\usepackage{ucs}
\usepackage{textcomp}
\usepackage{array}
\usepackage{indentfirst}
\usepackage{amsmath}
\usepackage{amssymb}
\usepackage{enumerate}
\usepackage[margin=1.2cm]{geometry}
\usepackage{authblk}
\usepackage{tikz}
\usepackage{icomma}
\usepackage{gensymb}
\usepackage{graphicx}
\usepackage{mathtools} 
\usepackage[makeroom]{cancel}

\DeclareGraphicsExtensions{,.png,.jpg}

\graphicspath{{pictures/}}

\renewcommand{\baselinestretch}{1.4}

\renewcommand{\C}{\mathbb{C}}
\newcommand{\N} {\mathbb{N}}
\newcommand{\Q} {\mathbb{Q}}
\newcommand{\Z} {\mathbb{Z}}
\newcommand{\R} {\mathbb{R}}
\newcommand{\ord} {\mathop{\rm ord}}
\newcommand{\Ima}{\mathop{\rm Im}}
\newcommand{\rk}{\mathop{\rm rk}}

\renewcommand{\r}{\right}
\renewcommand{\l}{\left}
\renewcommand{\inf}{\infty}
\newcommand{\Sum}[2]{\overset{#2}{\underset{#1}{\sum}}}
\newcommand{\Lim}[2]{\lim\limits_{#1 \rightarrow #2}}
\newcommand\tab[1][1cm]{\hspace*{#1}}

\newcommand{\task}[1] {\noindent \textbf{Задача #1.} \hfill}
\newcommand{\note}[1] {\noindent \textbf{Примечание #1.} \hfill}
\newenvironment{proof}[1][Доказательство]{%
	\begin{trivlist}
		\item[\hskip \labelsep {\bfseries #1:}]
		\item \hspace{14pt}
	}{
		$ \hfill\blacksquare $
	\end{trivlist}
	\hfill\break
}
\newenvironment{solution}[1][Решение]{%
	\begin{trivlist}
		\item[\hskip \labelsep {\bfseries #1:}]
		\item \hspace{15pt}
	}{
	\end{trivlist}
}

\newenvironment{answer}[1][Ответ]{%
	\begin{trivlist}
		\item[\hskip \labelsep {\bfseries #1:}] \hskip \labelsep
	}{
	\end{trivlist}
	\hfill
}

\title{Дискретная математика} 
\date{\today}
\author{Сидоров Дмитрий}
\affil{Группа БПМИ 219}


\begin{document}
	\maketitle
	
	\section*{№1}
	
		Найдите две последние цифры числа $99^{1000}$.
		
		\begin{solution}
			Найти две последние цифры числа $99^{1000}$ значит найти остаток $99^{1000}$ при делении на 100. Заметим, что $99 \equiv -1 \pmod{100}$, значит $99^{1000} \equiv (-1)^{1000} \pmod{100}$. Тк $ (-1)^{1000} = 1$, число $99^{1000}$ оканчивается на 01.
		\end{solution}
	
		\begin{answer}
			01
		\end{answer}
	
	\section*{№2}
	
		Докажите, что числа $a^2$ и $b^2$ дают одинаковые остатки при делении на $a - b$, если $a$ и $b$ — положительные целые числа, и $a > b$.
		
		\begin{proof}
			%Пусть числа $a^2$ и $b^2$ дают остатки $r_1$ и $r_2$ при делении на $a - b$ соотв. Тогда $a^2 = (a-b) \cdot q_1 + r_1$ и $b^2 = (a-b) \cdot q_2 + r_2$. Вычтем из первого уравнения второе, тогда $a^2 - b^2 = (a - b)(a + b) = (a - b)(q_1 - q_2) + (r_1 - r_2)$, значит $(r_1 - r_2)$ нацело делится на $a - b$. При этом $0 \leq r_1 < a-b, \ 0 \leq r_2 < a-b$ (тк остатки при делении на $a-b$), значит $r_1 - r_2 < a - b \Rightarrow r_1-r_2=0, \ r_1 = r_2$. 
			Заметим, что $a^2 - b^2 = (a-b)(a + b) \ \vdots \ (a - b)$. Пусть $b^2$ даёт остаток $r$ при делении на $(a - b)$, те $b^2 = (a - b) \cdot q_1 + r$. Тогда $a^2 = b^2 + (a-b)(a + b) = (a-b) \cdot q_1 + r + (a-b)(a+b) = (a-b)(q_1 + a + b) + r$, значит $a^2$ при делении на $(a - b)$ даёт остаток $r$, и значит числа $a^2$ и $b^2$ дают одинаковые остатки при делении на $a - b$.
		\end{proof}
	
	\section*{№3}
	
		Пусть x, y — целые числа. Докажите, что число x + 10y делится на 13 тогда и только тогда, когда
		y + 4x делится на 13.
		
		\begin{proof}
			$13 \mid x + 10 y \Leftrightarrow 13 \mid 4(x + 10y) \Leftrightarrow 13 \mid 4x + 40y \Leftrightarrow 13 \mid 4x + y + 39 y \Leftarrow 13 \mid 4x+y $
		\end{proof}
	
	\section*{№4}
	
		Решите сравнение $53x \equiv 1 \pmod{42}$ с помощью алгоритма Евклида.
		
		\begin{solution}
			$53x + 42y = 1$
			
			$a_i = a_{i-2} - q_{i-1} \cdot a_{i-1}$, $x_i = x_{i-2} - q_{i-1} \cdot x_{i-1}$, 		$y_i = y_{i-2} - q_{i-1} \cdot y_{i-1}$, где $q_{i-1}$ - неполное частное при делении $a_{i-2}$ на $a_{i-1}$
			
			\
			
			\begin{tabular}{ l l l l l}
				$i$ & $a_i$ & $x_i$ & $y_i$ & $q_i$  \\
				0 & 53 & 1 & 0 & -\\
				1 & 42 & 0 & 1 & 1\\
				2 & 11 & 1 & -1 & 3 \\
				3 & 9 & -3 & 4 & 1 \\
				4 & 2 & 4 & -5 & 4 \\
				5 & 1 & -19 & 24 & - \\
			\end{tabular}
		
		\
		
		Таким образом, $-19 \cdot 53 + 24 \cdot 42 = 1$, и если $53x \equiv 1 \pmod{42}$, то $ x = -19 + 42 = 23$
		\end{solution}
	
	\begin{answer}
		23
	\end{answer}
	
	\section*{№5}
	
		Докажите, что дробь $\frac{n^2 - n+1}{n^2+1}$ несократима при всех положительных целых $n$.
		
		\begin{proof}
			Если  НОД($n^2 - n+1,{n^2+1}$) = 1, то дробь несократима. Пусть НОД($n^2 - n+1,{n^2+1}$) = $x$, тогда $n^2 - n + 1 = x \cdot q$ и $n^2 + 1 = x \cdot p, \ p, q \in \Z$. 
			\begin{equation*}
				\begin{cases}
					n^2 - n + 1 = x \cdot q \\
					n^2 + 1 = x \cdot p \\ 
				\end{cases}
			\end{equation*}
		
		\begin{equation*}
			\begin{cases}
				- n = x \cdot (q - p) \\
				n^2 + 1 = x \cdot p \\ 
			\end{cases}
		\end{equation*}
			Тк $n > 0$ по условию, то:
		\begin{equation*}
			\begin{cases}
				-n^2 = -n \cdot x \cdot (q - p) \\
				n^2 + 1 = x \cdot p \\ 
			\end{cases}
		\end{equation*}

		\begin{equation*}
			\begin{cases}
				-n^2 = -n \cdot x \cdot (q - p) \\
				1 = x \cdot p  -n \cdot x \cdot (q - p)\\ 
			\end{cases}
		\end{equation*}
	
		Значит $1 = x \cdot p  -n \cdot x \cdot (q - p) \Rightarrow p - n  \cdot (q - p) = \frac{1}{x}$. Тк $p, q, n \in \Z$, то $p - n  \cdot (q - p) \in Z \Rightarrow \frac{1}{x} \in Z \Rightarrow x = 1$ и  НОД($n^2 - n+1,{n^2+1}$) = 1.
		
		\end{proof}
	
	\section*{№6}
	
		Может ли целое положительное число, в десятичной записи которого 100 нулей, 100 единиц и 100
		двоек, быть точным квадратом? (Т.е. квадратный корень целый.)
		
		\begin{solution}
			Целое положительное число, в десятичной записи которого 100 нулей, 100 единиц и 100 двоек, имеет сумму цифр $100 \cdot 0 + 100 \cdot 1 + 100 \cdot 2 = 300 \ \vdots \ 3$, и значит такое число делится на 3. Если такое целое положительное число делится на 3, то его квадрат делится на $3^2=9$. Пусть такое число может быть точным квадратом, тогда оно делится на 9, и значит сумма его цифр делится на 9, но $9 \cancel{\vert} 300 \Rightarrow$ противоречие, и целое положительное число, в десятичной записи которого 100 нулей, 100 единиц и 100
			двоек, не может быть точным квадратом.
		\end{solution}
	
		\begin{answer}
			нет
		\end{answer}
	
	\section*{№7}
	
		Найдите наименьшее целое положительное число N такое, что и сумма цифр десятичной записи
		числа N, и сумма цифр десятичной записи числа N + 1 делятся на 7.
		
		\begin{solution}
			Если последняя цифра целого положительного числа $N$ числа не равна 9, то сумма сумма цифр десятичной записи числа $N + 1$ на 1 больше суммы цифр числа $N$. Пусть число $N$ оканчивается на $n$ 9. Тогда число $N + 1$ оканчивается на $n$ нулей, и $n + 1$ число справа увеличивается на 1, а значит сумма цифр уменьшается на $9n - 1$ (при этом если $N$ $n$-значное, то $N+1$ становится $n+1$ - значным и  $n + 1$ цифра справа равна 1). Если сумма цифр десятичной записи
			числа $N$, и сумма цифр десятичной записи числа $N + 1$ делятся на 7, и сумма цифра может меняться на 1 или на $9n-1$. Тк $7\cancel{\vert} 1$, то необходимо найти такое наименьшее $n$, что $7 \vert (9n-1)$ и такое наименьшее $N$, что  $7 \vert N$. Тогда $n = 4$ (тк $9\cdot1-1 =8, \ 9\cdot 2-1 = 17$ и $9\cdot 3-1 = 26$ не делятся нацело на 7, а $9\cdot 4-1 = 35$ делится). Значит $N$ оканчивается на 4 девятки. Сумма цифр $N$ делится на 7, $4 \cdot 9 = 36$, значит наименьшее целое положительное число $N$ , что и сумма его цифр делится на 7 равно 6999, и при этом $N+1$ = 7000 и сумма его цифр тоже делится на 7.
		\end{solution}
	
		\begin{answer}
			6999
		\end{answer}
	
	
	
	
	
	
	
\end{document}