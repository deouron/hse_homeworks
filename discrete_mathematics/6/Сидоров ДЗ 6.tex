\documentclass[a4paper, 16pt]{article}

\usepackage[utf8]{inputenc}

\usepackage[russian, english]{babel}
\usepackage{subfiles}
\usepackage[utf8]{inputenc}
\usepackage[T2A]{fontenc}
\usepackage{ucs}
\usepackage{textcomp}
\usepackage{array}
\usepackage{indentfirst}
\usepackage{amsmath}
\usepackage{amssymb}
\usepackage{enumerate}
\usepackage[margin=1.2cm]{geometry}
\usepackage{authblk}
\usepackage{tikz}
\usepackage{icomma}
\usepackage{gensymb}
\usepackage{graphicx}

\DeclareGraphicsExtensions{,.png,.jpg}

\graphicspath{{pictures/}}

\renewcommand{\baselinestretch}{1.4}

\renewcommand{\C}{\mathbb{C}}
\newcommand{\N} {\mathbb{N}}
\newcommand{\Q} {\mathbb{Q}}
\newcommand{\Z} {\mathbb{Z}}
\newcommand{\R} {\mathbb{R}}
\newcommand{\ord} {\mathop{\rm ord}}
\newcommand{\Ima}{\mathop{\rm Im}}
\newcommand{\rk}{\mathop{\rm rk}}

\renewcommand{\r}{\right}
\renewcommand{\l}{\left}
\renewcommand{\inf}{\infty}
\newcommand{\Sum}[2]{\overset{#2}{\underset{#1}{\sum}}}
\newcommand{\Lim}[2]{\lim\limits_{#1 \rightarrow #2}}
\newcommand\tab[1][1cm]{\hspace*{#1}}

\newcommand{\task}[1] {\noindent \textbf{Задача #1.} \hfill}
\newcommand{\note}[1] {\noindent \textbf{Примечание #1.} \hfill}
\newenvironment{proof}[1][Доказательство]{%
	\begin{trivlist}
		\item[\hskip \labelsep {\bfseries #1:}]
		\item \hspace{14pt}
	}{
		$ \hfill\blacksquare $
	\end{trivlist}
	\hfill\break
}
\newenvironment{solution}[1][Решение]{%
	\begin{trivlist}
		\item[\hskip \labelsep {\bfseries #1:}]
		\item \hspace{15pt}
	}{
	\end{trivlist}
}

\newenvironment{answer}[1][Ответ]{%
	\begin{trivlist}
		\item[\hskip \labelsep {\bfseries #1:}] \hskip \labelsep
	}{
	\end{trivlist}
	\hfill
}

\title{Дискретная математика} 
\date{\today}
\author{Сидоров Дмитрий}
\affil{Группа БПМИ 219}


\begin{document}
	\maketitle
	
	\section*{№1}
	
	Граф
	G
	можно по крайней мере тремя различными способами правильно раскрасить в 2 цвета.
	Докажите, что
	G
	несвязный.
	
	\begin{proof}
		Докажем, что любой связный граф можно покрасить в 2 цвета не более 2 способами. Если граф 2-раскрашиваемый, то в неём длины всех циклов чётные. Выберем любую вершину $x$ графа. Пусть цвет вершины $x$ - 1, тк граф связный и 2-раскрашиваемый, то любая другая вершина $y$ графа будет покрашена в цвет 1, если длина пути из $x$ в $y$ чётна, а иначе вершина $y$ будет покрашена в цвет 2. Таким образом, цвет вершины $x$ однозначно определяет цвет всех остальных вершин графа. Аналогично, если цвет вершины $x$ - 2, то любая другая вершина $y$ графа будет покрашена в цвет 2, если длина пути из $x$ в $y$ чётна, в 1 иначе, и цвет вершины $x$ однозначно определяет цвет всех остальных вершин графа. Пусть существует 3-ий вариант раскраски связного графа. Тогда цвет $x$ либо 1, либо 2, а это уже даёт однозначную раскраску, которая совпадает с одной из предыдущих, а значит любой связный граф можно покрасить в 2 цвета не более 2 способами.
		
		Граф
		G
		можно по крайней мере тремя различными способами правильно раскрасить в 2 цвета, а значит он несвязный.
	\end{proof}
	
	\section*{№2}
	Докажите, что в дереве на
	$2n$
	вершинах можно выбрать независимое множество из
	$n$
	вершин.
	
	\begin{proof}
		
		Любое дерево 2-раскрашиваемо, тк в 2 цвета можно раскрасить любой граф, не имеющий циклов нечётной длины (а в дереве нет циклов), значит дерево на $2n$ вершинах можно раскрасить в 2 цвета, причём для любой пары вершин $x$, $y$ выполняется, что $x$ и $y$ покрашены в один цвет, если длина пути из $x$ в $y$ чётна, и в разный иначе. Таким образом, половину вершин дерева можно покрасить в цвет 1, а другую половину в цвет 2, а значит в дереве можно выбрать все вершины цвета 1 (или 2), и эти вершины образовывают независимое множество из $n$ вершин.
		
	\end{proof}
	
	\section*{№3}
	
	В графе 17 вершин. Они расставлены по кругу так, что каждое из 34 рёбер графа соединяет пару
	соседних в расстановке вершин или пару вершин, между которыми есть ровно одна другая вершина.
	Можно ли вершины этого графа правильно раскрасить в 3 цвета?
	
	\begin{solution}
		В графе 17 вершин и 34 ребра, значит каждая вершина будет соединена с 4 вершинами: 2 соседними и 2 через одну. Не ограничивая общности, пусть вершина 1 имеет цвет 1 (на рисунке цвет 1 - красный, 2 - синий, 3 - зелёный). Тода вершины 2 и 3 разного цвета (тк соединены ребром), и их цвет отличен от цвета 1 (не ограничивая общности, будет считать, что цвет вершины 2 - 2, а вершины 3 - 3). Значит, цвет вершины 4 - 1 (тк 4 соединена и с 3, и с 2), цвет вершины 5 - 2 (тк соединена с 4 и 3), цвет 6 - 3. Заметим, что таким образом, обходя граф по часовой стрелки, вершины будут разбиваться на тройки, где вершина с номером $x$, где $x = 1$ mod(3), будет иметь цвет 1, а вершины с номерами $x+1, x+2$ будут иметь цвет 2 и 3. Заметим, что 17 не делится нацело на 3 и вершина 16 должна иметь цвет 1, но цвет 1 имеет вершина 1, которая соединена с 16, а значит вершина 1 не может иметь цвет 1, но аналогично какой бы цвет не имела бы вершина 1, такой цвет должна иметь вершина 16, а значит такой граф нельзя правильно раскрасить в 3 цвета.
		
		\includegraphics{граф3}
		
	\end{solution}

	\begin{answer}
		нет
	\end{answer}
	
	\section*{№4}
	
	На столе лежит 200 фишек: 100 красных, на которых написаны числа от 1 до 100, и 100 синих,
	на которых также написаны числа от 1 до 100. Враг забрал 99 фишек со стола (по своему усмотрению). Докажите, что на столе осталась пара из красной и синей фишек, сумма чисел на которых не
	меньше 101
	
	\begin{proof}
		Рассмотрим двудольный граф, левая доля которого состоит из 100 вершин - красных фишек, пронумеруем их как $r_1, r_2,\dots ,r_{100}$, где индекс вершины - число, написанное на фишке, аналогично правая доля двудольного графа состоит из 100 вершин $b_1,b_2,\dots,b_{100}$ - синих фишек. Пусть в двудольном графе вершины соединены ребром, если они могут образовать пару из красной и синей фишки, значит каждая вершины из правой доли соединена с каждой вершиной левой доли (каждая вершина левой доли соединена с каждой вершиной правой доли), но при этом вершины каждой доли образуют независимое множество. Заметим, что если враг забирает 99 фишек максимального номинала (те для любой пары фишек с числами $x$ и $x+1$ он предпочитает забрать фишку с числом $x+1$, а для  графа убирание фишки - удаление соответсвующей вершины графа и всех рёбер, концом которых является эта вершина), но при этом на столе остаётся пара из красной и синей фишки, сумма чисел на которых не
		меньше 101, то на столе останется пара из красной и синей фишки, сумма чисел на которых не меньше 101, независимо от того, какие фишки забрал бы враг. Пусть враг забрал $x$ красных фишек, значит он забрал $99-x$ синих фишек. Если он забрал фишки максимального номинала, то в левой части графа останутся вершины $r_1,r_2,\dots,r_{100-x}$, а в правой останутся вершины $b_1,b_2,\dots,b_{1+x}$ (тк $100 - (99-x) = 1 + x)$. Заметим, что вершины $,r_{100-x}, b_{1+x}$ остались в графе и соединены ребром, значит на столе остались красная и синяя фишки номиналом $100-x$ и $1 + x$, сумма чисел на которых равна $100-x+1+x=101$. Значит, какие бы 99 фишек не забрал бы враг, на столе останется пара из красной и синей фишки, сумма чисел на которых не
		меньше 101.
	\end{proof}
	
	\section*{№5}
	
		В графе на 30 вершинах (необязательно двудольном) между любыми тремя вершинами есть хотя
		бы два ребра. Докажите, что в графе есть совершенное паросочетание (из 15 рёбер).
	
			\begin{proof}
				Заметим, что если произвольные вершины $x, y$ графа не соединены друг с другом ребром, то эти вершины соединены с каждой другой вершиной графа (тк по условию между любыми тремя вершинами есть хотя
				бы два ребра, а значит для любой тройки вершин $x,y, z$, где $z$ - произвольная вершина графа, отличная от $x, y$, $x$ будет соединена с $z$ и $y$ будет соединена с $z$). Таким образом, каждая вершина графа имеет как минимум $n-2$ соседей, где $n$ - количесвто вершин в графе. 
				
				Пусть $x$ вершин графа имеют 28 соседей (степень 28), тогда $30-x$ имеют степень 29, а значит сумма степень вершин графа равна $28x+30\cdot29-29x=30\cdot29-x$. Заметим, что сумма степень вершин графа равна удвоенному числу вершин в гарфе, а значит может быть только чётной, те ($30\cdot29-x$) - чётное число. $30\cdot29=870$ - чёт $\Rightarrow x$ - чёт, а это значит, что в графе чётное число вершин имеют степень 29 и чётное число вершин имеют степень 28. 
				
				Покажем по индукции, что в графе на 30 вершинах есть совершенное паросочетание (из 15 рёбер). 
				
				1) Покажем, что в графе на 4 вершинах, в котором между любыми тремя вершинами есть хотя бы два ребра, есть совершенное паросочетание (из 2 рёбер).
				
				\includegraphics{граф5}
				
				Если вершины 2 и 3 не соединены между собой, то каждая из них соединена с вершинами 1 и 4 (доказано ранее), а значит в графе есть совершенное паросочетание 12, 43.
				
				Если вершины 2 и 3 соединены между собой. Если 1 и 4 соединены между собой, то в графе есть совершенное паросочетание 14, 23, иначе каждая из вершин 1 и 4 соединена с вершинами 2 и 3, а значит в графе есть совершенное паросочетание12, 43.
					
				Таким образом, выполняется для $n=4$
				
				2) Пусть выполняется для $n=k$, где $k$ - чёт. Докажем, что выполняется для $n=k + 2$. В графе есть паросочетание из $k$ вершин, пусть в него входят все вершины, кроме вершин $a$ и $b$. Если $a$ и $b$ соединены между собой, то в графе $n=k + 2$ есть совершенное паросочетание. Пусть $a$ и $b$ не соединены между собой. Тогда $a$ и $b$ соединены со всеми остальными вершинами графа, а значит из паросочетания можно выбрать вершины $x, y$, которые соединены между собой, причём с ними соединены $a$ и $b$. Тогда в графе есть рёбра $ax$, $by$, а значит в графе на $n=k + 2$ вершинах есть совершенное паросочетание.
				
				Таким образом, по индукции в графе на 30 вершинах есть совершенное паросочетание (из 15 рёбер). 
			\end{proof}

	\section*{№6}
	
		Вася составил список из всех 12-элементных подмножеств 26-элементного множества, каждое записал по одному разу. Петя добавляет по одному элементу в каждое множество списка. Докажите, что
		Петя может так выполнить добавления, чтобы среди полученных 13-элементных множеств не было
		одинаковых.
		
		\begin{proof}
			Рассмотрим двудольный граф, левая доля которого состоит из вершин - 12-элементных подмножеств, а правая из вершин - 13-элементных подмножеств. Будем считать, что вершины из двух долей соединены ребром, если 2 подмножества отличаются ровно на 1 элемент (те вершина из левой доли соединена с вершиной из правой доли, если такое 13-элементое множество можно получить из 12-элементового добавлением ровно одного элемента). Тогда каждая вершина из левой доли соединена с $26-12=14$ вершинами из правой доли (тк всего 26 элементов, в 12-элементом множестве 12-элементов и 13-элементное множество можно получить добавлением в 12-элементное ровно 1 элемента из 14), при этом каждая вершина из правой доли соединена с 13 вершинами из левой доли (тк 12-элементное множество множно получить из 13-элементного удалением ровно одной вершины 13 способами). Таким образом, для каждого множества $X$ вершин левой доли множество соседей $G(X)$ из правой доли содержит не меньше вершин, чем $X$, а тогда по теореме Холла в графе есть паросочетание размера числа элементов в левой доли, те для каждого 12-элементного подмножества можно выбрать 13-элементное подмножество так, что 13-элементные подмножества не повторяются, а значит Петя может так выполнить добавления, что среди полученных 13-элементных множеств не будет
			одинаковых.
		\end{proof}
	
	\section*{№7}
	
		Постройте двудольный граф, в каждой доле которого 7 вершин, степени всех вершин равны 3 и при
		этом у любых двух вершин из одной доли есть ровно один общий сосед.
		
		\begin{solution}
			\includegraphics{граф7}
			
			Левая доля графа - вершины 1-7, правая - 8-14, каждая вершина имеет степень 3. Покажем, что у любых двух вершин из одной доли есть ровно один общий сосед. 
			
			1) Рассмотрим вершины левой доли: у вершин 1 и 2 ровно один общий сосед - 9, 1 и 3 - 11, 1 и 4 - 11, 1 и 5 - 8, 1 и 6 - 9, 1 и 7 - 8; 2 и 3 - 10, 2 и 4 - 12, 2 и 5 - 12, 2 и 6 - 9, 2 и 7 - 10, 3 и 4 - 11, 3 и 5 - 13, 3 и 6 - 13, 3 и 7 - 10, 4 и 5 - 12, 4 и 6 - 14, 4 и 7 - 14, 5 и 6 - 13, 5 и 7 - 8, 6 и 7 - 14.
			
			2) Рассмотрим вершины правой доли: у вершин 8 и 9 ровно один общий сосед - 1, 8 и 10 - 7, 8 и 11 - 1, 8 и 12 - 5, 8 и 13 - 5, 8 и 14 - 7, 9 и 10 - 2, 9 и 11 - 1, 9 и 12 - 2, 9 и 13 - 6, 9 и 14 - 6, 10 и 11 - 3, 10 и 12 - 2, 10 и 13 - 3, 10 и 14 - 7, 11 и 12 - 4, 11 и 13 - 3, 11 и 14 - 4, 12 и 13 - 5, 12 и 14 - 4, 13 и 14 - 6.
		\end{solution}






	
	\section*{№8}
	
	Докажите, что
	R(3
	,
	4) = 9
	. (R
	(3
	,
	4)
	— наименьшее из тех чисел
	n
	, что в любом графе на
	n
	вершинах
	есть либо клика размера 3, либо независимое множество размера 4.)
	
	\begin{proof}
		Докажем, что $R(3,4) > 8$. Существует граф на 8 вершинах такой, что в нём нет ни клики размера 3, ни независимого множества размера 4. Пример такого графа изображён ниже. А так как существует граф на 8 вершинах,  который не удовлетворяет условию, то существуют графы на $x<8$ вершинах, которые тоже не удовлетворяют условию. Таким образом, $R(3,4) > 8$.
		
		\includegraphics{граф8}
		
		На лекции доказано, что у произвольной вершины $v$ графа на $N_0=R(k-1, n) + R(k, n-1)$ вершине есть либо $R(k-1, n)$ соседей, либо $R(k,n-1)$ несоседей. В нашем случае у произвольной вершины $v$ есть либо $R(2,4)$ соседей, либо $R(3,3)$ несоседей. 
		
		Докажем, что $R(2, x) = x$. Заметим, что в графе есть клика размера 2, если в графе есть ребро. Однако максимальный размер графа, в котором нет рёбер и нет независимого множества размера $x$  $x-1$ (такой граф состоит из  $x-1$ изолированной вершины), а значит в любом графе, в котором больше $x-1$ вершины либо нет рёбер, но тогда в нём есть хотя бы $x$ изолированных вершин, либо есть ребро. Таким образом, $R(2, x) = x$, те $R(2, 4) = 4$.
		
			Докажем, что $R(3,3) = 6$. Существует граф на 5 вершинах такой, что в нём нет ни клики размера 3, ни независимого множества размера 3. Пример такого графа изображён ниже.
			
					\includegraphics{граф8_1}
			
			 А так как существует граф на 5 вершинах,  который не удовлетворяет условию, то существуют графы на $x<5$ вершинах, которые тоже не удовлетворяют условию. Таким образом, $R(3,3) > 5$. По доказанному $R(2, x) = x$, значит $R(2,3) = 3$. Найдём $R(3, 2)$.Очевидно, что $R(3, 2)$ больше 2 (тк в графе меньше 3 вершин и нет клики размера 3). Заметим, что если в графе нет независимого множества размера 2, то все вершины графа соединены ребром, тогда граф с 3 вершинами - полный, значит в нём есть клика размера 3. Если в графе с 3 вершинами нет клики размера 3, он не полный, тогда в нём есть независимое множество размера 2. Таким образом, $R(3, 2)$ = 3. Тк у произвольной вершины $v$ графа на $N_0=R(k-1, n) + R(k, n-1)$ вершине есть либо $R(k-1, n)$ соседей, либо $R(k,n-1)$, то у произвольной вершины графа на $N_0=R(2, 3) + R(3, 2) = 3 + 3 = 6$ вершинах есть либо $R(2, 3) = 3$ соседа, либо $R(3,2)=3$ несоседа. Рассмотрим граф на 6 вершинах и зафиксируем в нём вершину $x$. Возможны 2 случая: у $x$ есть 3 соседа, либо 3 несоседа.
			
			1) у $x$ есть 3 соседа
			
				Тогда $x$ соединён ребром с 3 вершинами и имеет 2 несоседа (тк всего в графе 6 вершин). Обозначим 2 несоседей как $a$ и $b$. Если $a$ и $b$ не соединены ребром, то в графе есть независимое множество размера 3 - $x, a, b$, иначе $a$ и $b$ соединены ребром и не соединены с $x$.  (на рисунке $x$ - 1, $a, b$ - 5, 6)
				
				\includegraphics{граф8_2}
				
				Тогда если среди вершин 2, 3, 4 хотя бы 2 соединены ребром, в графе есть клика размер 3, состоящая из этих двух вершин и вершины $x$, а иначе есть независимое множество размера 3, состоящее из вершин 2, 3, 4. Таким образом, если у $x$ есть 3 соседа, то в графе на 6 вершинах есть либо клика размера 3, либо независимое множество размера 3.
				
			2) у $x$ есть 3 несоседа
			
			Тогда $x$ соединён ребром с 2 вершинами и имеет 3 несоседа.
			
			\includegraphics{граф8_3}
			
			Если вершины 5 и 6 соединены ребром (не ограничивая общности, $x$ соединёна ребром с вершинами 5 и 6), то в графе есть клика размера 3 - 156. Если вершины 5 и 6 не соединены ребром, то среди вершин 2, 3, 4 есть либо клика размера 3, либо независимое множество размера 2 (доказано ранее), (если есть клика, то к графе на 6 вершинах есть клика размера 3), пусть есть независимое множество размера 2. Тогда это множество в объединении с вершиной $x$ (1) даёт независимое множество размера 3 в графе с 6 вершинами. Таким образом, если у $x$ есть 3 несоседа, то в графе на 6 вершинах есть либо клика размера 3, либо независимое множество размера 3.
			
			Из 1) и 2) следует, что в графе на 6 вершинах есть либо клика размера 3, либо независимое множество размера 3, а $R(3,3) > 5$, то $R(3,3) = 6$.
			
		Теперь докажем, что $R(3,4) \leq 9 = R(2, 4)  +R(3, 3) - 1= 4 + 6 - 1$. Заметим, что в графе на 9 вершинах сумма всех степеней верших графа чётна (тк равно удвоенному число рёбер), а значит найдётся вершина $x$ с чётной степенью. Обозначим степень вершины $x$ как $d_x$ ($d_x$ - чёт). Обозначим через $A$ - множество соседй $x$, через $B$ - множество несоседей. Тогда $|A| = d_x, |B| = 9 - d_x - 1 = 8 - d_x$. Заметим, что $|A|$ и $|B|$ - чёт. Тк в графе $9 = |A| + |B| + 1 = $ вершин, по принципу Дирихле либо $|A| \geq 3$, либо $|B| \geq 6$. При этом $|A|,|B| $ - чёт и тогда либо $|A| \geq 4$, либо $|B| \geq 6$. Значит в графе на 9 вершинах есть вершина $x$, которая имеет чётную степень и либо 4 соседа, либо 6 несоседей.
		
		1) Пусть у этой вершины (на рисунке $x$ - 1) есть 4 соседа. Тогда у этой вершины есть так же 4 несоседа.
		
			\includegraphics{граф8_4}
			
			Заметим, что если хотя бы 2 вершины из 2, 3, 4, 5 соединены ребром, то в графе на 9 вершинах есть клика размера 3. Если ни одна из этих 4 вершин не соединена между собой, то они образуют независимое множество размера 4. Таким образом, если у $x$ есть 4 соседа, то в графе на 9 вершинах есть либо клика размера 3, либо независимое множество размера 4.
			
		2) Пусть у этой вершины (на рисунке $x$ - 1) есть 2 соседа. Тогда у этой вершины есть так же 6 несоседа.
		
			\includegraphics{граф8_5}
			
			Заметим, что подграф, состоящий из 6 вершин 3, 4, 5, 6, 7, 8, 9 (и рёбер между этими вершинами), имеет либо клику размера 3, либо независимое множество размера 3 (доказано ранее), если есть клика, то исходный граф имеет клику размера 3, если клики нет, то искомый граф имеет независимое множество, состоящее из этих 3 вершин и вершины $x$. Таким образом, если у $x$ есть 6 несоседей, то в графе на 9 вершинах есть либо клика размера 3, либо независимое множество размера 4.
			
		Таким образом, в любом графе на
		9
		вершинах
		есть либо клика размера 3, либо независимое множество размера 4. $R(3, 4) > 8 \Rightarrow R(3, 4) = 9$.
	
		
		%Покажем, что в графе на 9 вершинах есть либо клика размера 3, либо независимое множество размера 4. 1) Пусть в таком графе нет независимого множества размера 4. Тогда среди любых 4 вершин хотя бы две соединены ребром.
	\end{proof}
	
	
	
	
	
	
	
	
	
	
	
	
	
	
	
\end{document}