\documentclass[a4paper, 16pt]{article}

\usepackage[utf8]{inputenc}

\usepackage[russian, english]{babel}
\usepackage{subfiles}
\usepackage[utf8]{inputenc}
\usepackage[T2A]{fontenc}
\usepackage{ucs}
\usepackage{textcomp}
\usepackage{array}
\usepackage{indentfirst}
\usepackage{amsmath}
\usepackage{amssymb}
\usepackage{enumerate}
\usepackage[margin=1.2cm]{geometry}
\usepackage{authblk}
\usepackage{tikz}
\usepackage{icomma}
\usepackage{gensymb}
\usepackage{graphicx}

\DeclareGraphicsExtensions{,.png,.jpg}

\graphicspath{{pictures/}}

\renewcommand{\baselinestretch}{1.4}

\renewcommand{\C}{\mathbb{C}}
\newcommand{\N} {\mathbb{N}}
\newcommand{\Q} {\mathbb{Q}}
\newcommand{\Z} {\mathbb{Z}}
\newcommand{\R} {\mathbb{R}}
\newcommand{\ord} {\mathop{\rm ord}}
\newcommand{\Ima}{\mathop{\rm Im}}
\newcommand{\rk}{\mathop{\rm rk}}

\renewcommand{\r}{\right}
\renewcommand{\l}{\left}
\renewcommand{\inf}{\infty}
\newcommand{\Sum}[2]{\overset{#2}{\underset{#1}{\sum}}}
\newcommand{\Lim}[2]{\lim\limits_{#1 \rightarrow #2}}
\newcommand\tab[1][1cm]{\hspace*{#1}}

\newcommand{\task}[1] {\noindent \textbf{Задача #1.} \hfill}
\newcommand{\note}[1] {\noindent \textbf{Примечание #1.} \hfill}
\newenvironment{proof}[1][Доказательство]{%
	\begin{trivlist}
		\item[\hskip \labelsep {\bfseries #1:}]
		\item \hspace{14pt}
	}{
		$ \hfill\blacksquare $
	\end{trivlist}
	\hfill\break
}
\newenvironment{solution}[1][Решение]{%
	\begin{trivlist}
		\item[\hskip \labelsep {\bfseries #1:}]
		\item \hspace{15pt}
	}{
	\end{trivlist}
}

\newenvironment{answer}[1][Ответ]{%
	\begin{trivlist}
		\item[\hskip \labelsep {\bfseries #1:}] \hskip \labelsep
	}{
	\end{trivlist}
	\hfill
}

\title{Дискретная математика} 
\date{\today}
\author{Сидоров Дмитрий}
\affil{Группа БПМИ 219}


\begin{document}
	\maketitle
	
	\section*{№1}
	
		Найдите наибольшее количество вершин в связном графе, сумма степеней вершин в котором равна 20.
		
		\begin{solution}
			Тк необходимо найти наибольшое число вершин в связном графе, то все вершины графа являются мостами, значит такой граф является деревом. Сумма степеней всех вершин графа равна удвоенному числу его рёбер. Значит в искомом графе $\frac{20}{2} = 10$ рёбер. В дереве число вершин на 1 больше числа рёбер, значит, в дереве, в котором сумма степеней вершин равна 20 11 вершин. Примером такого графа является дерево, в котором 2 вершины имеют степень 1, а 9 имеют степень 2 (1 вершина соединена только с 2, 2 с 1 и 3, 3 с 2 и 4, ..., 10 соединена с 11 и 9, 11 соединена только с 10).
		\end{solution}
	
		\begin{answer}
			11
		\end{answer}

	\section*{№2}
	
		В связном графе на n вершинах нет мостов. Какое наименьшее количество рёбер может быть в таком графе?
		
		\begin{solution}			
			Если в связном графе нет мостов, то после удаления любого ребра в графе количество компонент связности не меняется. Заметим, что в графе не может быть меньше $n-1$ ребра (тк иначе граф не связен). Если в графе ровно $n-1$ ребро, то граф является деревом, а в дереве все рёбра - это мосты, что противоречит условию. Таким образом, в таком графе не может быть меньше $n$ рёбер. Пример такого графа с $n$ рёбрами - это граф, в котором вершина 1 соединена с вершиной 2 и $n$, вершина 2 соединена с вершинами 1 и 3, ..., вершина $n-1$ соединена с вершинами $n-2$ и $n$, вершина $n$ соединена с вершинами 1 и $n-1$ (все вершины графа образуют простой цикл). В таком графе нет мостов, тк удаление любой вершины не меняет количество компонент связности: она остаётся единственной.
		\end{solution}
	
		\begin{answer}
			n
		\end{answer}
	
	\section*{№3}
	
		В дереве на 13 вершинах есть ровно две вершины степени 6. Следует ли из этого, что в этом дереве
		есть вершина степени 2? Приведите пример дерева на 14 вершинах, в котором есть ровно две вершины
		степени 6 и нет ни одной вершины степени 2.
		
		\subsection*{a)}
			\begin{solution}	
				В дереве число вершин на 1 больше числа рёбер, значит, в дереве, в котором 13 вершин 12 рёбер. Сумма степеней всех вершин графа равна удвоенному числу его рёбер, значит, в этом дереве сумма степеней всех вершин равна 24. В этом дереве есть ровно 2 вершины степени 6 (усл), значит, сумма степеней оставшихся 11 вершин равна 24 - 12 = 12. В дереве все вершины связаны, т. е. степень каждой вершины больше 0, значит, по принципу Дирихле найдётся вершина степени 2.
			\end{solution}
		
			\begin{answer}
				да
			\end{answer}
		
		\subsection*{b)}
		
			\begin{solution}	
			Если в дереве 14 вершин, то в нем 13 рёбер, а сумма степеней всех вершин равна 26. Таким образом, в этом дереве 14 вершин, 2 вершины имеют степень 6, тогда сумма степеней 12 оставшихся вершин равна 14, т. к. в графе нет ни одной вершины степени 2, то все оставшиеся вершины дерева, кроме одной (которая имеет степень 3), имеют степень 1. Пример такого графа:
			
			\includegraphics{граф2}
			\end{solution}
		
		\section*{№4}
		
			В связном графе степени всех вершин чётные. Докажите, что граф останется связным и после
			удаления любого из рёбер.
			
			\begin{proof}
				Пусть после удаления ребра, соединяющего вершины $x, y$ граф стал несвязным. Тогда в графе нет путя, содержащего $x, y$, а степени $x, y$ стали нечет (тк до удаления ребра они были чёт, но уменьшились на 1). Рассмотрим компаненту связности $x$. В ней все вершины, кроме $x$, имеют чётные степени (а $x$ имеет нечёт степень). Тогда сумма всех степеней в этой компаненте связности нечет, а это противоречит тому, что сумма степеней всех вершин графа равна удвоенному числу его рёбер, т. е. является чётной. Таким образом, предположение неверно и после удаления любого из рёбер графа граф останется связным.
			\end{proof}
	
		\section*{№5}
		
			Докажите, что если в графе больше 5 вершин, либо сам граф, либо его дополнение содержат цикл
			длины 3. Приведите пример такого графа на 5 вершинах, что ни граф, ни его дополнение не содержат
			цикла длины 3.
			
			\begin{proof}
				Заметим, что граф и его дополнение образуют полный граф (для каждой вершины сумма рёбер, выходящих из вершины в начальном графе, и рёбер, выходящих из вершины в дополнении графа, равна $n-1$, где $n$ - кол-во вершин в графе). Выберем произвольную вершину $A$. В графе больше 5 вершин, т. е. 6 и более, значит, по принципу Дирихле, либо в начальном графе, либо в дополнении графа степень $A$ больше или равна 3. 
				
				1) Пусть степень вершины $A$ в начальном графе меньше 3, значит, по доказанному, степень $A$ в дополнении графа больше или равна 3. Обозначим 3 вершины, с которыми соединена $A$ в дополнении графа, как $X, Y, Z$. Заметим, что если в дополнении графа есть хотя бы одно ребро $XY, YZ, ZX$, то существует цикл длины 3 (либо $AXY$, либо $AYZ$, либо $AZX$), а если таких рёбер нет, то в начальном графе $X, Y, Z$ соединены рёбрами, т. е. существует цикл $XYZ$. Значит либо в начальном графе, либо в его дополнении есть цикл длины 3. 
				
				2) Пусть степень вершины $A$ в начальном графе больше 3. Тогда аналогично 1) в начальном графе существуют 3 вершины $X, Y, Z$, с которыми соединена $A$, и либо в начальном графе существует цикл длины 3 (либо $AXY$, либо $AYZ$, либо $AZX$), либо в дополнении графа существует цикл $XYZ$.
				
				Таким образом, либо в начальном графе, либо в его дополнении есть цикл длины 3, если в графе больше 5 вершин. 
			\end{proof}
				Пример графа на 5 вершинах, такого что ни граф, ни его дополнение не содержат цикла длины 3. (Каждый из графов содержит только цикл длины 5)
		
				\includegraphics{граф5}
		
		\section*{№6}
		
			В дереве нет вершин степени 2. Докажите, что количество висячих вершин (т.е. вершин степени 1)
			больше половины общего количества вершин.
			
			\begin{proof}
				В дереве число вершин на 1 больше числа рёбер. Пусть в дереве $n$ вершин, тогда рёбер в дереве $n-1$. Значит сумма степеней вершин в дереве равна $2n-2$ (удвоенное число рёбер). Необходимо доказать, что в дереве больше половины вершин (т. е. больше $\frac{n}{2}$) имеют степень 1. Пусть это не так, тогда в дереве степень 1 имеют $x \leq \frac{n}{2}$ вершины. Значит сумма степеней оставшихся $n - x$ вершин равна $2n - 2 - x$. Заметим, что оставшиеся вершины могут иметь только степень, не меньшую 3 (0 не могут, тк дано дерево, 1, 2 по усл). Значит сумма степеней вершин не меньше чем $3(n-x) = 3n - 3x$. Сравним $3n - 3x$ и $2n-2-x$: $3n - 3x \lor 2n-2-x$;	 
				 $n + 2 \lor 2x;$  $x \leq \frac{n}{2} \Rightarrow 2x \leq n \Rightarrow n + 2 > 2x \Rightarrow 3n - 3x > 2n-2-x$ Таким образом, получили, что минимальная возможная сумма степеней вершин в дереве больше суммы степеней вершин в графе, т. е. получили противоречие. Значит предположение неверно, и в дереве больше половины вершин имеют степень 1.
			\end{proof}
			
		\section*{№7}	
			В дереве на 2021 вершине нет простого пути длины 6. Докажите, что в этом дереве есть вершина степени не меньше 33.
			\begin{proof}
				%Пусть в этом дереве нет вершины степени не меньше 33 $\Rightarrow$ степени всех вершин меньше 33. В графе с 2021 вершиной 2020 рёбер. Значит сумма степеней вершин в этом дереве равна 4040.
				Если дереве на 2021 вершине нет простого пути длины 6, то в дереве длина всех путей меньше или равна 5.  Рассмотрим путь самой большой длины в этом дереве. Заметим, что если степени первой и последней вершины в этом пути не равны 1, то в дереве есть путь большей длины, чем выбранный (тк в дереве нет циклов, то если степени этих вершин больше 1, то эти вершины будут соединены ещё хотя бы с одной вершиной не из выбранного пути, и найдется путь длинее выбранного), таким образом, первая и последняя вершина в этом пути имеют степень 1. Пусть длина самого длинного пути в дереве равна 5. Пронумеруем вершины в этом пути как 1, 2, 3, 4, 5, 6. 
				
				\includegraphics{граф7.1}
				
				По доказанному степени вершин 1 и 6 равны 1. Построим граф с максимальным количеством вершин, в котором степень каждой вершины меньше 33 и нет простого пути длины 6. Покажем, что в этом графе будет меньше 2021 вершины (таким образом, если в начальном графе не было бы пути длины 5, то в нём было бы меньше вершин, чем в построенном графе, т. е. тоже меньше 2021). 
				Тк необходимо построить граф с максимальным количеством вершин, пусть степень вершин 2 и 5 равны 32, т. е. вершины 2 и 5 соединим еще с 30 вершинами (каждую) (причём ни одна из этих 60 не соединена ни с одной другой из остальных 59, тк иначе в графе не все рёбра будут мостами), в графе стало $6 + 30*2 = 66$ вершин. Заметим, что степень новых вершин должна быть 1, тк иначе существуют пути длины 6 (например, путь 65432XY, где X - новая вершина, соединеная с 2, а Y - вершина, соединеная с X, отличная от 2, аналогично 12345XY, где X - новая вершина, соединеная с 5, а Y - новая вершина, соединеная с X, отличная от 5). Аналогично соединим каждую из вершин 3, 4 еще с 30 новыми вершинами (1), а каждую из 60 полученных вершин еще с 31 новой вершиной (2). Заметим, что 60 новых вершин имеют степень 32 и соединены либо с 3, либо с 4, и еще с 31 висячей вершиной. Также заметим, что больше в граф нельзя добавить вершины, тк иначе будет существовать простой путь длины 6, например, ZXY4321 (где X - вершина, полученная на шаге (2), а Y - вершина, полученная на шаге (1), а Z - вершина, соединеная с X, отличная от Y) (или простой путь ZXY3456). Таким образом, в этом графе $6 + 2\cdot30 + 2 \cdot 30 \cdot 31 = 1926 < 2021$ вершин. Значит,  необходимо добавить в граф еще несколько вершин, причем, тк в дереве нет простого пути длины 6, эти вершины можно добавить только к вершинам с степенью 32, значит в графе будет вершина с степенью не меньше 33. Таким образом, в графе с простым путем длины 5 (и в котором нет вершины степени не меньше 33) максимальное количество вершин равно 1926 < 2021, значит в графах, в которых нет просто пути длины 5 (и в которых нет вершины степени не меньше 33) количество вершин меньше 1926 $\Rightarrow$ меньше 2021, а значит если в дереве на 2021 вершине нет пути длины 6, то в этом дереве есть вершина степени не меньше 33.
				
				%Среди всех графов с суммой степеней вершин 4040 рассмотрим такой, в котором максимальная длина просто пути минимальна. В таком графе максимальное количество вершин должно иметь максимальную степень, т. е. степень 32. Построим такой граф. 
			\end{proof}
			

\end{document}