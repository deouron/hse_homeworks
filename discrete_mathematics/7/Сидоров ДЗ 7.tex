\documentclass[a4paper, 16pt]{article}

\usepackage[utf8]{inputenc}

\usepackage[russian, english]{babel}
\usepackage{subfiles}
\usepackage[utf8]{inputenc}
\usepackage[T2A]{fontenc}
\usepackage{ucs}
\usepackage{textcomp}
\usepackage{array}
\usepackage{indentfirst}
\usepackage{amsmath}
\usepackage{amssymb}
\usepackage{enumerate}
\usepackage[margin=1.2cm]{geometry}
\usepackage{authblk}
\usepackage{tikz}
\usepackage{icomma}
\usepackage{gensymb}
\usepackage{graphicx}

\DeclareGraphicsExtensions{,.png,.jpg}

\graphicspath{{pictures/}}

\renewcommand{\baselinestretch}{1.4}

\renewcommand{\C}{\mathbb{C}}
\newcommand{\N} {\mathbb{N}}
\newcommand{\Q} {\mathbb{Q}}
\newcommand{\Z} {\mathbb{Z}}
\newcommand{\R} {\mathbb{R}}
\newcommand{\ord} {\mathop{\rm ord}}
\newcommand{\Ima}{\mathop{\rm Im}}
\newcommand{\rk}{\mathop{\rm rk}}

\renewcommand{\r}{\right}
\renewcommand{\l}{\left}
\renewcommand{\inf}{\infty}
\newcommand{\Sum}[2]{\overset{#2}{\underset{#1}{\sum}}}
\newcommand{\Lim}[2]{\lim\limits_{#1 \rightarrow #2}}
\newcommand\tab[1][1cm]{\hspace*{#1}}

\newcommand{\task}[1] {\noindent \textbf{Задача #1.} \hfill}
\newcommand{\note}[1] {\noindent \textbf{Примечание #1.} \hfill}
\newenvironment{proof}[1][Доказательство]{%
	\begin{trivlist}
		\item[\hskip \labelsep {\bfseries #1:}]
		\item \hspace{14pt}
	}{
		$ \hfill\blacksquare $
	\end{trivlist}
	\hfill\break
}
\newenvironment{solution}[1][Решение]{%
	\begin{trivlist}
		\item[\hskip \labelsep {\bfseries #1:}]
		\item \hspace{15pt}
	}{
	\end{trivlist}
}

\newenvironment{answer}[1][Ответ]{%
	\begin{trivlist}
		\item[\hskip \labelsep {\bfseries #1:}] \hskip \labelsep
	}{
	\end{trivlist}
	\hfill
}

\title{Дискретная математика} 
\date{\today}
\author{Сидоров Дмитрий}
\affil{Группа БПМИ 219}


\begin{document}
	\maketitle
	
	\section*{№1}
	
		Про функцию
		$f$
		из множества
		$X$
		в множество
		$Y$
		и множество
		$B$
		$\subseteq$
		$Y$
		известно, что
		$f^{-1}$
		$(B) =X$
		.
		Верно ли, что
		$B=Y$?
		
		\begin{solution}
			Например, для множеств $X = \{1\}, \ Y = \{2, 3\}$ и $B=\{2\}$ и $f(1)=2$ верно, что $f^{-1} (B) =X$, но при этом $B \ne Y$.
		\end{solution}
	
		\begin{answer}
			нет
		\end{answer}
	
	\section*{№2}
	
		Функция
		f
		определена на множестве
		X
		и принимает значения в множестве
		Y
		, при этом
		A
		$\cup$
		B
				$\subseteq$
		X
		и
		$f(A) = f(B)$. Верно ли, что при этих условиях
		$f^{-1} (f(A)) = f^{-1}(f(B))$?
		Приведите доказательство или контрпример в каждом случае.
	
		\begin{solution}
			Для любого $a \in A$ элемент $f(a)$ принадлежит $f(A)$, а значит $a$ принадлежит прообразу множества $f(A)$. Таким образом, $A \subseteq f^{-1} (f(A))$ и аналогично $B \subseteq f^{-1} (f(B))$. По условию $f(A) = f(B)$, значит $\forall a \in A \ \exists b \in B: f(a) = f(b) = z$. Пусть все такие $z$ образуют множество $Z$, тогда $f^{-1} (f(a)) = f^{-1}(f(b)) = f^{-1}(Z)$.
		\end{solution}
	
		\begin{answer}
			Верно
		\end{answer}
	
	\section*{№3}
	
		Функция
		$f$
		определена на множестве
		$A \cup B$
		и принимает значения в множестве
		$Y$
		. Если заменить в
		утверждении
		$f(A \bigtriangleup B)$ ?
		$f(A) \bigtriangleup f(B)$
		знак
		?
		на один из знаков включения
		$\subseteq$
		или
		$\supseteq$
		, получится утверждение. Какие из получившихся двух
		утверждений верны для любой
		$f$
		? Приведите доказательство или контрпример в каждом случае.
		
		\begin{solution}
			Контрпример для $f(A \bigtriangleup B)$ $\subseteq$ $f(A) \bigtriangleup f(B)$:
			Пусть $A=\{1, 2, 3, 4\}, B=\{4, 5\} (\Rightarrow A \cup B = \{1, 2, 3, 4, 5\}), f(A) = \{1, 2, 3\} (f(1) = 1, f(2) = 2, f(3) = 3, f(4) = 3), f(B) = \{3, 5\} (f(4) = 3, f(5) = 5)	$, таким образом, $(A \bigtriangleup B = \{1, 2, 3, 5\} \Rightarrow f(A \bigtriangleup B) = \{1, 2, 3, 5\}$, но $f(A) \bigtriangleup f(B) = \{1, 2, 3\} \bigtriangleup \{3, 5\} = \{1, 2, 5\}$ и $f(A \bigtriangleup B)$ $\subseteq$ $f(A) \bigtriangleup f(B)$, те $\{1, 2, 3, 5\}$ $\subseteq$ $\{1, 2, 3\}$ - ложь.
			
			Докажем, что  $f(A \bigtriangleup B)$ $\supseteq$ $f(A) \bigtriangleup f(B)$ верно для любой $f$. Для любого $a \in (f(A) \setminus f(B)): \  f(a)^{-1} \in (A \setminus B)$, аналогично для любого $b \in (f(B) \setminus f(A)): \  f(b)^{-1} \in (B \setminus A)$, а значит $f(A \bigtriangleup B)$ $\supseteq$ $f(A) \bigtriangleup f(B)$.
			%\includegraphics{№3}
			
		\end{solution}
	
		\begin{answer}
			$f(A \bigtriangleup B)$ $\supseteq$ $f(A) \bigtriangleup f(B)$
		\end{answer}
	
	\section*{№4}
	
		О функциях
		f
		из множества
		A
		в множество
		B
		и
		g
		из множества
		B
		в множество
		C
		известно, что
		g
		◦
		f
		биекция. Верно ли, что
		g
		всюду определена? (Множества
		A
		,
		B
		,
		C
		не обязательно конечные.)
	
		\begin{solution}
			Неверно, тк существует контрпример: пусть множества $A, B, C$ совпадают с множеством натуральных чисел, а $f(x) = x  +1, g(x) = x - 1$. Тогда $f(1)=2, f(2)=3, f(3)=4, f(4)=5, \dots$, а $g(2)=1, g(3)=2, g(4)=3, g(5)=4, \dots$, таким образом, $g $ ◦ $ f = g(f(x))$ - биекция, тк $g(f(1)) = 1, g(f(2)) = 2, g(f(3)) = 3, \dots$, но при этом $g(1)$ не определена.
		\end{solution}
	
		\begin{answer}
			Неверно
		\end{answer}
	
	\section*{№5}
	
		О всюду определённых функциях
		f
		,
		g
		из множества
		A
		в себя известно, что
		f ◦ g ◦ f
		= id
		A
		. Верно ли,
		что
		f
		— биекция? (Множество
		A
		не обязательно конечное.)
		
		\begin{solution}
			$(f$ ◦ $g)$ ◦ $f$ = $f$ ◦ $(g$ ◦ $f)$, тк композиция функций ассоциативна, $f(g(x))$ ◦ $f$ = $f$ ◦ $g(f(x)) = id_A$
			Функция является биекцией тогда и только тогда, когда она является инъекицей и сюръекцией. Заметим, что $f$ является левой и правой обратной, а значит является инъекцией и сюръекцией (доказано на семинаре), а значит $f$ - биекция.
		\end{solution}
	
		\begin{answer}
			 Верно
		\end{answer}
	
	\section*{№6}
	
		О функциях
		f
		,
		g
		из множества
		A
		в себя (не обязательно всюду определённых) известно, что
		g
		◦
		f
		всюду
		определённая. Множество
		A
		состоит из 2021 элемента. Найдите минимально возможное количество
		элементов в образе
		$f$
		◦
		$g(A).$
		
		\begin{solution}
			%Пусть $|f(A)| = x$. При этом $|f(g(A))| \leq |g(A)|$, $|g(f(A))| \leq |g(A)|$, $|g(f(A))| \leq |f(A)|=x$
			 %и $|f(g(A))| \leq |f(A)| = x$ (тк $f(g(A))$ содержится в $f(A)$, $g(f(A))$ содержится в $g(A)$). По условию $g$
		%◦
			%$f$, $g(f(A))$
			%всюду определена, те функция $g$ определена на каждом элементе множества $f(A)$, те $|g(f(A))| \geq x$, а значит функция $g(A)$ определена на не менее чем $x$  элементах, те $|g(A)| \geq x$,
			 %и не определена на не более чем $2021-x$ элементах. Заметим, что $|g(f(A))| \leq |f(A)|=x$ и $|g(f(A))| \geq x$, значит $|g(f(A))| = x$. Это означает, что $g(f(A))$ определена ровно на $x$ элементах и не определена на $2021 - x$ элементах.
			 
			 Покажем, что минимально возможное количество
			 элементов в образе
			 $f$
			 ◦
			 $g(A)$ равно 1. Количество
			 элементов в образе
			 $f$
			 ◦
			 $g(A)$ не равно 0, тк $g(f(A))$
			 всюду определена, те функция $g$ определена на каждом элементе множества $f(A)$, в том числе  $f(g(A))$. Пример, в котором количество
			 элементов в образе
			 $f$
			 ◦
			 $g(A)$ равно 1: пусть множество $А = \{1, 2, 3, \dots, 2021\} $ (состоит из 2021 элемента), $f(1) = 1$, $g(1) = 1$, и $f$ и $g$ определены только в точке 1, тогда $g$ ◦ $f$ всюду определена, и количество
			 элементов в образе
			 $f$
			 ◦
			 $g(A)$ равно 1.
		\end{solution}
	
		\begin{answer}
			1
		\end{answer}
	
	
	\section*{№7}
	
		В графе на
		$n$
		вершинах для любой пары вершин
		$u$
		и
		$v$
		есть ровно две вершины, с которыми соединены
		и
		$u$
		, и
		$v$. Докажите, что степени всех вершин в этом графе одинаковы.
		
		\begin{proof}
			%Докажем индукцией по числу вершин. Для графа на 4 вершинах выполняется (пример ниже, степень каждой вершины равна 3).
			%\includegraphics{граф7}
			%Пусть выполняется для $k$ вершин. Докажем, что выполняется для $k+1$ вершины. Пусть это не так. Тогда в графе есть $k$ вершин, степени которых равны $d_x$, и одна вершина $y$, степень которой равна $d_y, \ d_y \ne d_x$. По условию для любой пары вершин
			%$u$
			%и $v$ есть ровно две вершины, с которыми соединены
			%и
			%$u$, и
			%$v$, а значит для любой вершины $x$ из $k$ вершин с степенью $d_x$ и вершины $y$ есть ровно 2 вершины, с которомы соединены и $x$, и $y$.
			
			По условию для любой пары вершин
			$u$
			и
			$v$
			есть ровно две вершины, с которыми соединены
			и
			$u$
			, и
			$v$. Заметим, что если выбрать произвольные вершины $a, b$, то в графе есть ровно 2 вершины $x, y$, с которыми соедиенны и $a$, и $b$, а значит вершины $a, b, x, y$ образуют простой цикл длины 4.
			
			Выберем в графе произвольную вершину $x$, обозначим её степень $d_x$ (те из $x$ выходит $d_x$ рёбер). Заметим, что если выбрать произвольную пару ребёр, которые исходят из вершины $x$ (пусть это рёбра $xa, xb$), то существует вершина $y$ такая, что существуют рёбра $ay, by$ (тк по условию для любой пары вершин
			$u$
			и
			$v$
			есть ровно две вершины, с которыми соединены
			и
			$u$
			, и
			$v$), те образуется цикл длины 4 с вершиной $x$. Заметим, что пару рёбер из $x$ можно выбрать $\frac{d_x(d_x - 1)}{2}$ способами, а значит $x$ входит в $\frac{d_x(d_x - 1)}{2}$ простых цикла длины 4.
			
			Пусть в графе $k$ штук простых циклов длины 4. Заметим, что каждая вершина графа входит в одно и то же число простых циклов длины 4 (по условию), а значит каждая вершина входит в $\frac{4k}{n}$ цикла длины 4. Вернёмся к ранее выбранной вершине $x$: заметим, что для $x$ $\frac{4k}{n}=\frac{d_x(d_x - 1)}{2}$. Заметим, что из этого уравнения можно найти $d_x$, причём $d_x^2 - d_x - \frac{8k}{n}= 0$, а значит $d_x$ зависит только от $k$ и $n$ и не зависит от выбранной вершины $x$, те какую бы вершину $x$ мы не выбрали, её степень будет одна и та же. Таким образом, степени всех вершин в этом графе одинаковы.
			
		\end{proof}
	
	
	
	
	
	
	
	
	
	
	
	
	
	
\end{document}