\documentclass[a4paper, 16pt]{article}

\usepackage[utf8]{inputenc}

\usepackage[russian, english]{babel}
\usepackage{subfiles}
\usepackage[utf8]{inputenc}
\usepackage[T2A]{fontenc}
\usepackage{ucs}
\usepackage{textcomp}
\usepackage{array}
\usepackage{indentfirst}
\usepackage{amsmath}
\usepackage{amssymb}
\usepackage{enumerate}
\usepackage[margin=1.2cm]{geometry}
\usepackage{authblk}
\usepackage{tikz}
\usepackage{icomma}
\usepackage{gensymb}
\usepackage{graphicx}

\DeclareGraphicsExtensions{,.png,.jpg}

\graphicspath{{pictures/}}

\renewcommand{\baselinestretch}{1.4}

\renewcommand{\C}{\mathbb{C}}
\newcommand{\N} {\mathbb{N}}
\newcommand{\Q} {\mathbb{Q}}
\newcommand{\Z} {\mathbb{Z}}
\newcommand{\R} {\mathbb{R}}
\newcommand{\ord} {\mathop{\rm ord}}
\newcommand{\Ima}{\mathop{\rm Im}}
\newcommand{\rk}{\mathop{\rm rk}}

\renewcommand{\r}{\right}
\renewcommand{\l}{\left}
\renewcommand{\inf}{\infty}
\newcommand{\Sum}[2]{\overset{#2}{\underset{#1}{\sum}}}
\newcommand{\Lim}[2]{\lim\limits_{#1 \rightarrow #2}}
\newcommand\tab[1][1cm]{\hspace*{#1}}

\newcommand{\task}[1] {\noindent \textbf{Задача #1.} \hfill}
\newcommand{\note}[1] {\noindent \textbf{Примечание #1.} \hfill}
\newenvironment{proof}[1][Доказательство]{%
	\begin{trivlist}
		\item[\hskip \labelsep {\bfseries #1:}]
		\item \hspace{14pt}
	}{
		$ \hfill\blacksquare $
	\end{trivlist}
	\hfill\break
}
\newenvironment{solution}[1][Решение]{%
	\begin{trivlist}
		\item[\hskip \labelsep {\bfseries #1:}]
		\item \hspace{15pt}
	}{
	\end{trivlist}
}

\newenvironment{answer}[1][Ответ]{%
	\begin{trivlist}
		\item[\hskip \labelsep {\bfseries #1:}] \hskip \labelsep
	}{
	\end{trivlist}
	\hfill
}

\title{Дискретная математика} 
\date{\today}
\author{Сидоров Дмитрий}
\affil{Группа БПМИ 219}


\begin{document}
	\maketitle
	
	\section*{№1}
	
		Сколькими способами можно распределить 26 больных по 8 палатам (в каждой палате помещается не менее 26 человек)?
	
		\begin{solution}
			Для каждого больного существует 8 палат, в которые его можно распределить. Таким образом, для каждого из 26 больных есть 8 вариантов распределения, а значит всего способов $8^{26}$.
		\end{solution}
	
		\begin{answer}
			$8^{26}$
		\end{answer}
	
	\section*{№2}
	
		Сколько есть 4-значных чисел, в десятичной записи которых есть хотя бы одна цифра 0?
		
		\begin{solution}
			Всего 4-значных чисел 10000 - 1000 = 9000. Из них в $9^4 = 6561$ нет ни одной цифры 0. Таким образом, 4-значных чисел, в десятичной записи которых есть хотя бы одна цифра 0, $9000-6561 = 2439$.
		\end{solution}
	
		\begin{answer}
			2439
		\end{answer}
	
	\section*{№3}
	
		Докажите, что количество двоичных слов длины
		n
		(последовательностей длины
		n
		, составленных из
		нулей и единиц), в которых не встречается двух нулей подряд, равно
		(
		n
		+ 1)
		-му числу Фибоначчи
		$F_{n+1}$
		.
		(
		$F_0$
		= 1
		;
		$F_1$
		= 1
		;
		$F_{n+2}$
		=
		$F_{n+1}$
		+
		$F_n$
		для всех
		n
		$\geq$
		2
		.)
		
		\begin{proof}
			Докажем, используя метод мат индукции. При $n=1$ кол-во двоичных слов равно 2 (1 и 2), при $n=2$ 3 (11, 10, 01), при $ n=3$  5 (111, 110, 101, 011, 010). Пусть для $n = k$ количество двоичных слов длины
			$k$, в которых не встречается двух нулей подряд, равно $F_{k + 1} = F_k + F_{k-1}$. Рассмотрим двоичное слово длины $n = k + 1$. Оно может оканчиваться либо на 0, либо на 1. 
			
			1) оканчивает на 0. Тогда предпоследний элемент последовательности равен 1 (тк 2 нуля не идут подряд), а значит количество таких слов равно количеству слов длины $k + 1 - 2 = k - 1$ (тк $k$ - ый и $k + 1$-ый элементы однозначно определены), а их $F_{k - 1+1} = F_k = F_{k-1} + F_{k-2}$. 
			
			2) оканчивается на 1. Тогда предпоследний элемент последовательности может быть любым, а значит количество таких слов равно количеству слов длины $k + 1 - 1 = k$, и количество таких слов равно количеству слов длины $k$, а их $F_{k + 1} = F_k + F_{k-1}$.
			
			Таким образом, слов длины $k + 1$ $F_k + F_{k +1} = F_{k+2} = F_{(k+1) + 1}$. Значит по методу мат индукции количество двоичных слов длины
			$n$, в которых не встречается двух нулей подряд, равно
			($n+ 1$)
			-му числу Фибоначчи
			$F_{n+1}$.
		\end{proof}
	
	\section*{№4}
	
		Каких чисел больше среди первого триллиона ($10^{12}$) целых неотрицательных чисел: тех, в десятичной
		записи которых есть цифра
		1
		, или тех, в десятичной записи которых этой цифры нет?
		
		\begin{solution}
			%Заметим, что 12-значных чисел, в которых нет 1 $8 \cdot 9^{11} = 
			Задачу можно переформулировать так: "Необходимо посчитать количество десятичных последовательностей длины 12 (отбросили $10^{12}$, тк оно содержит 1), в которых нет цифры 1", те любому $n$ значному числу, где $n \leq 12$, будет соответствовать последовательность вида: к $12 - n$ нулям приписываем это число. Заметим, что тогда таких последовательностей будет $9^{12} = 282429536481$, что меньше $\frac{10^{12}}{2}$, а значит последовательнотей, в которых нет 1 меньше, чем тех, в которых 1 есть, а значит и чисел среди первого триллиона, в десятичной
			записи которых нет цифры 1, меньше, те чисел, в которых 1 есть, больше.
		\end{solution}
	
		\begin{answer}
			В которых есть 1
		\end{answer}
	
	\section*{№5}
	
		Сколькими способами можно закрасить клетки таблицы
		3
		×
		4
		так, чтобы незакрашенные клетки
		содержали или верхний ряд, или нижний ряд, или две средних вертикали?
		
		\begin{solution}
			Рассмотрим 3 случая, в которых точно выполняет одно условие.
			
			1) Верхний ряд незакрашен. Тогда любая из оставшихся 8 клеток может быть закрашена, либо незакрашена, а значит таких способом $2^8 =256$.
			
			2) Нижний ряд незакрашен. Аналогично с 1) есть 256 способов.
			
			3) 2 средние вертикали незакрашены. Тогда любая из оставшихся 6 клеток может быть закрашена, либо незакрашена, а значит таких способом $2^6 =64$.
			
			В каждом из случаев точно выполняется одно условие, но при этом может выполняться условие, которое на текущем шаге не рассматривалось. Таким образом, необходимо найти количество способов раскрасить таблицу так, что точно будут выполняться 2 условия, и количество способов раскрасить таблицу так, что будут выполняться все 3 условия. Тогда ответ будет равен: сумма из рассмотренных случаев 1), 2), 3) - способы с 2 условиями + способы с 3 условиями (другими словами, можно представить случаи как 3 множества и найти их объединение, те применить формулу включений и исключений).
			
			1) Верхний ряд и нижний ряд. Остаётся 4 клетки, способов раскраски $2^4 = 16$.
			
			2) Верхний ряд и 2 средних вертикали. Остаётся 4 клетки, способов раскраски $2^4 = 16$.
			
			3) Нижний ряд и 2 средних вертикали. Аналогично 2), 16 способов.
			
			4) Верхний ряд, нижний ряд и 2 средних вертикали. Остаётся 2 клетки, 4 способа.
			
			Таким образом, способов, в которых незакрашенные клетки
			содержали или верхний ряд, или нижний ряд, или две средних вертикали 256 + 256 + 64 - 16 - 16 - 16 + 4 = 532.			
		\end{solution}
	
		\begin{answer}
			532
		\end{answer}
	
	\section*{№6}
		
		Для полета на Марс набирают группу людей, в которой каждый должен владеть хотя бы одной
		из профессий повара, медика, пилота или астронома. При этом в техническом задании указано, что
		каждой профессией из списка должно владеть ровно 6 человек в группе. Кроме того указано, что
		в группе должен найтись ровно один человек, владеющий всеми этими профессиями; каждой парой
		профессий должны владеть ровно 4 человека; каждой тройкой — ровно 2.
		Выполнимо ли такое техническое задание?
		
		\begin{solution}
			Обозначим множества поваров, медиков, пилотов, астрономов как ПО, М, ПИ, А. Тогда |ПО| = |М| = |ПИ| = |А| = 6. 
			
			При этом |ПО| $\cap$ |М| $\cap$ |ПИ| $\cap$ |А| = 1,
			
			 |ПО| $\cap$ |М| = |ПО| $\cap$ |А| = |ПО| $\cap$ |ПИ| = |М| $\cap$ |ПИ| = |М| $\cap$ |А| = |ПИ| $\cap$ |А| = 4
			 
			 Аналогично для троек = 2.
			 
			 Тогда по формуле включений и исключений |ПО $\cup$ М $\cup$ ПИ $\cup$ А| = 4 $\cdot$ 6 -  6 $\cdot$ 4 + 4 $\cdot$ 2 - 1 $\cdot$ 1 = 24 - 24 + 8 - 1 = 7 (тк всего 4 профессии, 6 пар, 4 тройки, 1 четверка, а в паре 4 человека, в тройке 2, в четверке 1). Но при этом |ПО| $\cap$ |М|  = |ПО|  + |М|  - |ПО| $\cap$ |М| = 6 + 6 - 4= 8, что невозможно, тк невозможна ситуация, когда всего 7 человек, а поваров и медиков 8, те больше. Значит техническое задание невыполнимо.
		\end{solution}
	
		\begin{answer}
			Нет
		\end{answer}
	
	\section*{№7}
	
		Сколько двоичных слов длины 9 содержат подслово 011?
		Подслово
		— это последовательность стоящих
		подряд символов.
		
		\begin{solution}
			Выведем формулу $f(n)$ для чисел, которые не содержат подслово 011 (и потом из всех двоичных слов длины 9, те $2^9 = 512$, вычтем количество таких слов). Обозначим длину слова за $n$. Заметим, что в слове не может содержаться подслово длины 3, если его длина меньше 3, те при $n \leq 2$  во всех словах с длиной $n$ нет подслова 011, а значит $f(1) = 2, f(2) = 4$. Пусть $n \geq 3$. Тогда возможны 2 случая: либо слово оканчивается на 0, либо оканчивается на 1.
			
			1) Оканчивается на 0. Значит слово не оканчивается на 011, и в таком случае формула будет выглядеть как $f(n) = f(n - 1)$.
			
			2) Оканчивается на 1. Тогда опять возможны 2 случая. 2.1) Слово оканчивается на 01. Тогда слово не оканчивается на 011, а значит $f(n) = f(n - 2)$. 2.2) Слово оканчивается на 11. Заметим, что в этом случае, тк в числе нет подслова 011, перед 11 может стоять только 1, значит слово оканчивается на 111. Перед 111 так же не может стоять 0, значит слово оканчивается на 1111 итд. Значит, если слово оканчивается на 11, оно состоит только из 1, а значит в этом случае $f(n) = 1$.
			
			Таким образом, получим, что $f(n) = f(n-1) + f(n - 2) + 1$, а тк $f(1) = 2,\ f(2) = 4$, получим, что $f(3) = 7,\ f(4) = 12,\ f(5) = 20,\ f(6) = 33,\ f(7) = 54,\ f(8) = 88,\ f(9) = 143$.  Таким образом, среди всех двоичных слов длины 9 не содержат подслово 011 143, а значит 512 - 143 = 369 двоичных слов длины 9 содержат подслово 011.
		\end{solution}
	
		\begin{answer}
			369
		\end{answer}
	
	
	
	
	
	
	
	
	
	
	
	
	
	
	
	
\end{document}