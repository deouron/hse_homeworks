\documentclass[a4paper, 16pt]{article}

\usepackage[utf8]{inputenc}

\usepackage[russian, english]{babel}
\usepackage{subfiles}
\usepackage[utf8]{inputenc}
\usepackage[T2A]{fontenc}
\usepackage{ucs}
\usepackage{textcomp}
\usepackage{array}
\usepackage{indentfirst}
\usepackage{amsmath}
\usepackage{amssymb}
\usepackage{enumerate}
\usepackage[margin=1.2cm]{geometry}
\usepackage{authblk}
\usepackage{tikz}
\usepackage{icomma}
\usepackage{gensymb}
\usepackage{graphicx}

\DeclareGraphicsExtensions{,.png,.jpg}

\graphicspath{{pictures/}}

\renewcommand{\baselinestretch}{1.4}

\renewcommand{\C}{\mathbb{C}}
\newcommand{\N} {\mathbb{N}}
\newcommand{\Q} {\mathbb{Q}}
\newcommand{\Z} {\mathbb{Z}}
\newcommand{\R} {\mathbb{R}}
\newcommand{\ord} {\mathop{\rm ord}}
\newcommand{\Ima}{\mathop{\rm Im}}
\newcommand{\rk}{\mathop{\rm rk}}

\renewcommand{\r}{\right}
\renewcommand{\l}{\left}
\renewcommand{\inf}{\infty}
\newcommand{\Sum}[2]{\overset{#2}{\underset{#1}{\sum}}}
\newcommand{\Lim}[2]{\lim\limits_{#1 \rightarrow #2}}
\newcommand\tab[1][1cm]{\hspace*{#1}}

\newcommand{\task}[1] {\noindent \textbf{Задача #1.} \hfill}
\newcommand{\note}[1] {\noindent \textbf{Примечание #1.} \hfill}
\newenvironment{proof}[1][Доказательство]{%
	\begin{trivlist}
		\item[\hskip \labelsep {\bfseries #1:}]
		\item \hspace{14pt}
	}{
		$ \hfill\blacksquare $
	\end{trivlist}
	\hfill\break
}
\newenvironment{solution}[1][Решение]{%
	\begin{trivlist}
		\item[\hskip \labelsep {\bfseries #1:}]
		\item \hspace{15pt}
	}{
	\end{trivlist}
}

\newenvironment{answer}[1][Ответ]{%
	\begin{trivlist}
		\item[\hskip \labelsep {\bfseries #1:}] \hskip \labelsep
	}{
	\end{trivlist}
	\hfill
}

\title{Дискретная математика} 
\date{\today}
\author{Сидоров Дмитрий}
\affil{Группа БПМИ 219}


\begin{document}
	\maketitle
	
	\section*{№1}
	
		Докажите, что множество конечных подмножеств рациональных чисел счётно.
		
		\begin{proof}
			Объединение конечного или счётного числа конечных или счётных множеств конечно или счётно, а число конечных подмножеств рациональных чисел счётно. Это можно доказать следующим образом: начинаем выбирать все подмножетсва, сумма которых равна 1, их конечное число, потом считаем те, сумма которых равна 2, их тоже конечное число, итд, таким образом, число конечных множеств подмножеств рациональных чисел счётно. Множество рациональных чисел счётно. Таким образом, множество конечных подмножеств рациональных чисел счётно.
		\end{proof}
	
	\section*{№2}
	
		Функция называется периодической, если для некоторого числа
		$T$
		и любого
		$x$
		выполняется
		$f
		(
		x
		+
		T
		) =
		f
		(
		x
		)$
		. Докажите, что множество периодических функций
		$f
		:
		Z
		\rightarrow
		Z$
		счётно.
		
		\begin{proof}
			Обозначим как $A_T$ множество периодических функций с периодом $T$. Объединение конечного или счётного числа конечных или счётных множеств конечно или счётно, а значит, если в $A_T$ счётное число элементов, то множество периодических функций
			$f
			:
			Z
			\rightarrow
			Z$
			счётно (множество из множеств периодических функций счётно). Докажем это: по определению периодической функции 	$f
			(
			x
			+
			T\documentclass[a4paper, 16pt]{article}
			
			\usepackage[utf8]{inputenc}
			
			\usepackage[russian, english]{babel}
			\usepackage{subfiles}
			\usepackage[utf8]{inputenc}
			\usepackage[T2A]{fontenc}
			\usepackage{ucs}
			\usepackage{textcomp}
			\usepackage{array}
			\usepackage{indentfirst}
			\usepackage{amsmath}
			\usepackage{amssymb}
			\usepackage{enumerate}
			\usepackage[margin=1.2cm]{geometry}
			\usepackage{authblk}
			\usepackage{tikz}
			\usepackage{icomma}
			\usepackage{gensymb}
			\usepackage{graphicx}
			
			\DeclareGraphicsExtensions{,.png,.jpg}
			
			\DeclareMathOperator*\lowlim{\underline{lim}}
			\DeclareMathOperator*\uplim{\overline{lim}}
			
			\graphicspath{{pictures/}}
			
			\renewcommand{\baselinestretch}{1.4}
			
			\renewcommand{\C}{\mathbb{C}}
			\newcommand{\N} {\mathbb{N}}
			\newcommand{\Q} {\mathbb{Q}}
			\newcommand{\Z} {\mathbb{Z}}
			\newcommand{\R} {\mathbb{R}}
			\newcommand{\ord} {\mathop{\rm ord}}
			\newcommand{\Ima}{\mathop{\rm Im}}
			\newcommand{\rk}{\mathop{\rm rk}}
			
			\renewcommand{\r}{\right}
			\renewcommand{\l}{\left}
			\renewcommand{\inf}{\infty}
			\newcommand{\Sum}[2]{\overset{#2}{\underset{#1}{\sum}}}
			\newcommand{\Lim}[2]{\lim\limits_{#1 \rightarrow #2}}
			\newcommand\tab[1][1cm]{\hspace*{#1}}
			
			\newcommand{\task}[1] {\noindent \textbf{Задача #1.} \hfill}
			\newcommand{\note}[1] {\noindent \textbf{Примечание #1.} \hfill}
			\newenvironment{proof}[1][Доказательство]{%
				\begin{trivlist}
					\item[\hskip \labelsep {\bfseries #1:}]
					\item \hspace{14pt}
				}{
					$ \hfill\blacksquare $
				\end{trivlist}
				\hfill\break
			}
			\newenvironment{solution}[1][Решение]{%
				\begin{trivlist}
					\item[\hskip \labelsep {\bfseries #1:}]
					\item \hspace{15pt}
				}{
				\end{trivlist}
			}
			
			\newenvironment{answer}[1][Ответ]{%
				\begin{trivlist}
					\item[\hskip \labelsep {\bfseries #1:}] \hskip \labelsep
				}{
				\end{trivlist}
				\hfill
			}
			
			\title{Мат анализ} 
			\date{\today}
			\author{Сидоров Дмитрий}
			\affil{Группа БПМИ 219}
			
			
			\begin{document}
				\maketitle
				
				\section*{№1}
				
				\subsection*{a)}
				
				$a_n = \frac{(-1)^n}{n} + \frac{1 + (-1)^n}{2}$
				
				\begin{solution}
					Заметим, что дробь $ \frac{(-1)^n}{n}$ положительна при чётном $n$ и нечётна при нечётном. Если $n$ - чёт, то дробь положительна, числитель чётный, значит максимальное значение дробь принимает при $n=2$: $\frac{1}{2}$. При нечётном  $n$ дробь отрицательна, числитель равен -1, наименьшее значение дробь принимает при $n=1$: $-1 $
					
					Дробь $\frac{1 + (-1)^n}{2}$ может принимать только 2 значения: $\frac{2}{2} = 1$ при чётном $n$ и 0 при нечётном. Таким образом  $\lowlim\limits_{n \to \inf} a_n = -1 + 0 = -1$, $\uplim\limits_{n \to \inf} a_n = \frac{1}{2} +1 = 1.5$
				\end{solution}
				
				\begin{answer}
					$\lowlim\limits_{n \to \inf} a_n = -1$, $\uplim\limits_{n \to \inf} a_n = 1.5$
				\end{answer}
				
				\subsection*{b)}
				
				$a_n = \frac{n}{n+1} \cdot \sin^2(\frac{\pi n}{4})$
				
				\begin{solution}
					$ \sin^2(\frac{\pi n}{4})$ не больше 1 и не меньше 0, значит $\uplim\limits_{n \to \inf} a_n = \lim\limits_{n \to \inf}  \frac{n}{n+1} =  \lim\limits_{n \to \inf}  \frac{1}{1+\frac{1}{n}} = 1$, например, при $n=8k-2$.
					
					Заметим, что $ \sin^2(\frac{\pi n}{4}) \geq 0$ и $\frac{n}{n+1} \geq 0 \Rightarrow a_n \geq 0$ Например, при $n = 4k: \  a_n = \frac{4k}{4k+1} \cdot 0 = 0 \Rightarrow 	\lowlim\limits_{n \to \inf} a_n = 0$
				\end{solution}
				
				\begin{answer}
					$\lowlim\limits_{n \to \inf} a_n = 0, \uplim\limits_{n \to \inf} a_n = 1$
				\end{answer}
				
				\subsection*{c)}
				
				$a_n = 1 + 2 \cdot (-1)^{n+1} +3 \cdot (-1)^{\frac{n(n-1)}{2}}$
				
				\begin{solution}
					Заметим, что $(-1)^{n+1}$ и $(-1)^{\frac{n(n-1)}{2}}$ могут принимать только значения -1 и 1, значит члены $a_n$ не могут быть меньше $-4 \  (1 - 2 - 3)$ и не могут быть больше $6 \  (1 + 2 + 3)$. Заметим, что при $n=4k+2: (-1)^{n+1} = (-1)^{4k+3} = -1$, а $(-1)^{\frac{n(n-1)}{2}} = (-1)^{\frac{(4k+2)(4k+1)}{2}} = (-1)^{(2k+1)(4k+1)} = -1$, значит $a_n$ может принимать значение -4, например, при $n=4k+2$. При $n=4k+1: (-1)^{4k+2} = (-1)^2$, а $(-1)^{\frac{(4k+1)(4k)}{2}} = (-1)^{2p}=1$, значит $a_n$ может принимать значение 6. Таким образом, 	$\lowlim\limits_{n \to \inf} a_n = -4, \uplim\limits_{n \to \inf} a_n = 6$
				\end{solution}
				
				\begin{answer}
					$\lowlim\limits_{n \to \inf} a_n = -4, \uplim\limits_{n \to \inf} a_n = 6$
				\end{answer}
				
				\section*{№2}
				
				\subsection*{a)} 
				\begin{answer}
					$a_n = \{-1,\ 0,\ 1,-1,\ 0,\ 1,-1,\ 0,\ 1,\dots\}$
				\end{answer}
				
				\subsection*{b)} 
				\begin{answer}
					$a_n = n$
				\end{answer}
				
				\subsection*{c)}
				\begin{solution}
					Множество рациональных чисел счётно (д-во: можно написать таблицу, в которой первая строка - дроби с знаменателем 1: $0, 1, -1, 2, -2, 3 \dots$, вторая - дроби с знаменателем 2: $\frac{0}{2}, \frac{1}{2}, \frac{-1}{2}, \frac{2}{2}, \frac{-2}{2}, \frac{3}{2}, \dots$ итд. Далее в таблице можно выбрать числа "змейкой" так, что образуется последовательность, в которую входят все рациональные числа: $0, 1, \frac{1}{2}, \frac{0}{2}, \frac{0}{2}, \frac{0}{3}, \frac{1}{3}, \frac{-1}{3}, \frac{-1}{2}, -1, 2, \frac{2}{2}, \frac{2}{3}, \dots$). В такую последовательность входят все рациональные числа и можно выбрать подпоследовательность, сходящуюся к любому рациональному числу, тк в $\varepsilon$-окрестность любого рационального числа входит бесконечное количество рациональных чисел.
				\end{solution} 	
				\begin{answer}
					$0, 1, \frac{1}{2}, \frac{0}{2}, \frac{0}{2}, \frac{0}{3}, \frac{1}{3}, \frac{-1}{3}, \frac{-1}{2}, -1, 2, \frac{2}{2}, \frac{2}{3}, \dots$
				\end{answer}
				
				\section*{№3}
				
				\begin{solution}
					Из №2 с) в множестве рациональных чисел каждый член является частичным пределом, таким образом, множеством частичных пределов последовательности $\{a_n\}_{n=1}^{\inf}$ является отрезок $[0;1]$.
				\end{solution}
				
				\begin{answer}
					$[0;1]$
				\end{answer}
				
				
				
				
				
				
				
				
				
				
				
				
				
				
				
				
				
				
				
			\end{document}
		