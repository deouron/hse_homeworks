\documentclass[a4paper, 16pt]{article}

\usepackage[utf8]{inputenc}

\usepackage[russian, english]{babel}
\usepackage{subfiles}
\usepackage[utf8]{inputenc}
\usepackage[T2A]{fontenc}
\usepackage{ucs}
\usepackage{textcomp}
\usepackage{array}
\usepackage{indentfirst}
\usepackage{amsmath}
\usepackage{amssymb}
\usepackage{enumerate}
\usepackage[margin=1.2cm]{geometry}
\usepackage{authblk}
\usepackage{tikz}
\usepackage{icomma}
\usepackage{gensymb}
\usepackage{graphicx}

\DeclareGraphicsExtensions{,.png,.jpg}

\graphicspath{{pictures/}}

\renewcommand{\baselinestretch}{1.4}

\renewcommand{\C}{\mathbb{C}}
\newcommand{\N} {\mathbb{N}}
\newcommand{\Q} {\mathbb{Q}}
\newcommand{\Z} {\mathbb{Z}}
\newcommand{\R} {\mathbb{R}}
\newcommand{\ord} {\mathop{\rm ord}}
\newcommand{\Ima}{\mathop{\rm Im}}
\newcommand{\rk}{\mathop{\rm rk}}

\renewcommand{\r}{\right}
\renewcommand{\l}{\left}
\renewcommand{\inf}{\infty}
\newcommand{\Sum}[2]{\overset{#2}{\underset{#1}{\sum}}}
\newcommand{\Lim}[2]{\lim\limits_{#1 \rightarrow #2}}
\newcommand\tab[1][1cm]{\hspace*{#1}}

\newcommand{\task}[1] {\noindent \textbf{Задача #1.} \hfill}
\newcommand{\note}[1] {\noindent \textbf{Примечание #1.} \hfill}
\newenvironment{proof}[1][Доказательство]{%
	\begin{trivlist}
		\item[\hskip \labelsep {\bfseries #1:}]
		\item \hspace{14pt}
	}{
		$ \hfill\blacksquare $
	\end{trivlist}
	\hfill\break
}
\newenvironment{solution}[1][Решение]{%
	\begin{trivlist}
		\item[\hskip \labelsep {\bfseries #1:}]
		\item \hspace{15pt}
	}{
	\end{trivlist}
}

\newenvironment{answer}[1][Ответ]{%
	\begin{trivlist}
		\item[\hskip \labelsep {\bfseries #1:}] \hskip \labelsep
	}{
	\end{trivlist}
	\hfill
}

\title{Дискретная математика} 
\date{\today}
\author{Сидоров Дмитрий}
\affil{Группа БПМИ 219}


\begin{document}
	\maketitle
	
	\section*{№1}
	
		Вероятность события $A$ равна 0.8, вероятность события $B$ равна 0.5, а вероятность события $A \cup B$
		равна 0.9. Найдите условную вероятность $Pr[A | B]$.
		
		\begin{solution}
			$Pr[A|B] = \frac{Pr[A \cap B]}{Pr[B]}$. Найдём $Pr[A \cap B]$. $Pr[A] + Pr[B] - Pr[A \cap B] = Pr[A \cup B] \Rightarrow Pr[A \cap B] = Pr[A] + Pr[B] - Pr[A \cap B] = 1.3 - 0.9 = 0.4$. Таким образом, $Pr[A|B] = \frac{Pr[A \cap B]}{Pr[B]} = \frac{0.4}{0.5} = \frac{4}{5}$.
		\end{solution}
	
		\begin{answer}
			$ \frac{4}{5}$
		\end{answer}
	
	\section*{№2}

		Выбирается случайное подмножество $S$ множества целых чисел от -10 до 10 (включительно), все исходы равновозможны. Найдите математическое ожидание суммы модулей чисел, входящих в $S$.
		
		\begin{solution}
			Для каждого числа от -10 до 10 вероятность вхождения в подмножество равна $\frac{1}{2}$. Значит математическое ожидание суммы модулей чисел, входящих в $S$, равно $\sum\limits_{i = -10}^{10} |i| \cdot \frac{1}{2} = \frac{1}{2}(0 + 2 \cdot (1 + 2 + \dots + 10)) = \frac{1}{2} \cdot 2 \cdot \frac{1 + 10}{2} \cdot 10 = 55$
		\end{solution}
	
		\begin{answer}
			55
		\end{answer}
	
	\section*{№3}
	
		О неотрицательных случайных величинах $X, Y$ известно, что $Pr[X > 1] \geq \frac{2}{3}, \ Pr[Y > 1] \geq \frac{2}{3}$. Докажите, что $E[XY ] \geq \frac{1}{3}$.
		
		\begin{proof}
			%По неравенству Маркова $Pr[X > 1] < \frac{E[X]}{1} = E[X]$ и $Pr[Y > 1] < \frac{E[Y]}{1} = E[Y]$, значит $E[X] > \frac{2}{3}, E[Y] > \frac{2}{3}$.
			%Заметим, что $Pr[XY > 1] \geq \frac{2}{3} \cdot \frac{2}{3} = \frac{4}{9}$ (тк если $X > 1$ и $Y > 1$, то $XY > 1$). Значит $Pr[XY \geq 1] \geq \frac{4}{9}$. Тогда по неравенству Маркова, $Pr[f \geq \alpha] \leq \frac{E[f]}{\alpha}$, $Pr[XY \geq 1] \leq \frac{E[XY]}{1} = E[XY]$, и тк $Pr[XY \geq 1] \geq \frac{4}{9}$, $E[XY] \geq \frac{4}{9} \geq \frac{1}{3}$.
			$Pr[X > 1] + Pr[Y > 1] - Pr[X > 1 \cap Y > 1] = Pr[X > 1 \cup Y > 1] \leq 1 \Rightarrow Pr[X > 1 \cap Y > 1] \geq Pr[X > 1] + Pr[Y > 1] - 1 \geq \frac{2}{3} + \frac{2}{3} - 1 = \frac{1}{3}$. Тогда, тк $X \geq 0, Y\geq0$, $E[XY] \geq 1 \cdot \frac{1}{3} + 0 \cdot \frac{2}{3} = \frac{1}{3}$
		\end{proof}
	
	\section*{№4}
	
		Пусть у нас есть 4 различных шара, которые мы равновероятно и случайно раскладываем по трем
		различным коробкам. Найдите вероятность того, что все коробки непустые при условии, что в первой
		коробке лежит ровно один шар. Ответ должен быть дан в виде простой несократимой дроби.
		
		\begin{solution}
			
		\end{solution}
			Если в первой коробке лежит ровно один шар, то в оставшихся двух коробках лежат 3 различных шара. Тогда всего возможных вариантов распределить 3 шара среди 2 коробок $2^3 = 8$. Если во второй и в третьей коробках лежит хотя бы один шар, а всего оставшихся шаров 3, то в одной коробке лежит 1 шар, а во второй 2. Выбрать коробку с 2 шарами можно 2 способами, 2 из 3 шаров для неё можно выбрать $C_3^2 = 3$ способами, значит благоприятных исходов $2 \cdot 3 = 6$ (в коробку с одним шаром кладём один оставшийся шар, те один способ). Тогда вероятность того, что все коробки непустые при условии, что в первой
			коробке лежит ровно один шар, равна $\frac{6}{8} = \frac{3}{4}$.
		\begin{answer}
			$\frac{3}{4}$
		\end{answer}
	
	\section*{№5}
	
		О событиях $A$ и $B$ вероятностного пространства $U$ известно, что $0 < Pr[A] < 1$ и $0 < Pr[B] < 1$.
		Могут ли при этом события $A \cup B$ и $B$ быть независимыми?
		
		\begin{solution}
			События $A \cup B$ и $B$ независимы, если $Pr[(A \cup B) \cap B] = Pr[A \cup B] \cdot Pr[B]$. При этом $Pr[(A \cup B) \cap B]  = Pr[B]$, а значит события $A \cup B$ и $B$ независимы, если $Pr[B] = Pr[A \cup B] \cdot Pr[B]$. Пример, в котором события $A \cup B$ и $B$ могут быть независимыми: Пусть монетку подбрасывают 1 раз, событие $A$ = "выпал орёл", $B$ = "выпала решка". Тогда $0 < Pr[A] = \frac{1}{2} < 1$ и $0 < Pr[B]=\frac{1}{2} < 1$. При этом $Pr[A \cup B] = 1 \Rightarrow Pr[B] = \frac{1}{2} = 1 \cdot \frac{1}{2} = Pr[A \cup B] \cdot Pr[B]$, а значит события $A \cup B$ и $B$ независимы.
		\end{solution}
	
		\begin{answer}
			могут
		\end{answer}
	
	\section*{№6}
	
		Вероятностное пространство — двоичные последовательности длины 42, все исходы равновозможны.
		Случайная величина $L$ равна количеству 1 среди первых 28 членов последовательности, случайная
		величина $R$ равна количеству 1 среди последних 14 членов последовательности. Найдите $E[LR]$. Ответ
		должен быть дан в виде простой несократимой дроби.
		
		\begin{solution}
			Веротяность того, что элемент двоичной последовательности длины 42 равен 1, равна $\frac{1}{2}$. Пусть $a_i = 1$, если $i$-ый элемент последовательности равен 1, и $a_i = 0$ иначе. Таким образом, $E[L] = \sum \limits_{i = 1} ^{28} a_i = \frac{28}{2} = 14$ и $E[R] = \sum \limits_{i = 29} ^{42} a_i = \frac{14}{2} = 7$. При этом по определению величины $L, R$ независимы, если для любых $x, y$ события $L = x$ и $R = y$ независимы, значит, тк  события $L = x$ и $R = y$ независимы (тк все исходы равновозможны и позиции членов для $L$ и $R$ не пересекаются (случайная величина $L$ равна количеству 1 среди первых 28 членов последовательности, случайная
			величина $R$ равна количеству 1 среди членов последовательности от 29-го до 42-го)), то $E[L]$ и $E[R]$ независимы, а значит $E[LR] = E[L] \cdot E[R] = 14 \cdot 7 = 98$.
		\end{solution}
	
		\begin{answer}
			98
		\end{answer}
	
	\section*{№7}
	
		Докажите, что вероятность того, что случайное рефлексивное и симметричное отношение, определенное на множестве из $n$ элементов, будет отношением эквивалентности, стремится к 0 при $n$, стремящемся к бесконечности. (Распределение на рефлексивных и симметричных отношениях равномерное.)
		
		\begin{proof}
			%Отношение является отношением эквивалентности, если оно одновременно рефлексивно, симметрично и транзитивно, таким образом, случайное рефлексивное и симметричное отношение, определенное на множестве из $n$ элементов, будет отношением эквивалентности, если оно будет транзитивным. %Небходимо найти веротяность события $A=$"отношение, определенное на множестве из $n$ элементов, будет отношением эквивалентности" при условии$B=$ "отношение, определенное на множестве из $n$ элементов, является рефлексивным и симметричным". $Pr[A | B] = \frac{Pr[A \cap B]}{Pr[B]}$. Найдем $Pr[B]$ и $Pr[A \cap B]$.
			%Таким образом, для отношения эквивалентности необходимо, чтобы случайное рефлексивное и симметричное отношение было транзитивно. Найдём вероятность того, что случайное отношение транзитивно.	
			 %Рассмотрим отношение на элементах $a, b, c$ и найдём количество отношений, которые рефлексивные и транзитивные на этом 3-элементном множестве. Всего отношений $2^3 = 8$ (каждый элемент либо входит, либо не входит в отношение). При этом отношение не транзитивных отношений будет не больше 3 (для тройки элементов $a, b, c$ отношение не будет транзитивным отношением, если пары $(a, b), (b, c)$ входят в отношение, а пара $(1, c)$ не входит, всего 6 упорядоченных пар, )
			 
			 По теореме о отношениях эквивалентности "Любое отношение R, являющееся отношением эквивалентности на
			 множестве A, делит A на классы эквивалентности — непересекающиеся подмножества множества A, при этом любые два элемента одного класса находятся в
			 отношении R, а любые два элемента разных классов не находятся в отношении R". Таким образом, количество отношений эквивалентности на множестве из $n$ элементов равно числу способов разбить $n$ элементов на непустые попарно непересекающиеся подмножества. Покажем, что таких разбиений не больше, чем $n!$. Для этого покажем, как каждому разбиению можно сопоставлять перестановку из $n$  элементов, причём каждой перестановке будет сопоставлено не более 1 разбиения (те получится инъекция).  Пусть $a_1 < a_2 < \dots < a_{n_1},\ b_1 < b_2 < \dots < b_{n_2}, \ \dots$ - некоторое разбиение множества. Тогда сопоставим ему перестановку, в которой подмножества упорядочены по возрастанию наименьших элементов (те если $a_1 < b_1$ и нет такого элемента $x_1$, что $a_1 < x_1 < b_1$, то перестановка выглядит как $a_1 a_2  \dots  a_{n_1} b_1  b_2  \dots b_{n_2} \ \dots$). Например, если множество состоит из чисел от 1 до $n$, и оно разбивается на подмножества $\{1, 3, 5\}, \{2, 6, 7\}, \{4, 8\}, \dots,$, то такому разбиению соответствует перестановка 13526748... Тогда разбиений не больше, чем $n!$, тк такое сопоставление инъективно, и из каждой перестановки можно однозначно получить разбиение множества на непустые попарно непересекающиеся подмножества (по перестановке идём слева направо, пока текущий элемент больше предыдущего. Если текущий элемент меньше предыдущего, то все элементы слева от текущего образуют подмножество и стираются из перестановки. Продолжаем, пока не дойдём до последнего элемента, тогда все оставшиеся образуют последнее подмножество). Таким образом, количество отношений эквивалентности на множестве из $n$ элементов равно числу способов разбить $n$ элементов на непустые попарно непересекающиеся подмножества и не больше $n!$. 
			 
			Найдём число рефлексивных и симметричных отношений. В рефлексивное отношение входят пары вида $(x, x), 1 \leq x \leq n, x \in N$, а остальные $n^2 - n$ могут входить или не входить. Тк отношение ещё и симметрично, то, если в отношение входит пара $(a, b)$, то в отношение так же вхоит пара $(b, a)$, а значит всего $2^{\frac{n(n-1)}{2}}$ рефлексивных и симметричных отношений (каждая из $\frac{n(n-1)}{2}$ пары либо входит, либо нет). Таким образом, вероятность того, что случайное рефлексивное и симметричное отношение, определенное на множестве из $n$ элементов, будет отношением эквивалентности, не больше $\frac{n!}{2^{\frac{n(n-1)}{2}}} \leq \frac{n^n}{2^{\frac{n(n-1)}{2}}} =
			\frac{2^{n \cdot \log n}}{2^{\frac{n(n-1)}{2}}} \to 0$ при $n \to \inf$, тк $\frac{2n \cdot \log n}{n(n-1)} = \frac{2 \log n}{n - 1}\to 0$ при $n \to \inf$.
			 
		\end{proof}
	
	
	
	
	
	
	
	
\end{document}