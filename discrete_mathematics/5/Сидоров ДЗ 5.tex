\documentclass[a4paper, 16pt]{article}

\usepackage[utf8]{inputenc}

\usepackage[russian, english]{babel}
\usepackage{subfiles}
\usepackage[utf8]{inputenc}
\usepackage[T2A]{fontenc}
\usepackage{ucs}
\usepackage{textcomp}
\usepackage{array}
\usepackage{indentfirst}
\usepackage{amsmath}
\usepackage{amssymb}
\usepackage{enumerate}
\usepackage[margin=1.2cm]{geometry}
\usepackage{authblk}
\usepackage{tikz}
\usepackage{icomma}
\usepackage{gensymb}
\usepackage{graphicx}

\DeclareGraphicsExtensions{,.png,.jpg}

\graphicspath{{pictures/}}

\renewcommand{\baselinestretch}{1.4}

\renewcommand{\C}{\mathbb{C}}
\newcommand{\N} {\mathbb{N}}
\newcommand{\Q} {\mathbb{Q}}
\newcommand{\Z} {\mathbb{Z}}
\newcommand{\R} {\mathbb{R}}
\newcommand{\ord} {\mathop{\rm ord}}
\newcommand{\Ima}{\mathop{\rm Im}}
\newcommand{\rk}{\mathop{\rm rk}}

\renewcommand{\r}{\right}
\renewcommand{\l}{\left}
\renewcommand{\inf}{\infty}
\newcommand{\Sum}[2]{\overset{#2}{\underset{#1}{\sum}}}
\newcommand{\Lim}[2]{\lim\limits_{#1 \rightarrow #2}}
\newcommand\tab[1][1cm]{\hspace*{#1}}

\newcommand{\task}[1] {\noindent \textbf{Задача #1.} \hfill}
\newcommand{\note}[1] {\noindent \textbf{Примечание #1.} \hfill}
\newenvironment{proof}[1][Доказательство]{%
	\begin{trivlist}
		\item[\hskip \labelsep {\bfseries #1:}]
		\item \hspace{14pt}
	}{
		$ \hfill\blacksquare $
	\end{trivlist}
	\hfill\break
}
\newenvironment{solution}[1][Решение]{%
	\begin{trivlist}
		\item[\hskip \labelsep {\bfseries #1:}]
		\item \hspace{15pt}
	}{
	\end{trivlist}
}

\newenvironment{answer}[1][Ответ]{%
	\begin{trivlist}
		\item[\hskip \labelsep {\bfseries #1:}] \hskip \labelsep
	}{
	\end{trivlist}
	\hfill
}

\title{Дискретная математика} 
\date{\today}
\author{Сидоров Дмитрий}
\affil{Группа БПМИ 219}


\begin{document}
	\maketitle
	
	\section*{№1}
	
	Вершины ориентированного графа — целые числа от 0 до 9. Ребро идет из вершины x в вершину y если y - x = 2
	или
	x - y
	= 3
	. Найдите количество компонент сильной связности в этом графе
	
		\begin{solution}
			Ребро идёт из вершины $x$ в вершину $y$, если $y-x=2$ или $x-y=3$. Таким образом, из вершины $x$ идёт ребро(а) в вершину(ы) $x+2$ и $x-3$. Изобразим такой граф.
			
			\includegraphics{граф1}
			
			Заметим, что в графе есть цикл 1352468579630241, в который входят все вершины графа. Таким образом, каждая вершина графа сильно связана с каждой другой вершиной, т.~е.~ граф имеет одну компоненту сильной связности.
		\end{solution}

		\begin{answer}
			1
		\end{answer}
	
	\section*{№2}
		Известно, что в ориентированном графе на
		$\geq$
		2
		вершинах из любой вершины в любую другую идёт
		ровно один простой путь. Верно ли, что исходящие степени вершин в этом графе равны 1?
		
		\begin{solution}
			
			Неверно. Например, в графе на рисунке ниже 3 $\geq 2$ вершины, из вершины 1 в 3 идёт ровно один простой путь 13, в 2 ровно один простой путь 132, из 3 в 1 31, из 3 в 2 32, из 2 в 3 23, из 2 в 1 132, но при этом исходящая степень вершины 3 равна 2.
			
			\includegraphics{граф2}
			
		\end{solution}
	
		\begin{answer}
			неверно
		\end{answer}
	
	\section*{№3}
	
		Турниром
		называется такой ориентированный граф, в котором нет петель и для любых двух различных вершин
		x
		,
		y
		есть ровно одно ребро с концами
		x
		,
		y
		.
		Докажите, что в любом турнире есть вершина, из которой достижима любая вершина турнира.
		
		\begin{proof}
			Докажем с помощью индукции по числу вершин. При $n=2$ в графе есть 2 вершины $x, y$, которые соединены ребром. Возможны 2 случая: $x$ достижима из $y$ или $y$ достижима из $x$, тогда либо из $y$ (в 1 случае), либо из $x$ (во 2 случае) достижима любая вершина турнира.
			
			Пусть в турнире с $n=k$ вершинами есть вершина, из которой достижима любая вершина турнира. Рассмотрим турнир с $n=k+1$ вершиной. В этом турнире можно выделить турнир с $k$ вершинами, в котором есть вершина, из которой достижима любая вершина турнира. Обозначим её как $A$, а $k+1$-ую вершину турнира обозначим как $B$. Тк граф - турнир, то существует ровно одно ребро с концами $A$ и $B$. Возможны 2 варианта: $A$ достижима из $B$, либо $B$ достижима из $A$.
			
			1) Если $A$ достижима из $B$, то тк из $A$ достижимы все остальные вершины турнира, то из $B$ достижима любая вершина турнира.
			
			2) Если $B$ достижима из $A$, то тк из $A$ достижимы все остальные вершины турнира, то из $A$ достижимы все остальные вершины турнира и вершина $B$, те любая вершина турнира.
			
			Таким образом, в графе с $n=k+1$ вершиной есть вершина, из которой достижима любая вершина турнира, а значит по методу мат инд в любом турнире есть вершина, из которой достижима любая вершина турнира.
		\end{proof}
	
	\section*{№4}
	
		Пусть в ориентированном графе
		G
		исходящая степень каждой вершины равна входящей. Если стереть ориентацию на рёбрах, то получится связный неориентированный граф
		H. Докажите, что ориентированный граф
		G
		сильно связен.
		
		\begin{proof}
			По условию в ориентированном графе
			G
			исходящая степень каждой вершины равна входящей, а значит в графе H степень каждой вершины чётна. Граф H связен по условию. Значит граф H содержит эйлеров цикл. 
				
				Пусть G не сильно связен, значит все вершины G образуют как минимум 2 компоненты сильной связности. Рассмотрим компоненты связности $A$ и $B$, которые не совпадают. Компоненты сильной связности не пересекаются, значит для любой вершины $a \in A$ и $b \in B$ либо не существует пути из $a$ в $b$, либо не существует пути из $b$ в $a$ (либо обоих). Однако граф H содержит эйлеров цикл, а значит в H хотя бы по одной вершине (обозначим как $a$ и $b$) из 2 этих компонент имеют общее ребро, а тогда в G существует путь из $a$ в $b$, либо из $b$ в $a$ (по направлению ребра из одной компоненты в другую, пусть это путь из $a$ в $b$). Покажем, что существует путь из $b$ в $a$. Тк в G исходящая степень каждой вершины равна входящей, а вершины в каждой компоненте сильной связности сильно связны, то каждая компонента образует эйлеров цикл, а тогда, если в $b$ входит ребро из $a$, то из $b$ должно выходить ребро в компоненту сильной связности, отличную от $B$ (тк внутри $B$ у $b$ исходящая степень равна входящей (те чет), в H, те объединении всех компонент, степень $b$ чет, но есть ребро в $a$), а значит для любой компоненты сильной связности в G верно, что в эту компоненту можно попасть из другой и из этой компоненты можно выйти в другую. Таким образом, из $B$ можно попасть $A$ (возможно через другие компоненты), а значит существует путь из $b$ в $a$. Таким образом, все вершины в G сильно связны.
		\end{proof}
	
	\section*{№5}
	
		Таблица из 100 строк и 2 столбцов заполнена числами от 0 до 9 так, чтобы выполнялись условия:
		(а) все строки таблицы различны; (б) ни одну строку в таблице нельзя получить из какой-нибудь вышестоящей строки заменами большего числа на меньшее. Докажите, что какая-то из строк таблицы равна
		(5
		,
		5)
		. Какой по счёту может быть эта строка (начиная с верха)? Укажите все возможные значения и
		докажите корректность приведенного ответа
		
		\subsection*{1)}
		
			\begin{proof}
				Пусть ни одна строка таблицы не равна (5, 5). Тк в таблице 100 строк и 2 столбца, то в таблице 100 пар чисел, причём выполняется условие, что все строки таблицы различны. Из чисел от 0 до 9 можно составить $10 \cdot 10 = 100$ различных пар чисел, но, тк в таблице нет строки (5, 5), то таблицу можно заполнить не более чем 99 различными строками, но в таблице 100 строк, а значит по принципу Дирихле хотя бы одна строка повторяется, что противоречит условию "все строки таблицы различны". Значит какая-то из строк таблицы равна (5, 5).  
			\end{proof}
	
	\subsection*{2)}
	
		\begin{solution}
			Таблица заполнена числами так, что выполняется условие "ни одну строку в таблице нельзя получить из какой-нибудь вышестоящей строки заменами большего числа на меньшее", а значит следующие десятки строк в таблице должны раполагаться подряд, иначе в таблице не может быть 100 различных строк: $(0, 0), (0, 1), \dots, (0,9); (1, 0), (1, 1),$ $(1,2), \dots, (1,9); (2,0), (2,1), (2,9);\dots ; (9,0), \dots, (9, 8), (9,9); (0,0), (1,0), (2,0), \dots (9,0); (0, 1), (1,1),\dots, (1,9); \dots; (9,0), \dots,(9,8),(9,9)$ Для любой пары $x$ и $y$, где $x$ выше $y$, выполняется, что одновременно оба числа в паре $x$ не больше чисел в паре $y$ (соотв), а значит выше строки $(5,5)$ находятся как минимум 50 пар (которые начинаются или заканчиваются цифрами от 0 до 4 включительно, т е $\frac{5 \cdot 10 \cdot 2}{2} = 50$). При этом 51-ая строка равна либо $(5, 0)$, либо $(0, 5)$, и между 50-ой строкой и строкой (5, 5) ещё 4 строки вида $(5, x)$ (либо $(x, 5)$), где $x\in {1, 2, 3, 4}$ (тк если между 51-ой строкой и строкой $(5,5)$ стоят другие строки, то в таблице будет не 100 различных строк, тк будут не все комбинации вида $(a, b)$, где $a, b \in {0,1, 2, 3, 4, 5, 6, 7, 8, 9}$), и строка (5, 5) может быть только 56-ой.
		\end{solution}
	
		\begin{answer}
			56
		\end{answer}
	
	\section*{№6}
	
		Граф
		$K_6$
		состоит из 6 вершин, каждая пара которых соединена ребром. Найдите наименьшую длину
		пути, проходящего по всем рёбрам этого графа. (напомним, что длина пути на 1 меньше количества
		вершин.)
		
		\begin{solution}
			%Неориентированный граф без вершин нулевой степени содержит эйлеров цикл тогда и только тогда, когда он связен и степени всех вершин чётны. В графе К степень каждой вершины равна 5 - нечет (тк граф состоит из 6 вершин, каждая пара которых соединена ребром, т. е. является полным, значит он связен). Таким образом, К не содержит эйлеров цикл, а значит длина пути, проходящего по все рёбрам К не меньше 16 (тк в К $\frac{6\cdot5}{2} = 15$ рёбер). 
			
			
			%Пусть длина наименьшего пути равна 16. Тогда в графе есть одно ребро, по которому прошли 2 раза. Обозначим это ребро как $x$, а его концы как $a$ и $b$. Рассмотрим путь из $a$ в $b$, длиной больше 1 (те рассматриваем путь - это не ребро $ab$), тогда этот путь содержит все рёбра графа (тк по предположению наименьшая длина пути, проходящего по всем рёбрам этого графа, равна 16, и в графе есть ровно одно ребро, по которому прошли 2 раза), но тогда этот путь проходит по каждому ребру этого графа, а его длина равна 15. Те наименьшая длина пути, проходящего по всем рёбрам этого графа, одновременно равна 16 и 15, противоречие, значит наименьшая длина пути, проходящего по всем рёбрам этого графа, не равна 16, те больше или равна 17. 
			
			%Покажем, что в графе нет пути длиной 16, который проходит по всем рёбрам. Пусть такой путь существует. В графе 15 рёбер, а значит одной ребро дважды содержится в таком пути.
			
			Граф $K_6$ состоит из 6 вершин, каждая пара которых соединена ребром, значит граф $K_6$ полный, в нём $\frac{5\cdot6}{2} = 15$ рёбер. Значит длина пути, проходящего по всем рёбрам графа, не меньше 15. Докажем, что если путь проходит через все рёбра графа единожды, то в графе ровно 2 вершины имеют нечет степень. Пусть путь начинается в вершине $A$ и заканчивается в вершине $B$. Тогда степень вершин $A$ и $B$ нечет, тк если путь проходит через вершину $X$, причём $X$ - это не первая и не последняя вершина пути, те отличная от $A$, $B$, то степень $X$ чётна, тк при каждом прохождении через вершину $X$ существует ребро, через которую путь "вошёл" в $X$ и ребро, через которое "вышел", а если путь проходит через каждое ребро единожды, то рёбра не повторяются. Таким образом, степени всех вершин графа, кроме $A$ и $B$ чётны, а степени $A$ и $B$ равны чётному числу + 1, те нечет. Значит если в графе существует путь длины 15, который проходит через все рёбра графа, то в графе ровно 2 вершины имеют нечет степень, а в графе $K_6$ все 6 вершин имеют степень 5, те все вершины имеют нечётную степень. Значит длина искомого пути не равна 15.
			
			Пусть длина такого пути равна 16. Значит такой путь проходит через одно рёбро дважды. Тогда можно добавить одно ребро в граф так, чтобы искомый путь проходил по каждому ребру графа единожды (те если путь проходит дважды через ребро, соединяющее вершины $a$ и $b$, то в новом графе будет 2 ребра, соединяющих вершины $a$ и $b$, и по каждому из них искомый путь пройдет единожды). Тогда степени 2 вершин ($a$ и $b$) чет (равны 5 + 1 = 6), а степени остальных 4 вершин нечет (равны 5), а по доказанному в таком графе не существует пути, который проходит по всем рёбрам графа единожды, те в новом графе с 16 рёбрами не существует такого пути, а значит в $K_6$ нет пути длиной 16, который проходит по всем рёбрам.
			
			Если в $K_6$ есть такой путь длиной 17, то этот путь проходит через 2 рёбра дважды, тогда можно представить $K_6$ как граф, в котором 4 вершины имеют степень 6, а 2 5, те в таком графе существует путь, который проходит через все рёбра единожды (*). Покажем, что в графе $K_6$ есть путь длиной 17, который проходит по всем рёбрам. В полном графе с 5 вершинами (на рисунке это часть исходного графа без вершины 6 и рёбер, которым принадлежит эта вершина) $\frac{5\cdot4}{2} = 10$ ребёр, каждая вершина имеет степень 4 - чёт, значит в таком графе есть цикл, который проходит по всем рёбрам ровно 1 раз, те длина такого цикла 10.
			
			\includegraphics{граф6}
			
			Тк в таком графе есть цикл, то, не ограничивая общности, можно сказать, что цикл начинается и заканчивается в вершине 1. Тогда в исходном графе есть путь 16236456 длиной 7, который проходит по всем остальным рёбрам графа (* те граф, который является таким представлением $K_6$, имеет все вершины и рёбра $K_6$, а  также еще по одному ребру, которые соединяют вершины 2, 3 и 4, 5, и в таком графе есть путь, который проходит по всем рёбрам единожды). Таким образом, длина пути, который проходит по всем рёбрам графа $K_6$ равен $10+7=17$.
		\end{solution}
	
	
		\begin{answer}
			17
		\end{answer}
		
		
		
		
		
		
	
\end{document}