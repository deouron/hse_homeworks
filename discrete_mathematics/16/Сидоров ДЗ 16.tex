\documentclass[a4paper, 16pt]{article}

\usepackage[utf8]{inputenc}

\usepackage[russian, english]{babel}
\usepackage{subfiles}
\usepackage[utf8]{inputenc}
\usepackage[T2A]{fontenc}
\usepackage{ucs}
\usepackage{textcomp}
\usepackage{array}
\usepackage{indentfirst}
\usepackage{amsmath}
\usepackage{amssymb}
\usepackage{enumerate}
\usepackage[margin=1.2cm]{geometry}
\usepackage{authblk}
\usepackage{tikz}
\usepackage{icomma}
\usepackage{gensymb}
\usepackage{graphicx}

\DeclareGraphicsExtensions{,.png,.jpg}

\graphicspath{{pictures/}}

\renewcommand{\baselinestretch}{1.4}

\renewcommand{\C}{\mathbb{C}}
\newcommand{\N} {\mathbb{N}}
\newcommand{\Q} {\mathbb{Q}}
\newcommand{\Z} {\mathbb{Z}}
\newcommand{\R} {\mathbb{R}}
\newcommand{\ord} {\mathop{\rm ord}}
\newcommand{\Ima}{\mathop{\rm Im}}
\newcommand{\rk}{\mathop{\rm rk}}

\renewcommand{\r}{\right}
\renewcommand{\l}{\left}
\renewcommand{\inf}{\infty}
\newcommand{\Sum}[2]{\overset{#2}{\underset{#1}{\sum}}}
\newcommand{\Lim}[2]{\lim\limits_{#1 \rightarrow #2}}
\newcommand\tab[1][1cm]{\hspace*{#1}}

\newcommand{\task}[1] {\noindent \textbf{Задача #1.} \hfill}
\newcommand{\note}[1] {\noindent \textbf{Примечание #1.} \hfill}
\newenvironment{proof}[1][Доказательство]{%
	\begin{trivlist}
		\item[\hskip \labelsep {\bfseries #1:}]
		\item \hspace{14pt}
	}{
		$ \hfill\blacksquare $
	\end{trivlist}
	\hfill\break
}
\newenvironment{solution}[1][Решение]{%
	\begin{trivlist}
		\item[\hskip \labelsep {\bfseries #1:}]
		\item \hspace{15pt}
	}{
	\end{trivlist}
}

\newenvironment{answer}[1][Ответ]{%
	\begin{trivlist}
		\item[\hskip \labelsep {\bfseries #1:}] \hskip \labelsep
	}{
	\end{trivlist}
	\hfill
}

\title{Дискретная математика} 
\date{\today}
\author{Сидоров Дмитрий}
\affil{Группа БПМИ 219}


\begin{document}
	\maketitle
	
	\section*{№1}
	
		Случайно и равновероятно выбирается целое число x в промежутке от 1 до 100. Найдите вероятность
		того, что десятичная запись x содержит 8 при условии, что она содержит 5. Ответ привести в виде
		числа (обыкновенная дробь, числитель и знаменатель записаны в десятичной системе).
		
		\begin{solution}
			Вероятностное пространство - числа от 1 до 100, те 100 чисел. Пусть условие $A$ означает, что десятичная запись x содержит 8, а условие $B$, что десятичная запись x содержит 5. Тогда необходимо найти $Pr[A|B]$. Известно, что $Pr[A|B] = \frac{Pr[A \cap B]}{Pr[B]}$. Найдём $Pr[A \cap B]$ и $Pr[B]$. Всего исходов 100 (тк 100 чисел), из них 5 содержится в числах вида $\overline{a5}$, $9 \geq a \geq 0$ (при $a = 0$ число равно 5) и в числах вида $\overline{5a}, 0 \leq a \leq 9$, при этом число 55 посчитается дважды, т е всего 19 чисел, в которых содержится 5. Значит $Pr[B] = \frac{19}{100}$. Для события $A \cap B$, те что в числе содержится и 5, и 8, благоприятных исходов два - это числа 58 и 85, значит $Pr[A \cap B] = \frac{2}{100}$. Таким образом, вероятность
			того, что десятичная запись x содержит 8 при условии, что она содержит 5 равна $Pr[A|B] = \frac{\frac{2}{100}}{\frac{19}{100}} = \frac{2}{19}$. 
		\end{solution}
	
		\begin{answer}
			$\frac{2}{19}$
		\end{answer}
	
	\section*{№2}
	
		Случайно выбирается всюду определённая функция $f$ : $\{1, 2, \dots , n\} \rightarrow \{1, 2, \dots , n\}$. Все исходы равновозможны. Независимы ли события «$f(1) > f(2)$» и «$f(2) > f(3)$»?
		
		\begin{solution}
			Пусть $f(1) > f(2)$ - событие $A$, а $f(2) > f(3)$ - событие $B$. Два события $A, B$ независимы, тогда и только тогда, когда $Pr[A \cap B] = Pr[A] \cdot Pr[B]$. По условию случайно выбирается всюду определённая функция $f$ : $\{1, 2, \dots , n\} \rightarrow \{1, 2, \dots , n\}$ и все исходы равновозможны, значит задачу можно переформулировать так: "Вероятностное пространство - множество всех перестановок чисел от 1 до $n$, независимы ли события "число на первой позиции в перестановке больше числа на второй позиции" и "число на второй позиции в перестановке больше числа на третьей позиции" ?". Таким образом, в новой формулировке события $A$ - это событие "первое число больше второго", а $B$ - это "второе число больше третьего". Найдём $Pr[A], Pr[B], Pr[A \cap B]$ и сравним $Pr[A \cap B]$ и $Pr[A] \cdot Pr[B]$. 
			
			1) $Pr[A], Pr[B]$:
			
			Всего возможных перестановок $n!$. Перестановок, в которых число на первой позиции больше числа на второй позиции $C_n^2 \cdot (n-2)!$ (выбираем 2 числа, ставим из так, что первое больше второго, остальные ставим как угодно). Таким образом, $Pr[A] = \frac{C_n^2 \cdot (n-2)!}{n!} = \frac{(n-2)!}{(n-2)! \cdot 2!} = \frac{1}{2}$. Аналогично $Pr[B] = \frac{1}{2}$.
			
			2) $Pr[A \cap B]$:
			
			$Pr[A \cap B] = Pr[$ "первое число больше второго и второе больше третьего"$] = \frac{C_n^3 \cdot (n-3)!}{n!}$ (выбираем 3 числа, расставляем их так, что первое число больше второго и второе больше третьего, всего $n!$ перестановок). Таким образом, $Pr[A \cap B] = \frac{1}{3!} = \frac{1}{6} \ne \frac{1}{2} \cdot \frac{1}{2} = \frac{1}{4}$. Таким образом, $Pr[A \cap B] \ne Pr[A] \cdot Pr[B]$, а значит события не независимы.
		\end{solution}
	
		\begin{answer}
			нет
		\end{answer}
	
	\section*{№3}
	
		В розыгрыше лото случайно и равновероятно выбираются 5 чисел из множества $\{1, 2 \dots , 36\}$. Найдите вероятность события «среди выбранных чисел нет 20» при условии события «среди выбранных
		чисел нет 21». Ответ привести в виде числа (обыкновенная дробь, числитель и знаменатель записаны
		в десятичной системе).
		
		\begin{solution}
			Вероятностное пространство - количество 5-ти элементных подмножеств множества $\{1, 2 \dots , 36\}$, те всего исходов $C_{36}^5 = \frac{36!}{31! \cdot 5!}$. Пусть событие «среди выбранных чисел нет 20» - это событие $A$, а событие «среди выбранных чисел нет 21» - это событие $B$. Необходимо найти $Pr[A|B] = \frac{Pr[A \cap B]}{Pr[B]}$. Событие $A \cap B$ означает, что среди выбранных 5 чисел нет ни 20, ни 21. Таких исходов $C_{34}^5 = \frac{34!}{5! \cdot 29!}$ (количество пятёрок чисел, среди которых нет 20 и 21). Таким образом, $Pr[A \cap B] = \frac{\frac{34!}{5! \cdot 29!}}{\frac{36!}{31! \cdot 5!}}$. Найдём $Pr[B]$. Количество благоприятных исходов равно $C_{35}^5 = \frac{35!}{30! \cdot 5!}$ (количество пятёрок из всех чисел, кроме 21). Таким образом, $Pr[B] = \frac{\frac{35!}{30! \cdot 5!}}{\frac{36!}{31! \cdot 5!}}$. Таким образом, $Pr[A \cap B] = \frac{\frac{34!}{5! \cdot 29!}}{\frac{35!}{30! \cdot 5!}} = \frac{30}{35} = \frac{6}{7}$.			
		\end{solution}
	
		\begin{answer}
			$\frac{6}{7}$
		\end{answer}
	
	\section*{№4}
	
		В одной коробке лежит 10 фишек, пронумерованных числами от 1 до 10. Во второй коробке лежит
		11 фишек, пронумерованных числами от 1 до 11. Фишки в обоих коробках между собой различаются
		только номерами. Вы выбираете одну из коробок случайно и равновероятно, затем случайно и равновероятно вынимаете фишку из выбранной коробки. Какова вероятность, что выбрана коробка с 11
		фишками при условии, что вынута фишка с номером 7?
		
		\begin{solution}
			Пусть $A$ - это событие "выбрана коробка с 11
			фишками", а $B$ - "вынута фишка с номером 7". Необходимо найти $Pr[A|B] = \frac{Pr[A \cap B]}{Pr[B]}$. Событие $A \cap B$ означает, что выбрана коробка с 11 фишками, и при этом вынута фишка с номером 7. Коробки выбираются равномерно, значит с вероятностью $\frac{1}{2}$ была выбрана коробка с 11 фишками. При этом, тк в этой коробке 11 фишек, вероятность вынуть фишку с номером 7 равна $\frac{1}{11}$. Тогда $Pr[A \cap B] = \frac{1}{2} \cdot \frac{1}{11} = \frac{1}{22}$. Найдём $Pr[B]$. Фишку с номером 7 можно вынятнуть как из первой, так и из второй коробки. Вероятность выбрать каждую из коробок равна $\frac{1}{2}$, при этом вероятность вытянуть фишук с номером 7 из первой коробки равна $\frac{1}{10}$ (тк в ней 10 фишек), а из второй $\frac{1}{11}$ (тк в ней 11 фишек). Таким образом, $Pr[B] = \frac{1}{2} \cdot \frac{1}{10} + \frac{1}{2} \cdot \frac{1}{11} = \frac{1}{20} + \frac{1}{22}$. Значит вероятность, что выбрана коробка с 11
			фишками при условии, что вынута фишка с номером 7 равна $Pr[A | B] = \frac{\frac{1}{22}}{ \frac{1}{20} + \frac{1}{22}} = \frac{1}{\frac{21 \cdot 22}{220}} = \frac{10}{21}$	
		\end{solution}
	
		\begin{answer}
			$\frac{10}{21}$
		\end{answer}
	
	\section*{№5}
	
		О некотором курсе известно, что 10\% задач в домашних заданиях содержат ошибки. Если спросить
		у учебного ассистента, есть ли ошибка в задаче, то правильный ответ будет дан с вероятностью 4/5.
		Если спросить лектора, есть ли ошибка в задаче, правильный ответ будет дан с вероятностью 3/4.
		Студент спросил про некоторую случайно выбранную задачу у учебного ассистента и тот сказал, что
		ошибки нет. Студент уточнил у лектора и тот сказал, что ошибка есть.
		Считая ответы учебного ассистента и лектора независимыми, найдите вероятность того, что при таких
		условиях в задаче есть ошибка.
		
		\begin{solution}
			Обозначим события $A, B, C$ следующим образом: $A$ = "в задаче есть ошибка", $B$ = "ассистент сказал, что в задаче нет ошибки", $C$ =  "лектор сказал, что в задаче есть ошибка". Из условия известно, что $Pr[A] = 0.1$,
			необходимо найти $Pr[A | B \cap C]$. $Pr[A | B \cap C] = \frac{Pr[B \cap C | A]}{Pr[B \cap C]} \cdot Pr[A]$ по формуле Байеса. 
			
			Тк в задаче есть ошибка с вероятностью 0.1, ассистент отвечает правильно с вероятностью $\frac{4}{5}$, а лектор с вероятностью $\frac{1}{5}$, $Pr[B \cap C] = 0.1 \cdot \frac{1}{5} \cdot \frac{3}{4}$ (ошибка есть, ассистент ошибается, лектор отвечает верно)+ $0.9 \cdot \frac{4}{5} \cdot \frac{1}{4}$ (ошибки нет, ассистен отвечает верно, лектор ошибается) = $\frac{3}{200} + \frac{36}{200} = \frac{39}{200}$. 
			
			При этом $Pr[B \cap C | A] = \frac{1}{5} \cdot \frac{3}{4} = \frac{3}{20}$ (в задаче есть ошибка, при этом ассистен ошибся, лектор ответил верно). Таким образом, $Pr[A | B \cap C] =  \frac{\frac{3}{20}}{\frac{39}{200}} \cdot \frac{1}{10} = \frac{1}{13}$ - вероятность того, что в задаче есть ошибка при таких ответах.
			
			%$Pr[B|A] = 1 -\frac{4}{5} = \frac{1}{5}, Pr[C|A] = \frac{3}{4}$. Необходимо найти $Pr[A|B]$ и $Pr[A|C]$. Для этого найдём $Pr[B]$ и $Pr[C]$. По формуле полной вероятности $Pr[B] = Pr[A] \cdot Pr[B|$"ошибка есть"$] + Pr[\overline{A}] \cdot Pr[B|$"ошибки нет"$] = 0.1 \cdot \frac{1}{5} + 0.9 \cdot \frac{4}{5} = 0.74$. Тогда по формуле Байеса $Pr[A|B] = Pr[A] \cdot \frac{Pr[B|A]}{Pr[B]} = 0.1 \cdot \frac{0.2}{0.74} = \frac{1}{37}$. Аналогично для $Pr[A|C]$: $Pr[C] = Pr[A] \cdot Pr[C|$"ошибка есть"$] + Pr[\overline{A}] \cdot Pr[C|$"ошибки нет"$] = 0.1 \cdot \frac{3}{4} + 0.9 \cdot \frac{1}{4} = 0.3$ $\Rightarrow$ $Pr[A|C] = 0.1 \cdot \frac{0.75}{0.3} = \frac{1}{4}$.
 		\end{solution}
 	
 		\begin{answer}
 			$\frac{1}{13}$
 		\end{answer}
 	
	\section*{№6}
	
		Про события A, B, C известно, что A и B независимы, B и C независимы, C и A независимы.
		Следует ли из этого, что события A и B $\cap$ C независимы?
		
		\begin{solution}
			Тк A и B независимы, B и C независимы, C и A независимы, то $Pr[A \cap B] = Pr[A] \cdot Pr[B], Pr[B \cap C] = Pr[B] \cdot Pr[C], Pr[A \cap C] = Pr[A] \cdot Pr[C]$. Если события A и B $\cap$ C независимы, то должно выполняться $Pr[A \cap B \cap C] = Pr[A] \cdot Pr[B \cap C]$.
			
			Пусть есть множество $M = \{0, 1\}$, событие $B$ - это событие "число $b \in M$ чётное", $C$ - событие "число $c \in M$ чётное", а $A$ - это событие "число $a = b +c$ чётное" ($b, c$ выбираются независимо, с вероятностью $\frac{1}{2}$). Тогда возможно 4 случая: $0 +0=0,\ 1 + 0 = 1,\ 0 + 1 = 1,\ 1 + 1 = 2$. При этом $Pr[A] = \frac{2}{4} = \frac{1}{2}$,  $Pr[A \cap B] = Pr[A \cap C] = \frac{1}{4}$ (возможен только вариант $0 + 0 = 0$), $Pr[B] = Pr[C] = \frac{1}{2}$ (условие). Очевидно, что $B$ и $C$ независимы, покажем, что   A и B и C и A тоже независимы. $Pr[A \cap B] = Pr[A] \cdot Pr[B] = \frac{1}{2} \cdot \frac{1}{2} = \frac{1}{4}, Pr[A \cap C] = Pr[A] \cdot Pr[C] = \frac{1}{2} \cdot \frac{1}{2} =  \frac{1}{4}$. Таким образом, условие выполняется. При этом $Pr[A \cap B \cap C]  = \frac{1}{4}$ (благоприятный исход только при $a = b = c = 0$), а $Pr[A] \cdot Pr[B \cap C] = \frac{1}{2} \cdot \frac{1}{4} = \frac{1}{8}$. Таким образом, $Pr[A \cap B \cap C] \ne Pr[A] \cdot Pr[B \cap C]$, а значит события A и B $\cap$ C зависимы.
		\end{solution}
	
		\begin{answer}
			нет
		\end{answer}
	
	\section*{№7}
	
		Докажите, что существуют такое вероятностное пространство, вероятностное распределение на нём
		и три события A, B, C, что
		$Pr[A \cap B \cap C] = Pr[A] \cdot Pr[B] \cdot Pr[C]$,
		но никакая пара событий не является независимой.
		
		\begin{proof}
			Существуют такое вероятностное пространство, вероятностное распределение на нём
			и три события A, B, C, что
			$Pr[A \cap B \cap C] = Pr[A] \cdot Pr[B] \cdot Pr[C]$,
			но никакая пара событий не является независимой. Например: пусть игральный кубик подбрасывают дважды, те вероятностное пространство - пары чисел от 1 до 6, все исходы равновероятные. Пусть событие $A$ - "первым броском выпало число 1, 2 или 3", $B$ - "первым броском выпало число 3, 4 или 5", $C$ - "сумма чисел за 2 броска равна 9". Тогда, тк всего возможных исходов 36, а благоприятных исходов для $A$ и для $B$ - $3 \cdot 6 = 18$, а для $C$ - 4 ((3, 6), (4, 5), (5, 4), (6, 3)), $Pr[A] Pr[B]= \frac{18}{36} = \frac{1}{2}, \ Pr[C] = \frac{4}{36} = \frac{1}{9}$, и выполняется $Pr[A \cap B \cap C] = Pr[A] \cdot Pr[B] \cdot Pr[C] = \frac{1}{36}$ (благоприятный исход для $A \cap B \cap C$ только (3, 6)), но при этом $Pr[A \cap B] = \frac{6}{36} = \frac{1}{6}$ (благоприятные исходы вида $(3, x), \ 1 \leq x \leq 6, x \in \N ) \ne Pr[A] \cdot Pr[B] = \frac{1}{4}$, $Pr[A \cap C] = \frac{1}{36}$ (благоприятный исход (3, 6)) $\ne Pr[A] \cdot Pr[C] = \frac{1}{18}$, $Pr[B \cap C] = \frac{3}{36} = \frac{1}{12}$ (благоприятные исходы (3, 6), (4, 5), (5, 4)) $\ne Pr[B] \cdot Pr[C] = \frac{1}{18}$, а значит никакая пара событий не является независимой.
		\end{proof}
	
	
	
	
	
	
	
	
	
	
	
	
	
\end{document}