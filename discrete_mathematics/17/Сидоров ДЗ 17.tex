\documentclass[a4paper, 16pt]{article}

\usepackage[utf8]{inputenc}

\usepackage[russian, english]{babel}
\usepackage{subfiles}
\usepackage[utf8]{inputenc}
\usepackage[T2A]{fontenc}
\usepackage{ucs}
\usepackage{textcomp}
\usepackage{array}
\usepackage{indentfirst}
\usepackage{amsmath}
\usepackage{amssymb}
\usepackage{enumerate}
\usepackage[margin=1.2cm]{geometry}
\usepackage{authblk}
\usepackage{tikz}
\usepackage{icomma}
\usepackage{gensymb}
\usepackage{graphicx}

\DeclareGraphicsExtensions{,.png,.jpg}

\graphicspath{{pictures/}}

\renewcommand{\baselinestretch}{1.4}

\renewcommand{\C}{\mathbb{C}}
\newcommand{\N} {\mathbb{N}}
\newcommand{\Q} {\mathbb{Q}}
\newcommand{\Z} {\mathbb{Z}}
\newcommand{\R} {\mathbb{R}}
\newcommand{\ord} {\mathop{\rm ord}}
\newcommand{\Ima}{\mathop{\rm Im}}
\newcommand{\rk}{\mathop{\rm rk}}

\renewcommand{\r}{\right}
\renewcommand{\l}{\left}
\renewcommand{\inf}{\infty}
\newcommand{\Sum}[2]{\overset{#2}{\underset{#1}{\sum}}}
\newcommand{\Lim}[2]{\lim\limits_{#1 \rightarrow #2}}
\newcommand\tab[1][1cm]{\hspace*{#1}}

\newcommand{\task}[1] {\noindent \textbf{Задача #1.} \hfill}
\newcommand{\note}[1] {\noindent \textbf{Примечание #1.} \hfill}
\newenvironment{proof}[1][Доказательство]{%
	\begin{trivlist}
		\item[\hskip \labelsep {\bfseries #1:}]
		\item \hspace{14pt}
	}{
		$ \hfill\blacksquare $
	\end{trivlist}
	\hfill\break
}
\newenvironment{solution}[1][Решение]{%
	\begin{trivlist}
		\item[\hskip \labelsep {\bfseries #1:}]
		\item \hspace{15pt}
	}{
	\end{trivlist}
}

\newenvironment{answer}[1][Ответ]{%
	\begin{trivlist}
		\item[\hskip \labelsep {\bfseries #1:}] \hskip \labelsep
	}{
	\end{trivlist}
	\hfill
}

\title{Дискретная математика} 
\date{\today}
\author{Сидоров Дмитрий}
\affil{Группа БПМИ 219}


\begin{document}
	\maketitle
	
	\section*{№1}
	
		В лотерее на выигрыши уходит 25\% от стоимости проданных билетов. Каждый билет стоит 40 рублей. Докажите, что вероятность выиграть не менее 1000 рублей не больше 1\%.
		
		\begin{proof}
			По неравенству Маркова 
			$Pr[f \geq \alpha] \leq \frac{E[f]}{\alpha}$. 
			Если в лотерее на выигрыши уходит 25\% от стоимости проданных билетов, и каждый билет стоит 40 рублей, то, если в лотерее учатвует $n$ человек, они потратят $40n$ рублей, и общий выигрыш составит $40n \cdot 0.25 = 10n$ рублей, а значит математическое ожидание выигрыша равно $E[f] = \frac{10n}{n} = 10$. Таким образом, при $\alpha = 1000$ $Pr[f \geq 1000] \leq \frac{10}{1000} = \frac{1}{100} \Rightarrow$ вероятность выиграть не менее 1000 рублей не больше 1\%.
		\end{proof}
	
	\section*{№2}
	
		Магазин назначил за каждый товар целую цену, а при покупке добавляет равновероятно к цене
		товару случайное число копеек (от 0 до 99). Покупатель взял 20 разных товаров и направился к кассе.
		Найдите математическое ожидание доплаты покупателя — разности между итоговой суммой к оплате
		и суммой к оплате без добавленных копеек
		
		\begin{solution}
			Вероятностное пространство состоит из чисел от 0 до 99. Обозначим переплату за $i$-ый товар как $X_i$, тогда $E[X_i] = \sum\limits_{j=0}^{99} j \cdot Pr[j] = \frac{1}{100} \cdot (0 + 1 + \dots + 99) = \frac{1}{100} \cdot \frac{99 \cdot 100}{2} = 49,5$ (равновероятно добавляют от 0 до 99 копеек, всего 100 значений). Значит математическое ожидание доплаты покупателя за каждый товар равно 49,5 копеек, а значит за 20 товаров математическое ожидание доплаты покупателя равно $49,5 \cdot 20 = 990$ копеек.
		\end{solution}
	
		\begin{answer}
			990
		\end{answer}
	
	\section*{№3}
	
		Вероятностное пространство — перестановки $(x_1, \dots , x_n)$ элементов от 1 до n. Найдите математическое ожидание количества чисел, не поменявших своё место. Формально, случайная величина —
		количество элементов в множестве $\{i | x_i = i\}$.
		
		\begin{solution}
			Пусть для каждого элемента от 1 до $n$ случайная величина $f_i$ равна 1, если $x_i = i$, и $f_i = 0$ иначе. Тогда $\sum\limits_{i = 1}^n f_i = S$, где $S$ -  количество чисел, не поменявших своё место. Тогда $E[f_i] = 1 \cdot Pr[A_i] + 0 \cdot Pr[\overline{A_i}] = Pr[A_i]$, где $A_i$ - событие "$x_i = i$". Тогда $E[S] = E[f_1] + \dots + E[f_n] = Pr[A_1] + \dots + Pr[A_n] = n \cdot Pr[A_1]$ (тк $Pr[A_1] = Pr[A_2] = \dots = Pr[A_n])$. Найдём $Pr[A_1]$. Исходы - перестановки, а значит всего исходов $n!$. Благоприятных (те в которых $x_1 = 1$) $(n-1)!$ (фиксируем $x_1 = 1$, остальные $x$ - любые). Значит $Pr[A_1] = \frac{(n-1)!}{n!} = \frac{1}{n}$. Таким образом, математическое ожидание количества чисел, не поменявших своё место, равно $E[S] = n \cdot \frac{1}{n} = 1$.  
		\end{solution}
	
		\begin{answer}
			1
		\end{answer}
	
	\section*{№4}
		
		Вероятностное пространство: пары $(X, Y )$ подмножеств $n$ - элементного множества $\{1, 2, \dots , n\}$. Все
		исходы равновозможны. Найдите математическое ожидание $|X \cup Y |$.
		
		\begin{solution}
			%Пусть $z = |X| + |Y| - |X\cap Y|$. Пусть $|X \cup Y |= S$, тогда $S = S_1 + S_2 + \dots + S_n$, где $S_i = 1$, если $i \in X \cup Y$ и $S_i = 0$ иначе. Тогда для случайной величины $S_i$ $E[S_i] = 1 \cdot Pr[A_i] + 0 \cdot Pr[\overline{A_i}] = Pr[A_i]$, где $A_i$ - событие "$i \in  X \cup Y$". Заметим, что $Pr[A_i] = Pr["i \in X \cup Y"] = \frac{z}{n}$. Таким образом, $E[ \ |X \cup Y | \ ] = E[S] = \sum\limits_{i=1}^n E[S_i] = n \cdot \frac{z}{n} = z = |X| + |Y| - |X\cap Y|$.
			%Необходимо найти $E[|X\cup Y|] = E[|X| + |Y| - |X \cap Y|] = E[|X|] + E[|Y|] - E[|X \cap Y|]$. Найдём $E[|X|]$ и $E[|Y|]$. В каждом подмножестве  может быть равновероятно от 0 до $n$ элементов, те вероятность того, что подмножество состоит из $i$ элементов равна $\frac{1}{n}$. Тогда $E[|X|] = E[|Y|] = \sum \limits_{i = 1}^n i \cdot \frac{1}{n} = \frac{(1 + n)n}{2n} = \frac{1 + n}{2}$, значит $E[|X|] + E[|Y|] = 1 + n$. 
			Найдём вероятность для каждого элемента множества $\{1, 2, \dots , n\}$, что он принадлежит $|X \cup Y |$. Произвольный элемент не принадлежит подмножеству $X$ с вероятностью $\frac{1}{2}$ и так же не принадлежит подмножеству $Y$ с вероятностью $\frac{1}{2} \Rightarrow$ вероятность того, что произвольный элемент не принадлежит $|X \cup Y |$ равна $\frac{1}{4} \Rightarrow$ вероятность события "элемент $i \in |X \cup Y |$" равна $1 - \frac{1}{4} = \frac{3}{4}$ $\forall \ 0 \leq i \leq n, i \in \Z$. 
			
			Пусть $f_i = 1$, если $i \in |X \cup Y |$, и $f_i = 0$ иначе. Тогда математическое ожидание $|X \cup Y |$ равно $E[X] = \sum\limits_{i=1}^n f_i = n \cdot \frac{3}{4}$ (тк $E[f_i] = 1 \cdot Pr[A_i] + 0 \cdot Pr[\overline{A_i}] = Pr[A_i]$, где $A_i$ - событие "элемент $i \in |X \cup Y |$").
		\end{solution}
	
		\begin{answer}
			%$|X| + |Y| - |X\cap Y|$
			$\frac{3n}{4}$
		\end{answer}
	
	\section*{№5}
	
		Про неотрицательную случайную величину $X$ известно, что $Pr[X < 3] = 1/3$ и $Pr[X \geq 6] = 1/6$.
		Найдите все возможные значения математического ожидания $E[X]$.
		
		\begin{solution}
			$X$ - неотрицательная величина $(X \in [0; +\inf)) \Rightarrow Pr[X \in [0; 3)] = 1/3, Pr[X \in [6; +\inf)] = 1/6 \Rightarrow Pr[X \in [3; 6)] = 1 - \frac{1}{3} - \frac{1}{6} = \frac{1}{2}$. Заметим, что если $X \in [0; 3) \ X \geq 0$,  если $X \in [3; 6) \ X \geq 3$, а  если $X \in [6; +\inf) \ X \geq 6$. Таким образом $E[X] \geq 0 \cdot \frac{1}{3} + 3 \cdot \frac{1}{2} + 6\cdot \frac{1}{6} = 2.5$. При этом, пусть максимальное значение, которое принимает $X$ равно $x$. Тогда $E[X] = \frac{0 + 3}{2} \cdot \frac{1}{3} + \frac{3 + 6}{2} \cdot \frac{1}{2} + \frac{6 + x}{2} \cdot \frac{1}{6} = 0.5 + 2.25 + 0.5 + \frac{x}{12} = 3.25 + \frac{x}{12}$. Тк $x$ не ограничено сверху, то и $E[X]$ не ограничено сверху. Таким образом, $E[X] \geq 2.5$. 
		\end{solution}
	
		\begin{answer}
			$E[X] \geq 2.5$
		\end{answer}
	
	\section*{№6}
	
		Игральная кость бросается три раза. M — максимальное количество очков, выпавшее в этих бросках.
		Найдите E[M].
		
		\begin{solution}
			Пусть $A_i$ - событие "i - максимальное количество очков, выпавшее за 3 броска". Тогда $E[M] = \sum\limits_{i = 1}^6 i \cdot Pr[A_i]$. Найдём $Pr[A_1], Pr[A_2], \dots, Pr[A_6]$. Вероятностное пространство - тройки чисел от 1 до 6, всего $6^3$ исходов. Если максимальное количество очков, выпавшее за 3 броска, равно 1, то в каждом броске выпало 1 очко, значит $Pr[A_1] = \frac{1}{6^3}$. Если максимальное количество очков, выпавшее за 3 броска, равно 2, то благоприятными исходами являются все тройки чисел вида $(x_1, x_2, x_3), 1 \leq x \leq 2$, кроме (1, 1, 1), значит всего благоприятных исходов $2^3 - 1 = 7, Pr[A_2] = \frac{7}{6^3}$. Аналогично для количества очков равного 3, благоприятными исходами являются все тройки чисел вида $(x_1, x_2, x_3), 1 \leq x \leq 3$, кроме тех, которые являются благоприятными исходами для $A_1$ или $A_2$, те их $3^3 - 1 - 7 = 27 - 8 = 19$, $Pr[A_3] = \frac{19}{6^3}$. Аналогично получаем, что $Pr[A_4] = \frac{4^3 - 1 - 7 - 19}{6^3} = \frac{37}{6^3}, 
			Pr[A_5] = \frac{5^3 - 1 - 7 - 19 - 37}{6^3} = \frac{61}{6^3}, 
			Pr[A_4] = \frac{6^3 - 1 - 7 - 19 - 37 - 61}{6^3} = \frac{91}{6^3}$. Таким образом, $E[M] = \sum\limits_{i = 1}^6 i \cdot Pr[A_i] = 1 \cdot \frac{1}{6^3} + 2 \cdot \frac{7}{6^3} + 3 \cdot \frac{19}{6^3} + 4 \cdot \frac{37}{6^3} + 5 \cdot \frac{61}{6^3} + 6 \cdot \frac{91}{6^3} = \frac{1071}{216} = \frac{119}{24}$.
		\end{solution}
	
		\begin{answer}
			$\frac{119}{24}$
		\end{answer}
	
	\section*{№7}
	
		Каждое из чисел $a_1, \dots , a_n$ выбирается случайно, равномерно и независимо среди чисел $1, 2, . . . , n$.
		Найдите математическое ожидание количества различных чисел среди $a_1, \dots , a_n$.
		
		\begin{solution}
			Пусть случайная величина $X$ - количество различных чисел среди $a_1, \dots , a_n$, те необходимо найти $E[X]$. 
			По условию "каждое из чисел $a_1, \dots , a_n$ выбирается случайно, равномерно и независимо",  а значит вероятность выбрать каждое число за одно действие равна $\frac{1}{n}$. Таким образом, вероятность не выбрать число $k$ за одно действие равна $1 - \frac{1}{n}$. Всего выбирают числа $n$ раз, число $k$ не будет выбрано, если за каждое действие выбрали не $k$, те с вероятностью $(1 - \frac{1}{n})^n$, и значит число $k$ будет выбрано с вероятностью $1 - (1 - \frac{1}{n})^n$. Таким образом,  вероятность того, что число $k$ будет выбрано, равна  $1 - (1 - \frac{1}{n})^n$.
			
			 Пусть $I_k = 1$, если число $k$ выбрано, и $I_k = 0$ иначе. Тогда математическое ожидание количества различных чисел среди $a_1, \dots , a_n$ равно $E[X] = \sum\limits_{k=1}^n I_k =1 - (1 - \frac{1}{n})^n + \dots + 1 - (1 - \frac{1}{n})^n = n(1 - (1 - \frac{1}{n})^n)$ (тк $E[I_k] = 1 \cdot Pr[A_k] + 0 \cdot Pr[\overline{A_k}] = Pr[A_k]$, где $A_k$ - cобытие "$k$ выбрано").
		\end{solution}
	
		\begin{answer}
			$n(1 - (1 - \frac{1}{n})^n)$
		\end{answer}
	
	
	
	
	
	
	
	
	
	
	
\end{document}