\documentclass[a4paper, 16pt]{article}

\usepackage[utf8]{inputenc}

\usepackage[russian, english]{babel}
\usepackage{subfiles}
\usepackage[utf8]{inputenc}
\usepackage[T2A]{fontenc}
\usepackage{ucs}
\usepackage{textcomp}
\usepackage{array}
\usepackage{indentfirst}
\usepackage{amsmath}
\usepackage{amssymb}
\usepackage{enumerate}
\usepackage[margin=1.2cm]{geometry}
\usepackage{authblk}
\usepackage{tikz}
\usepackage{icomma}
\usepackage{gensymb}
\usepackage{graphicx}

\DeclareGraphicsExtensions{,.png,.jpg}

\graphicspath{{pictures/}}

\renewcommand{\baselinestretch}{1.4}

\renewcommand{\C}{\mathbb{C}}
\newcommand{\N} {\mathbb{N}}
\newcommand{\Q} {\mathbb{Q}}
\newcommand{\Z} {\mathbb{Z}}
\newcommand{\R} {\mathbb{R}}
\newcommand{\ord} {\mathop{\rm ord}}
\newcommand{\Ima}{\mathop{\rm Im}}
\newcommand{\rk}{\mathop{\rm rk}}

\renewcommand{\r}{\right}
\renewcommand{\l}{\left}
\renewcommand{\inf}{\infty}
\newcommand{\Sum}[2]{\overset{#2}{\underset{#1}{\sum}}}
\newcommand{\Lim}[2]{\lim\limits_{#1 \rightarrow #2}}
\newcommand\tab[1][1cm]{\hspace*{#1}}

\newcommand{\task}[1] {\noindent \textbf{Задача #1.} \hfill}
\newcommand{\note}[1] {\noindent \textbf{Примечание #1.} \hfill}
\newenvironment{proof}[1][Доказательство]{%
	\begin{trivlist}
		\item[\hskip \labelsep {\bfseries #1:}]
		\item \hspace{14pt}
	}{
		$ \hfill\blacksquare $
	\end{trivlist}
	\hfill\break
}
\newenvironment{solution}[1][Решение]{%
	\begin{trivlist}
		\item[\hskip \labelsep {\bfseries #1:}]
		\item \hspace{15pt}
	}{
	\end{trivlist}
}

\newenvironment{answer}[1][Ответ]{%
	\begin{trivlist}
		\item[\hskip \labelsep {\bfseries #1:}] \hskip \labelsep
	}{
	\end{trivlist}
	\hfill
}

\title{Дискретная математика} 
\date{\today}
\author{Сидоров Дмитрий}
\affil{Группа БПМИ 219}


\begin{document}
	\maketitle
	
	\section*{№1}
	
		Сколькими способами образовать 5 пар из 10 человек? (Порядок пар и порядок в паре несуществен-
		ны.)
		
		\begin{solution}
			Первую пару можно составить $C_{10}^2$ способами, вторую $C_{8}^2$, третью $C_{6}^2$, четвёртую $C_{4}^2$, пятую $C_{2}^2$. Получаем $C_{10}^2$ $\cdot$ $C_{8}^2$ $\cdot$ $C_{6}^2$ $\cdot$ $C_{4}^2$ $\cdot$ $C_{2}^2$ способов. При этом, если порядок пар и порядок в паре несуществены, то не влияет порядок пар, а значит, тк пары можно переставлять $5!$ способами, способов будет $\frac{C_{10}^2 \cdot C_{8}^2 \cdot C_{6}^2 \cdot C_{4}^2 \cdot C_{2}^2}{5!} = \frac{10! 8! 6! 4! 2!}{2! 8! 2! 6! 4! 2! 2! 2! 2! 5!} = \frac{10!}{5! 2^5} = \frac{6 \cdot 7 \cdot 8 \cdot 9 \cdot 10}{32} = 945$.
		\end{solution}
	
		\begin{answer}
			945
		\end{answer}
	
	\section*{№2}
		Сколькими способами можно расставить в ряд 12 различных чисел так, чтобы наибольшее и наименьшее из них стояли рядом?
		
		\begin{solution}
			Пусть $x, y$ - наибольшее и наименьшее числа соотв. Будем считать, что числа стоят рядом, а значит, можно считать, что нужно разместить 11 элементов, 10 из которых - числа, отличные от $x, y$, а 11-ый - пара из $x, y$. При этом нам не важно в каком порядке идут $x, y$, те пара может выглядеть как $(x, y)$, а может как $(y, x)$, те количество способов расставить 11 элементов необходимо умножить на 2. Таким образом, способов расставить в ряд 12 различных чисел так, чтобы наибольшее и наименьшее из них стояли рядом $11! \cdot 2$.
		\end{solution}
	
		\begin{answer}
			$11! \cdot 2$
		\end{answer}
	
	\section*{№3}
	
		Сколько есть частичных порядков на
		n-элементном множестве, в которых ровно одна пара элементов
		несравнима?
		
		\begin{solution}
			Рассмотрим порядок на $n$ элементах. Пусть в нём все элементы, кроме $x, y$ сравнимы. Тогда $x$ сравним со всеми остальными элементами, и $y$ сравним со всеми элементами, при этом, если существует такой элемент $z$, с которым сравним и $x$, и $y$, что $z < x$, а $y < z$, то $y < z < x$, те $x, y$ сравнимы, что противоречит условию (аналогично если $z > x$, а $y > z$, то $y > z > x$). Таким образом, $x$ больше всех элементов, которые меньше $y$, и меньше всех элементов, которые больше $y$ (аналогично для $y$). Теперь посчитаем количесвто искомых порядков. На порядке из $n$ элементов есть $\frac{n(n-1)}{2}$ способа выбрать пару элементов, далее остаётся $n-2$ элемента, порядок на них можно задать $(n-2)!$ способами, а потом, тк $x$ больше всех элементов, которые меньше $y$, и меньше всех элементов, которые больше $y$ (аналогично для $y$), есть $n - 1$ способ определить, какие элементы больше $x, y$, а какие меньше. Таким образом, всего таких порядков $\frac{n(n - 1)}{2} \cdot (n-2)! \cdot (n-1) = \frac{n!(n-1)}{2}$.
		\end{solution}
	
		\begin{answer}
			$\frac{n!(n-1)}{2}$
		\end{answer}
	
	\section*{№4}
	
		Сколько существует 6-значных чисел, в которых цифры идут в возрастающем порядке?
		
		\begin{solution}
			Заметим, что шестизначное число не может начинаться на 0, цифры возрастают, а значит в таких числах нет цифры 0. Заметим, что искомые числа получаются вычёркиванием из числа 123456789 3 цифр, те число 6-значных чисел, в которых цифры идут в возрастающем порядке равно количеству способов вычеркнуть из числа 123456789 3 цифры, а это $C_9^3 = \frac{9!}{3!6!} = \frac{7 \cdot 8 \cdot 9 }{6} = 84$ способа.
		\end{solution}
	
		\begin{answer}
			84
		\end{answer}
	
	\section*{№5}
	
		Сколькими способами можно выписать в ряд цифры от 0 до 9 так, чтобы чётные цифры шли в
		порядке возрастания, а нечётные — в порядке убывания?
		
		\begin{solution}
			Пронумеруем позиции в ряду от 1 до 10. Заметим, что если выделить 5 позиций для чётных цифр, то существует только один вариант распределения чётных цифр среди этих позиций (тк чётные цифры не возрастают, те их порядок однозначно определён, аналогично для нечётных). Таким образом, задача сводится к нахождению способов выбрать 5 позиций из 10 для чётных чисел (для нечётных однозначно определются оставшиеся 5 позиций), а это $C_{10}^5 = \frac{10!}{5!5!} = \frac{6 \cdot 7 \cdot 8 \cdot 9 \cdot 10}{2\cdot 3\cdot 4\cdot 5} = 252$ способа.
		\end{solution}
	
		\begin{answer}
			252
		\end{answer}
		
	\section*{№6}
	
		Нужно выбрать множество, состоящее их двух диагоналей правильного
		n-угольника (n$\geq$5), которые
		не пересекаются во внутренних точках. Сколькими способами это можно сделать?
		
		\begin{solution}
			 Заметим, что если зафиксировать 2 вершины, то между ними можно провести только 1 диагональ, если зафиксировать 3 вершины, то между ними можно провести 2 диагонали, но они не будут пересекаться, таким образом, пара диагоналей будет пересекаться только в том случае, если зафиксировано 4 вершины (больше можно не фиксировать, тк для существования пары диагоналей достаточно 3 или 4 вершины). Заметим, что в каждой такой четверке вершин пересекающиеся пары определяются однозначно (если представить 4 вершины как четырехугольник, то диагонали пересекаются только в том случае, если соединяют противоположные вершины). Значит количество способов выбрать пересекающиеся диагонали в  $n$-угольнике равно количеству четверок вершин, те $C_n^4 = \frac{n!}{4!(n-4)!}$.
			 
			 Зафиксируем произвольную вершину $n$-угольник. Из неё можно провести диагональ в $n - 3$ вершины $n$-угольника (нельзя в саму себе и в соседние), а значит в $n$-угольнике всего $\frac{n(n-3)}{2}$ диагоналей. Тогда способов выбрать 2 диагонали $\frac{n(n-3)}{2} (\frac{n(n-3)}{2} - 1) \frac{1}{2} = \frac{n^2(n-3)^2-2n(n-3)}{4} \cdot \frac{1}{2} = \frac{n(n-3)(n^2-3n-2)}{8}$.
			 			
			 Тогда количество способов выбрать не пересекающиеся диагонали равно $\frac{n(n-3)(n^2-3n-2)}{8} - \frac{n!}{4!(n-4)!} = \frac{n(n-3)(n^2-3n-2)}{8} - \frac{(n-3)(n-2)(n-1)n}{24} = \frac{3n(n-3)(n^2-3n-2)}{24} - \frac{(n-3)(n-2)(n-1)n}{24} = \frac{n(n-3)(3n^2-9n-6-n^2+3n-2)}{24} = \frac{n(n-3)(2n^2-6n-8)}{24} = \frac{n(n-3)(n^2-3n-4)}{12}$
		\end{solution}
	
		\begin{answer}
			 $\frac{n(n-3)(n^2-3n-4)}{12}$
		\end{answer}
	
	\section*{№7}
	
		Робот ходит по координатной плоскости. На каждом шаге он может увеличить одну координату на
		1 или обе координаты на 2. Сколько есть способов переместить Робота из точки
		(0
		,
		0)
		в точку
		(4
		,
		5)
		?
		
		\begin{answer}
			Обозначим $f(a, b)$ - количество способов добраться в клетку $(a, b)$. Заметим, что способов добраться в клетку с координатами $(x, y)$ $f(x, y) = f(x - 1, y) + f(x, y - 1) + f(x-2, y-2)$. При этом $f(1, z) = f(z, 1) = 1$ (тк добраться в самую левую/нижнюю клетку можно только одним способом). Зная это определим количество способов добраться до точки (4,5).
			
			\includegraphics{№7}
			
			Таким образом, всего 189 способа добраться из точки
			(0
			,
			0)
			в точку
			(4
			,
			5).
		\end{answer}
	
		\begin{answer}
			189
		\end{answer}
	
	\section*{№8}
	
		Докажите, что $\sum\limits_{j=0}^k \binom{r}{j} \binom{s}{k - j} = \binom{r + s}{k}$
	
		\begin{proof}
			По формуле бинома $(x + y) ^ n = \sum\limits_{k = 0}^n \binom{n}{k} x^{n-k}y^k$. При $y = 1$: $(x + 1)^n = (1 + x) ^ n = \sum\limits_{k = 0}^n \binom{n}{k} x^{k}$. Рассмотрим равенство $(x + 1)^{r+s} = (x+1)^r \cdot (x + 1)^s$. Тогда $(x + 1)^{r+s}  =  \sum\limits_{k = 0}^{r+s}\binom{r+s}{k} x^{k}$.
			
			Рассмотрим $(x+1)^r \cdot (x + 1)^s$: $(x+1)^r \cdot (x + 1)^s = \sum\limits_{i = 0}^r\binom{r}{i} x^{i} \cdot \sum\limits_{i = 0}^s\binom{s}{i} x^{i}$. Заметим, что два многочлена дают в произведении многочлен в степени $k$, если степень первого равна $j$, а степень второго $k - j$. Из произведения выберем все $x^k$. Тогда $\sum\limits_{i = 0}^r\binom{r}{i} x^{i} \cdot \sum\limits_{i = 0}^s\binom{s}{i} x^{i} = \sum\limits_{k = 0}^{r+s}x^j\cdot x^{k-j} \left( \sum\limits_{j=0}^k  \binom{r}{j} \binom{s}{k - j}\right) = \sum\limits_{k = 0}^{r+s}x^k \left( \sum\limits_{j=0}^k  \binom{r}{j} \binom{s}{k - j}\right) $. Тогда из равенства $(x + 1)^{r+s} = (x+1)^r \cdot (x + 1)^s$ следует, что $\sum\limits_{k = 0}^{r+s}\binom{r+s}{k} x^{k} = \sum\limits_{k = 0}^{r+s}x^k \left( \sum\limits_{j=0}^k  \binom{r}{j} \binom{s}{k - j}\right)$, а значит $\sum\limits_{j=0}^k \binom{r}{j} \binom{s}{k - j} = \binom{r + s}{k}$ (тк получим одинаковые многочлены, а значит их коэф равны).
		\end{proof}
		
	
	
	
	
	
	
	
	
	
	
	
	
	
	
	
\end{document}