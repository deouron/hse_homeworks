\documentclass[a4paper, 16pt]{article}

\usepackage[utf8]{inputenc}

\usepackage[russian, english]{babel}
\usepackage{subfiles}
\usepackage[utf8]{inputenc}
\usepackage[T2A]{fontenc}
\usepackage{ucs}
\usepackage{textcomp}
\usepackage{array}
\usepackage{indentfirst}
\usepackage{amsmath}
\usepackage{amssymb}
\usepackage{enumerate}
\usepackage[margin=1.2cm]{geometry}
\usepackage{authblk}
\usepackage{tikz}
\usepackage{icomma}
\usepackage{gensymb}
\usepackage{graphicx}

\DeclareGraphicsExtensions{,.png,.jpg}

\graphicspath{{pictures/}}

\renewcommand{\baselinestretch}{1.4}

\renewcommand{\C}{\mathbb{C}}
\newcommand{\N} {\mathbb{N}}
\newcommand{\Q} {\mathbb{Q}}
\newcommand{\Z} {\mathbb{Z}}
\newcommand{\R} {\mathbb{R}}
\newcommand{\ord} {\mathop{\rm ord}}
\newcommand{\Ima}{\mathop{\rm Im}}
\newcommand{\rk}{\mathop{\rm rk}}

\renewcommand{\r}{\right}
\renewcommand{\l}{\left}
\renewcommand{\inf}{\infty}
\newcommand{\Sum}[2]{\overset{#2}{\underset{#1}{\sum}}}
\newcommand{\Lim}[2]{\lim\limits_{#1 \rightarrow #2}}
\newcommand\tab[1][1cm]{\hspace*{#1}}

\newcommand{\task}[1] {\noindent \textbf{Задача #1.} \hfill}
\newcommand{\note}[1] {\noindent \textbf{Примечание #1.} \hfill}
\newenvironment{proof}[1][Доказательство]{%
	\begin{trivlist}
		\item[\hskip \labelsep {\bfseries #1:}]
		\item \hspace{14pt}
	}{
		$ \hfill\blacksquare $
	\end{trivlist}
	\hfill\break
}
\newenvironment{solution}[1][Решение]{%
	\begin{trivlist}
		\item[\hskip \labelsep {\bfseries #1:}]
		\item \hspace{15pt}
	}{
	\end{trivlist}
}

\newenvironment{answer}[1][Ответ]{%
	\begin{trivlist}
		\item[\hskip \labelsep {\bfseries #1:}] \hskip \labelsep
	}{
	\end{trivlist}
	\hfill
}

\title{Дискретная математика} 
\date{\today}
\author{Сидоров Дмитрий}
\affil{Группа БПМИ 219}


\begin{document}
	\maketitle
	
	\section*{№1}
	
		Про порядки
		A
		и
		B
		известно, что
		A
		+
		B
		$\cong$
		B
		+
		A
		. Верно ли, что тогда
		A
				$\cong$
		B
		? (
				$\cong$
		обозначает
		изоморфизм порядков.)
		
		\begin{solution}
			По определению суммой частичных порядков $X, Y$ $X + Y$ называется порядок на $X' \cup Y'$ ($X' \cong X, Y' \cong Y$, $X', Y'$ не пересекаются), в котором все элементы $X'$ меньше всех элементов $Y'$,  а пары элементов из $X'$ или $Q'$ сравниваются в порядках $P'$ и $Q'$ соотв. Пусть $A$ равно упорядоченному множеству натуральных чисел $N$, а $B = N + N$. Тогда  $A
			+
			B \cong B
			+
			A$, тк $N + N + N \cong N + N + N$, но при этом $A \cong B$ - ложь, тк в $N + N$ есть два наименьших элемента ($ 0 \in N$ и $0' \in N$ (для перехода к непересекающимся множествам обозначил элементы второго множества через $'$)), и при этом в $N$ есть только один наименьший элемент, а значит порядки неизоморфны.
		\end{solution}
	
		\begin{answer}
			Неверно
		\end{answer}
	
	\section*{№2}
	
		Рассмотрим два порядка: делители числа 42 (положительные целые числа, на которые 42 делится нацело) с отношением делимости (
		x
		|
		y
		по определению означает, что
		y
		делится на
		x
		нацело) и
		подмножества множества
		$\{
			1
			,
			2
			,
			3
		\}$
		с порядком по включению
		x
		$\subseteq$
		y
		. Изоморфны ли эти порядки?
		
		\begin{solution}
			Обозначим элементы множества $\{1, 2, 3\}$ как $\{a, b, c\}$ (чтобы не путать числа с делителями 42). Заметим, что $42 = 2 \cdot 3 \cdot 7$. Делителями числа 42 являются числа $1, 2, 3, 7, 2\cdot 3 = 6, 2 \cdot 7 = 14, 3 \cdot 7 = 21, 6\cdot 7 = 42$. Тогда порядки будут изоморфны, а изоморфизм будет выглядеть как $\varnothing \rightarrow 1; \{a\} \rightarrow 2; \{b\} \rightarrow 3; \{c\} \rightarrow 7; \{a, b\} \rightarrow 2 \cdot 3 = 6; \{a, c\} \rightarrow 2 \cdot 7 = 14; \{b, c\} \rightarrow 3 \cdot 7 = 21; \{a, b, c\} \rightarrow 2 \cdot 3 \cdot 7= 42$.
		\end{solution}
	
		\begin{answer}
			да
		\end{answer}
	
	\section*{№3}
	
		Докажите, что линейные порядки
		N
		×
		Z
		и
		Z
		×
		Z
		неизоморфны. (Упорядочение пар лексикографическое.)
		
			\begin{proof}
				Пусть порядки изоморфны $\Rightarrow$ существует изоморфизм. Будем считать, что 1 - наименьшее натурально число (0 - не натуральное). Тогда для пары $(1, 0)$ из N×Z есть пара $(x, y)$ из Z×Z, в которую она переходит. Рассмотрим пару $(x - 1, y)$ из Z×Z. При обратном отображении эта пара переходит в пару $(1, z)$ из N×Z, где $z < 0$.  Заметим, что между $(1, 0)$ и $(1, z)$ в N×Z конечное число элементов, а между парами $(x - 1, y)$ и $(x, y)$ - бесконечное, значит порядки неизоморфны.
			\end{proof}
	
	
	\section*{№4}
	
		Произведение цепей
		[0
		, . . . , n
		-
		1]
		×
		[0
		, . . . , n
		-
		1]
		упорядочено покоординатно. Найдите размер максимальной антицепи в этом порядке.
		
			\begin{solution}
				Покажем, что размер антицепи не больше $n$. Пусть существует антицепь размера $>n$. Тогда, тк кажый элемент произведения имеет вид $(x, y)$, где $0 \leq x \leq n - 1$ и $0 \leq y \leq n - 1$, в антицепь будут входить хотя бы 2 элемента вида $(x, a)$ и $(x, b)$ (или $(a, x)$ и $(b, x)$). А это значит, что в антицепь входят сравнимые элементы, что невозможно. Таким образом, размер антицепи в порядке $\leq n$.
				
				Приведём пример антицепи размера $n$. Прономеруем элементы в цепях от 1 до $n$ и составим антицепь из пар $(x, y)$, где $x$ - $i$ - ый элемент первой цепи, а $y$ - ($n - i$) - ый элемент второй цепи (те антицепь состоит из следующих элементов: $(0, n-1), (1, n - 2), (2, n - 3), \dots (n-2, 1), (n-1, 1)$). Элементы такого подмножества попарно несравнимы, тк для каждой пары $(a, b), (c, d)$ элементов в подмножестве либо $a < c, b > d$, либо $a > c, b < d$,  а значит такое подмножество является антицепью.
			\end{solution}
		
			\begin{answer}
				n
			\end{answer}
		
	\section*{№5}
	
		Докажите, что в любом конечном порядке на
		mn
		+ 1
		элементах есть либо цепь размера
		n
		+ 1
		, либо
		антицепь размера
		m
		+ 1.
		
		\begin{proof}
			В порядке будем искать максимальные элементы. Такие элементы образуют антицепь. Если её размер равен $m + 1$ - доказано, иначе её размер $\leq m$. Рассмотрим порядок без этих элементов. В новом порядке снова выберем максимальные элементы. Если таких элементов будет $m + 1$, то доказано, иначе их $\leq m$. Рассмотрим порядок и без этих элементов. Повторяем рассматривать максимальные элементы. Заметим, что если в порядке $mn + 1$ элемент, то мы так рассмотрим $n + 1$ антицепь. Тогда, тк мы рассматривали максимальные элементы, из каждой из $n + 1$ антицепи можно выбрать по однуму элементу, и эти элементы образуют цепь, длина которой равна $n + 1$.
		\end{proof}
		
	\section*{№6}
	
		Рассмотрим множество невозрастающих бесконечных последовательностей натуральных чисел с
		лексикографическим порядком. Является ли это множество фундированным?
		
		\begin{solution}
			По определению порядок называется фундированным, если каждое непустое подмножество имеет минимальный элемент (множество с фундированным порядком называется фундированным). Докажем, что множество невозрастающих бесконечных последовательностей натуральных чисел с
			лексикографическим порядком является фундированным, те каждое непустое подмножество этого множества имеет минимальный элемент.
			
			Рассмотрим непустое подмножество А из таких последовательностей. Рассмотрим первые члены последовательностей в подмножестве А. Среди таких элементов выберем минимальный $a_1$. Теперь рассмотрим множество последовательностей из А с первым членом $a_1$ и выберем среди вторых членов этих последовательностей минимальный. Обозначим его $a_2$. И так далее выбираем члены $a_3, a_4, \dots, a_n, \dots$. По условию последовательности невозрастающие, а значит $a_1 \geq a_2 \geq a_3 \geq \dots \geq a_n \geq \dots$. Так как множество состоит из невозрастающих бесконечных последовательностей натуральных чисел с
			лексикографическим порядком, то в какой-то момент будет только одна (либо совпадающие) последовательность, у которой на $n$ месте стоит $a_n$, а значит эта последовательность и будет минимальным элементом. Таким образом, по определению изначальное множество является фундированным.
		\end{solution}
	
		\begin{answer}
			Да
		\end{answer}
	
	\section*{№7}
	
		Имеется конечная последовательность нулей и единиц. За один шаг разрешается любую группу 01
		заменить на
		10
		$\dots$
		0
		(произвольное количество нулей). Докажите, что такие шаги нельзя выполнить
		бесконечное количество раз.
		
		\begin{proof}
			Заметим, что за каждый шаг не изменится количество единиц в последовательности. Каждой последовательности поставим в соответствие набор из цифр, где каждая цифра - номер единицы слева в последовательности. Рассмотрим множество из всех таких наборов: наборы можно сравнивать в лексикографическом порядке, и в множестве можно найти минимальный элемент, те оно фундированно. Заметим, что за каждый шаг, при замене любой группы 01 на 10 $\dots$ 0 получится набор, который меньше набора на предыдущем шаге, тк единица сдвинется влево. Тк множество фундированно, то любая убывающая цепь конечна, а значит шаги нельзя выполнить
			бесконечное количество раз.
		\end{proof}
	
		
	
	
	
	
	
	
	
	
	
	
	
	
	
	
	
	
	
	
\end{document}