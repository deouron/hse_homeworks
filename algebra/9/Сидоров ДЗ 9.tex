\documentclass[a4paper, 16pt]{article}

\usepackage[utf8]{inputenc}

\usepackage[russian, english]{babel}
\usepackage{subfiles}
\usepackage[utf8]{inputenc}
\usepackage[T2A]{fontenc}
\usepackage{ucs}
\usepackage{textcomp}
\usepackage{array}
\usepackage{indentfirst}
\usepackage{amsmath}
\usepackage{amssymb}
\usepackage{enumerate}
\usepackage[margin=1.2cm]{geometry}
\usepackage{authblk}
\usepackage{tikz}
\usepackage{icomma}
\usepackage{gensymb}
\usepackage{graphicx}

\newcommand{\dropsign}[1]{\smash{\llap{\raisebox{-.5\normalbaselineskip}{$#1$\hspace{2\arraycolsep}}}}}%

\DeclareGraphicsExtensions{,.png,.jpg}

\DeclareMathOperator*\lowlim{\underline{lim}}
\DeclareMathOperator*\uplim{\overline{lim}}

\graphicspath{{pictures/}}

\renewcommand{\baselinestretch}{1.4}

\renewcommand{\C}{\mathbb{C}}
\newcommand{\N} {\mathbb{N}}
\newcommand{\Q} {\mathbb{Q}}
\newcommand{\Z} {\mathbb{Z}}
\newcommand{\R} {\mathbb{R}}
\newcommand{\ord} {\mathop{\rm ord}}
\newcommand{\Ima}{\mathop{\rm Im}}
\newcommand{\rk}{\mathop{\rm rk}}

\renewcommand{\r}{\right}
\renewcommand{\l}{\left}
\renewcommand{\inf}{\infty}
\newcommand{\Sum}[2]{\overset{#2}{\underset{#1}{\sum}}}
\newcommand{\Lim}[2]{\lim\limits_{#1 \rightarrow #2}}
\newcommand\tab[1][1cm]{\hspace*{#1}}

\newcommand{\task}[1] {\noindent \textbf{Задача #1.} \hfill}
\newcommand{\note}[1] {\noindent \textbf{Примечание #1.} \hfill}
\newenvironment{proof}[1][Доказательство]{%
	\begin{trivlist}
		\item[\hskip \labelsep {\bfseries #1:}]
		\item \hspace{14pt}
	}{
		$ \hfill\blacksquare $
	\end{trivlist}
	\hfill\break
}
\newenvironment{solution}[1][Решение]{%
	\begin{trivlist}
		\item[\hskip \labelsep {\bfseries #1:}]
		\item \hspace{15pt}
	}{
	\end{trivlist}
}

\newenvironment{answer}[1][Ответ]{%
	\begin{trivlist}
		\item[\hskip \labelsep {\bfseries #1:}] \hskip \labelsep
	}{
	\end{trivlist}
	\hfill
}

\title{Алгебра} 
\date{\today}
\author{Сидоров Дмитрий}
\affil{Группа БПМИ 219}


\begin{document}
	\maketitle
	
	\section*{№1}
	
		Реализуем поле $\mathbb{F}_9$ в виде $\Z_3[x]/(x^2 + x + 2)$. Перечислите в этой реализации все элементы данного поля, являющиеся порождающими циклической группы $F^{\times}_9$ .
		
		\begin{solution}
			Заметим, что мультипликативная группа $\mathbb{F}_9$ содержит 8 элементов, тк  $\mathbb{F}_9^{\times} = F_9 \backslash \{0\}$, а значит множество порождающих элементов $\mathbb{F}_9^{\times} $
			совпадает с множеством элементов порядка 8. Таким образом, чтобы найти порождающие элементов группы
			$\mathbb{F}_9^{\times}$ нужно рассмотреть все элементы порядка 8, тк всякая циклическая группа, порождаемая элементом $x$, содержит $\text{ord}(x)$ элементов. Тк $\mathbb{F}_9 = \Z_3[x]/(x^2 + x + 2)$, то 
			$\mathbb{F}_9 = \{0, 1, 2, \overline{x},  \overline{x} + 1, 
			 \overline{x} + 2, 2 \overline{x}, 2 \overline{x} + 1, 2 \overline{x}+2\}$
			 (все многочлены над $\Z_3$, степень которых меньше 2).
			 Значит $\mathbb{F}_9^{\times} = \{1, 2, \overline{x},  \overline{x} + 1, 
			 \overline{x} + 2, 2 \overline{x}, 2 \overline{x} + 1, 2 \overline{x}+2\}$
			 ($\mathbb{F}_9$ без 0). Заметим, что выполняется $\overline{x}^2 + \overline{x} + 2 = 0 \Rightarrow \overline{x}^2 = -\overline{x} - 2 = 2\overline{x} + 1$ и $3 = 0$, тк характеристика поля равна 3. Выберем среди $\{1, 2, \overline{x},  \overline{x} + 1, 
			 \overline{x} + 2, 2 \overline{x}, 2 \overline{x} + 1, 2 \overline{x}+2\}$ те элементы, порядок которых равен 8. Для этого найдём порядок каждого элемента.
			 
			 $1 = 1 \Rightarrow \text{ord}(1) = 1$
			 
			 $2 \to 2^2 = 1 \Rightarrow \text{ord}(2) = 1$
			 
			 $\overline{x} \to \overline{x}^2 = 2\overline{x} + 1 \to
			 2\overline{x}^2 + \overline{x} = 2\overline{x} + 2 \to 2\overline{x}^2 + 2\overline{x} = 2 \to 2\overline{x} \to 2\overline{x}^2 = \overline{x} + 2 \to \overline{x}^2 + 2\overline{x} = \overline{x} + 1 \to \overline{x}^2 + \overline{x} = 1
			  \Rightarrow \text{ord}(\overline{x}) = 8$
			 
			 $\overline{x} + 1 \to \overline{x}^2 + 2\overline{x} + 1 = \overline{x} + 2 \to \overline{x}^2 + 2 = 2\overline{x} \to 2\overline{x}^2 + 2\overline{x} = 2 \to 2\overline{x} + 2 \to 2\overline{x}^2 + \overline{x} + 2 = 2\overline{x} + 1 \to 2\overline{x}^2 + 1 = \overline{x} \to \overline{x}^2 + \overline{x} = 1
			  \Rightarrow \text{ord}(\overline{x} + 1) = 8$
			 
			 $\overline{x} + 2 \to \overline{x}^2 + \overline{x} + 1 = 2 \to 2\overline{x} + 1 \to 2\overline{x}^2 + 2\overline{x} + 2 = 1 \Rightarrow \text{ord}(\overline{x} + 2) = 4$
			 
			 $2\overline{x} \to 4\overline{x}^2 = 2\overline{x} + 1 \to \overline{x}^2 + 2\overline{x} = \overline{x} + 1 \to 2\overline{x}^2 + 2\overline{x} = 2 \to \overline{x} \to 2\overline{x}^2 = \overline{x} + 2 \to 2\overline{x}^2 + \overline{x} = 2\overline{x} + 2 \to \overline{x}^2 + \overline{x} = 1
			  \Rightarrow \text{ord}(2\overline{x}) = 8$
			 
			 $2\overline{x} + 1 \to \overline{x}^2 + \overline{x} + 1 = 2 \to \overline{x} + 2 \to 2\overline{x}^2 + 2\overline{x} + 2 = 1
			  \Rightarrow \text{ord}(2\overline{x} + 1) = 4$
			 
			 $2\overline{x} + 2 \to \overline{x}^2 + 2\overline{x} + 1 = \overline{x} + 2  \to 2\overline{x}^2 + 1 = \overline{x} \to 2\overline{x}^2 + 2\overline{x} = 2 \to \overline{x} + 1 \to 2\overline{x}^2 + \overline{x} + 2 = 2\overline{x} + 1 \to \overline{x}^2  + 2 = 2\overline{x} \to \overline{x}^2 + \overline{x} = 1
			  \Rightarrow \text{ord}(2\overline{x} + 2) = 8$
			  
			  Значит $\overline{x},\ \overline{x} + 1,\ 2\overline{x}, \ 2\overline{x} + 2$ являются порождающими в $\mathbb{F}_9^{\times}$. Тогда, тк из преобразований выше видно, что каждый элемент из $\mathbb{F}_9^{\times}$ может быть представлен как один из элементов $\overline{x},\ \overline{x} + 1,\ 2\overline{x}, \ 2\overline{x} + 2$ в некоторой степени, то $\overline{x},\ \overline{x} + 1,\ 2\overline{x}, \ 2\overline{x} + 2$ являются порождающими циклической группы $\mathbb{F}_9^{\times}$.
		\end{solution}
	
		\begin{answer}
			$\overline{x},\ \overline{x} + 1,\ 2\overline{x}, \ 2\overline{x} + 2$
		\end{answer}
	
	\section*{№2}
	
		Проверьте, что многочлены $x^2 + 3$ и $y^2 + y + 1$ неприводимы над $\Z_5$, и установите явно изоморфизм между полями $\Z_5[x]/(x^2 + 3)$ и $\Z_5[y] / (y^2 + y+1)$.
		
		\begin{solution}
			Покажем, что многочлены $x^2 + 3$ и $y^2 + y + 1$ неприводимы над $\Z_5$. Известно, что многочлен степени 2 неприводим над полем $\Z_5$ тогда и только тогда, когда он
			не имеет корней в поле $\Z_5$. Покажем, что многочлены $x^2 + 3$ и $y^2 + y + 1$ не имеют корней в $\Z_5$, а значит они неприводимы в $\Z_5[x]$ и $\Z_5[y]$ соотв.
			
			  $(x^2 + 3)(0) = 3, \ (x^2 + 3)(1) = 4, \ (x^2 + 3)(2) = 2, \
			  (x^2 + 3)(3) = 2,\ (x^2 + 3)(4) = 4 \Rightarrow x^2 + 3$ не имеет корней в $\Z_5$ (тк $x^2 + 3 \ne 0$ при $0 \leq x \leq 4 )$, а значит неприводим.  
			  
			  Аналогично $y^2 + y + 1$ неприводим, тк 
			  $(y^2 + y + 1)(0) = 1, \ (y^2 + y + 1)(1) = 3, \ (y^2 + y + 1)(2) = 2, \
			  (y^2 + y + 1)(3) = 3, \ (y^2 + y + 1)(4) = 1$.
			  
			  Таким образом, получили, что что многочлены $x^2 + 3$ и $y^2 + y + 1$ неприводимы над $\Z_5$, а значит $\Z_5[x]/(x^2 + 3)$ и $\Z_5[y] / (y^2 + y+1)$ - это поля. Известно, что
			  $\exists a \in \Z_5 / (y^2 + y+1): \ (x^2 + 3)(a) = 0$, тогда рассмотрим гомоморфизм $\varphi : \ \Z_5[x] \to \Z_5[y] / (y^2 + y+1), \ f \to f(a)$ ($\varphi$ является гомоморфизмом, тк сохраняет сумму и произведение, тк является взятием значения многочлена в точке $a$). 
			  
			  Найдём $\text{Ker}\varphi$. По определению ядро состоит из таких многочленов $f$, для которых $f(a) = 0$. Тк ядро является главным идеалом в $\Z_5$, то $\exists g\in \Z_5: \
			  \text{Ker}\varphi = (g)$. Тогда, тк $ (x^2 + 3)(a) = 0$, то $ (x^2 + 3)$ делится на $g$, но тк $ (x^2 + 3)$ неприводим над $\Z_5$, то либо $g$ - константа, либо $g$ пропорционален $ x^2 + 3$. Заметим, что, если выполняется 1-ый случай, то $\varphi$ переводит все многочлены в 0, что невозможно $\Rightarrow$ $g$ пропорционален $ x^2 + 3$, а значит 
			  $\text{Ker}\varphi = (x^2 + 3)$. Тогда по теореме о гомоморфизме колец $\Z_5[x]/(x^2 + 3) \simeq \text{Im}\varphi$. При этом размерности полей $\Z_5[x]/(x^2 + 3)$ и $\Z_5[y] / (y^2 + y+1)$ совпадают (и равны 25), а значит, тк $\text{Im}\varphi \subseteq \Z_5[y] / (y^2 + y+1), \ 
			  \Z_5[y] / (y^2 + y+1) = \text{Im}\varphi$, а значит сущесвует изоморфизм, который каждому многочлену $f \in \Z_5[x]/(x^2 + 3)$ сопоставляет многочлен \\ $f(a) \in \Z_5[y] / (y^2 + y+1)$.
			  
			  Найдём этот изоморфизм явно. Для этого найдём описанный выше $a$. Тк $a \in \Z_5 / (y^2 + y+1)$, то $a$ можно представить в виде многочлена степени не выше 2 (тк каждый элемент $\Z_5 / (y^2 + y+1)$ представляется в виде многочлена  степени не выше 2 = $\text{deg} (y^2 + y+1)$), а значит $a = by + c, \ b, c \in \Z_5$. Тогда, тк $(x^2 + 3)(a) = 0$, а также выполняется $\overline{y}^2 = -\overline{y} - 1 = 4\overline{y} + 4$,  то $a^2 + 3 = (b\overline{y} + c)^2 + 3 =
			  b^2\overline{y}^2 + 2bc\overline{y} + c^2 + 3 = 
			  b^2(4\overline{y} + 4) + 2bc\overline{y} + c^2 + 3 =
			  4b^2\overline{y} + 4b^2 + 2bc\overline{y} + c^2 + 3 = 
			  \overline{y}(4b^2 + 2bc)+ 3 + 4b^2 + c^2 = 0$, а тк равенство выполняется, например, при $b = 2, \ c = 1$ (тк тогда $4b^2 + 2bc = 16 + 4 = 20 \ \vdots \ 5$ и $3 + 4b^2 + c^2 = 20\ \vdots \ 5$), то $a = by + c = 2x + 1$. 
			  
			  Таким образом, получили изоморфизм $\Z_5[x]/(x^2 + 3) \stackrel{\sim}{\to} \Z_5[y] / (y^2 + y+1)$, который задаётся как \\ $b\overline{x} + c \to b(2\overline{y} + 1) + c$.
		\end{solution}
	
		\begin{answer}
			$\Z_5[x]/(x^2 + 3) \stackrel{\sim}{\to} \Z_5[y] / (y^2 + y+1)$, $b\overline{x} + c \to b(2\overline{y} + 1) + c$
		\end{answer}
	
	\section*{№3}
	
		Перечислите все подполя поля $\mathbb{F}_{262144}$, в которых многочлен $x^3 + x^2 + 1$ имеет корень.
		
		\begin{solution}
			Заметим, что $262144 = 2^{18}$, а тк $18 = 2 \cdot 9 = 2 \cdot 3^2$, то подполя $\mathbb{F}_{262144}$ это $\mathbb{F}_2, \ 
			\mathbb{F}_{2^2}, \ \mathbb{F}_{2^3}, \ \mathbb{F}_{2^6}, \ \mathbb{F}_{2^9}, \ \mathbb{F}_{2^{18}}$. Заметим, что $(x^3 + x^2 + 1)(0) = 1, \ (x^3 + x^2 + 1)(1) = 1$ в $\Z_2 \Rightarrow x^3 + x^2 + 1$ не имеет корней в $\Z_2[x]$, а значит этот многочлен неприводим в $\Z_2[x]$, и тогда, тк $\text{deg}(x^3 + x^2 + 1) = 3$ и $2^3 = 8$, то можно реализовать поле $\mathbb{F}_{2^3}$ в виде $\Z_2[x] / (x^3 + x^2 + 1)$. Тогда, тк при $\overline{x}^3 + \overline{x}^2 + 1 = 0$ выполняется $\overline{x}^3 = - \overline{x}^2 - 1 =  \overline{x}^2 + 1$, то $\overline{x}^3 + \overline{x}^2 + 1 = 2\overline{x}^2 + 2 = 0$, а значит в поле $\mathbb{F}_{2^3}$ $x^3 + x^2 + 1$ имеет корень (тк $\overline{x}$ - это элемент поля $\Z_2[x] / (x^3 + x^2 + 1)$). Заметим, что, тк 3 делит 6, 9, 18, то в $\mathbb{F}_{2^6}, \ \mathbb{F}_{2^9}, \ \mathbb{F}_{2^{18}}$ $x^3 + x^2 + 1$ тоже имеет корень, тк эти поля являются расширением поля $\mathbb{F}_{2^3}$, а значит они содержат элемент, который является корнем многочлена $x^3 + x^2 + 1$. 
			
			Теперь рассмотрим, оставшиеся поля, те $\mathbb{F}_2, \ 
			\mathbb{F}_{2^2}$.
			Для $\mathbb{F}_2$ заметим, что это поле содержит 2 элемента, а тк каждое поле содержит 0 и 1, то $\mathbb{F}_2$ содержит только 0 и 1. При этом, как было показано выше, 0 и 1 не являются корнями $x^3 + x^2 + 1$, а значит $x^3 + x^2 + 1$ не имеет корень в $\mathbb{F}_2$. 
			
			Аналогично с $\mathbb{F}_{2^3}$ реализуем  поле $\mathbb{F}_{2^2}$ в виде $\Z_2[x] / (x^2 + x + 1) \ ($тк $(x^2 + x + 1)(0) = 1, \ (x^2 + x + 1)(1) = 1 \Rightarrow x^2 + x + 1$ неприводим в $\Z_2[x])$, и при этом элементы этого поля являются многочленами степени меньше 2, те это поле состоит из элементов $\{0, \ 1, \ \overline{x}, \ \overline{x} + 1\}$. Тогда при 
			$\overline{x}^2 + \overline{x} + 1 = 0$ выполняется $\overline{x}^2 = \overline{x} + 1$. 
			$(x^3 + x^2 + 1)(0) = 1, \ (x^3 + x^2 + 1)(1) = 1, \ (x^3 + x^2 + 1)(\overline{x}) = \overline{x}^2 + \overline{x} + \overline{x}^2 + 1 = 
			\overline{x} + 1, \ 
			(x^3 + x^2 + 1)(\overline{x} + 1) = (\overline{x} + 1)^3 + (\overline{x} + 1)^2 + 1 = (\overline{x} + 1)(\overline{x}^2 + 1) + (\overline{x}^2 + 1) + 1 =
			(\overline{x} + 1)\overline{x} + \overline{x} + 1 = 
			\overline{x} + 1 + \overline{x} + \overline{x} + 1 = \overline{x}$.
			Таким образом, значения $x^3 + x^2 + 1$ от всех элементов поля $\Z_2[x] / (x^2 + x + 1)$ не равно 0, а значит $x^3 + x^2 + 1$ не имеет корней в $\mathbb{F}_{2^2}$.
			
			Итого, многочлен $x^3 + x^2 + 1$ имеет корень в подполях 
			$ \mathbb{F}_{2^3}, \ \mathbb{F}_{2^6}, \ \mathbb{F}_{2^9}, \ \mathbb{F}_{2^{18}}$.
		\end{solution}
	
		\begin{answer}
			$ \mathbb{F}_{2^3}, \ \mathbb{F}_{2^6}, \ \mathbb{F}_{2^9}, \ \mathbb{F}_{2^{18}}$
		\end{answer}
	
	\section*{№4}
	
		Пусть $p$ - простое число, $q = p^n$ и $\alpha \in \mathbb{F}_q$. Докажите, что если многочлен $x^p - x - \alpha \in \mathbb{F}_q[x]$ имеет корень, то он разлагается на линейные множители.
		
		\begin{proof}
			Рассмотрим поле $\mathbb{F}_p$. Заметим, что тк $q = p^n$, те $q$ делится на $p$ (причём единственный способом), 
			то $\mathbb{F}_p \subseteq \mathbb{F}_q$. При этом $\forall 
			a, b \in \mathbb{F}_q$ выполняется $(a + b)^p = a^p + b^p$, тк 
			$\text{char} \mathbb{F}_q = p$. Рассмотрим произвольный элемент $y \in \mathbb{F}_p$. Заметим, что порядок $ \mathbb{F}_q^{\times}$ равен $p - 1$, те выполняется $y^{p-1} \cdot y = e \cdot y = y \Rightarrow y^p = y$. По условию $x^p - x - \alpha$ имеет корень. Обозначим его как $x_0$. Тогда $x_0^p - x_0  - \alpha = 0$. Рассмотрим $(x^p - x - \alpha)(x_0 - y)$ (значение многочлена при $x = x_0 - y$)  :
			$(x^p - x - \alpha)(x_0 - y) = (x_0 - y)^p - (x_0 - y) - \alpha = 
			x_0^p - y^p - x_0 +y - \alpha = x_0^p - y - x_0 + y - \alpha = 
			x_0^p - x_0  - \alpha = 0$. Таким образом, $x_0$ и $y$ образуют корень многочлена $x^p - x - \alpha$. Заметим, что в 
			$\mathbb{F}_p \ p$ элементов, а тк мы брали произвольный $y$, то  многочлен $x^p - x - \alpha$ имеет $p$ корней, но тк его степень тоже равна $p$, он разлагается на линейные множители.
		\end{proof} 
	
	
	
	
	
	
	
	
	
	
	
	
	
	
	
	
	
	
	
	
	
	
	
	
	
	
	
	
	
	
\end{document}