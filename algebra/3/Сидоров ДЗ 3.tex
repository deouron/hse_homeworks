\documentclass[a4paper, 16pt]{article}

\usepackage[utf8]{inputenc}

\usepackage[russian, english]{babel}
\usepackage{subfiles}
\usepackage[utf8]{inputenc}
\usepackage[T2A]{fontenc}
\usepackage{ucs}
\usepackage{textcomp}
\usepackage{array}
\usepackage{indentfirst}
\usepackage{amsmath}
\usepackage{amssymb}
\usepackage{enumerate}
\usepackage[margin=1.2cm]{geometry}
\usepackage{authblk}
\usepackage{tikz}
\usepackage{icomma}
\usepackage{gensymb}
\usepackage{graphicx}

\DeclareGraphicsExtensions{,.png,.jpg}

\DeclareMathOperator*\lowlim{\underline{lim}}
\DeclareMathOperator*\uplim{\overline{lim}}

\graphicspath{{pictures/}}

\renewcommand{\baselinestretch}{1.4}

\renewcommand{\C}{\mathbb{C}}
\newcommand{\N} {\mathbb{N}}
\newcommand{\Q} {\mathbb{Q}}
\newcommand{\Z} {\mathbb{Z}}
\newcommand{\R} {\mathbb{R}}
\newcommand{\ord} {\mathop{\rm ord}}
\newcommand{\Ima}{\mathop{\rm Im}}
\newcommand{\rk}{\mathop{\rm rk}}

\renewcommand{\r}{\right}
\renewcommand{\l}{\left}
\renewcommand{\inf}{\infty}
\newcommand{\Sum}[2]{\overset{#2}{\underset{#1}{\sum}}}
\newcommand{\Lim}[2]{\lim\limits_{#1 \rightarrow #2}}
\newcommand\tab[1][1cm]{\hspace*{#1}}

\newcommand{\task}[1] {\noindent \textbf{Задача #1.} \hfill}
\newcommand{\note}[1] {\noindent \textbf{Примечание #1.} \hfill}
\newenvironment{proof}[1][Доказательство]{%
	\begin{trivlist}
		\item[\hskip \labelsep {\bfseries #1:}]
		\item \hspace{14pt}
	}{
		$ \hfill\blacksquare $
	\end{trivlist}
	\hfill\break
}
\newenvironment{solution}[1][Решение]{%
	\begin{trivlist}
		\item[\hskip \labelsep {\bfseries #1:}]
		\item \hspace{15pt}
	}{
	\end{trivlist}
}

\newenvironment{answer}[1][Ответ]{%
	\begin{trivlist}
		\item[\hskip \labelsep {\bfseries #1:}] \hskip \labelsep
	}{
	\end{trivlist}
	\hfill
}

\title{Алгебра} 
\date{\today}
\author{Сидоров Дмитрий}
\affil{Группа БПМИ 219}


\begin{document}
	\maketitle
	
	\section*{№1}
	
		Сколько элементов порядков 2, 3, 6 и 9 в группе $\Z_2 \times \Z_6 \times \Z_9$?
		
		\begin{solution}
			В циклической подгруппе порядка $m$ уравнение $x^n=e$ имеет НОД$(m, n)$ решений.			
			Можно считать, что  группа $\Z_2 \times \Z_6 \times \Z_9$ состоит из набора из 3 чисел (3 координаты). Тогда необходимо найти количество решений уравнений $x^2 = 1, x^3 = 1, x^6=1, x^9=1$ в группе $\Z_2 \times \Z_6 \times \Z_9$. Получаем $2 \cdot 2 \cdot 1 = 4, \ 1 \cdot 3 \cdot 3 = 9, \ 2 \cdot 6 \cdot 3 = 36, \  1 \cdot 3 \cdot 9 = 27$. Значит элементов порядка 2 $4 - 1 = 3$ (минус 1, тк единичный элемент (ситуация, когда все координаты равны 0) имеет порядок 1). Элементов порядка 3 $9 - 1 = 8$ (аналогично не считаем единичный элемент). Элементов порядка 6 $36 - 1 - 3 - 8 = 24$ (не считаем единичный, а так же не считаем элементы порядка 2 и 3, тк $2 \mid 6$ и $3 \mid 6$). Элементов порядка 9 $27 - 1 - 8 = 18$ (не считаем единичный, а так же не считаем элементы порядка 3, тк $3 \mid 9$).				
		\end{solution}
	
		\begin{answer}
			3, 8, 24, 18
		\end{answer}
	
	\section*{№2}
	
		Сколько подгрупп порядков 7 и 14 в нециклической абелевой группе порядка 98?
		
		\begin{solution}
			Заметим, что $98 = 2 \cdot 7^2$. Найдём все абелевые группы порядка 98. Это $\Z_2 \times \Z_7 \times \Z_7$ и  $\Z_2 \times \Z_{49}$. При этом $\Z_2 \times \Z_{49} \simeq \Z_{98}$ (по теореме, тк 2 и 49 взаимно просты) и является циклической. Значит нециклической абелевой группой порядка 98 является только $\Z_2 \times \Z_7 \times \Z_7$. Найдём количество 
			элементов порядков 7 и 14, а дальше, для 7, тк 7 - простое число, а подгруппы простого порядка попарно не пересекаются, и в
			подгруппе порядка 7 имеется ровно 6 элементов порядка 7 (тк в группе $\Z_7$ для всех элементов выполняется $x^7 = 1$, но порядок $e$ равен 1 (остальные имеют порядок 7)), разделим это количество на 6, а для 14, тк $\Z_2 \times \Z_7 \simeq \Z_{14}$, те $\Z_{14}$ является циклической, тк $\Z_2$ и  $\Z_7$ циклические, разделим количество элементов порядка 14 на количество взаимно простых с 14 числел в промежутке от 1 до 14 (тк количество взаимно простых с 14 числел в промежутке от 1 до 14 равно количеству образующих элементов циклической подгруппы порядка 14, а каждый образующий входит ровно в одну подгруппу).
			
			Найдём количество элементов порядка 7: тк мы рассматриваем группу $\Z_2 \times \Z_7 \times \Z_7$, то их $7 \cdot 7 - 1 = 48$ (количество решений уравнения $x^7 = 1$ умножить на 2 минус 1, тк единичный элемент имеет порядок 1 (аналогично с №1)). Значит подгрупп порядка 7 $\frac{48}{6} = 8$. Найдём количество элементов порядка 14: в $\Z_2$ 2 элемента, надо взять элемент порядка 2, значит есть 1 способ (тк нельзя брать единичный), для $\Z_7 \times \Z_7$ можно брать любые элементы, кроме $(e, e)$, те всего $7 \cdot 7 - 1 = 48$ способов. Итого $1 \cdot 48 = 48$ элементов порядка 14. Тогда подгрупп порядка 14 $\frac{48}{6} = 8$ (взаимно простые с 14 числа в промежутке от 1 до 14 - это 1, 3, 5, 9, 11, 13, те 6 чисел).
		\end{solution}
	
		\begin{answer}
			8, 8
		\end{answer}
	
	\section*{№3}
	
		При каком наименьшем $n \in \N$ группа $\Z_{10} \times \Z_{12} \times \Z_{15}$ изоморфна прямому произведению $n$ циклических групп?
		
		\begin{solution}
			По теореме получаем $\Z_{10} \times \Z_{12} \times \Z_{15} \simeq (\Z_2 \times \Z_5) \times (\Z_3 \times \Z_4) \times (\Z_3 \times \Z_5) \simeq (\Z_2 \times \Z_5 \times \Z_3) \times (\Z_4 \times \Z_3 \times \Z_5) \simeq \Z_{30} \times \Z_{60} \Rightarrow n \leq 2$ . Покажем, что $n > 1$. Пусть $n=1$, тогда группа $\Z_{10} \times \Z_{12} \times \Z_{15}$ изоморфна  циклической группе, значит является циклической. Однако группа $\Z_{10} \times \Z_{12} \times \Z_{15}$ не циклическая, тк если бы она была бы циклической, то она была бы изоморфна группе $\Z_{10 \cdot 12 \cdot 15}$, те $\Z_{1800}$ (циклическая группа порядка $10 \cdot 12 \cdot 15$), но при этом для группы $\Z_{10} \times \Z_{12} \times \Z_{15}$ выполняется $x^{60} = e$ (60 = НОК$(10,12, 15)$), те в этой группе нет образующего элемента порядка 1800. Таким образом, $n > 1$, а значит $n = 2$.
		\end{solution}
	
		\begin{answer}
			2
		\end{answer}
	
	\section*{№4}
	
		Пусть $k$ - наибольший порядок элементов конечной абелевой группы $A$. Докажите, что порядок любого элемента $A$ делит $k$.
		
		\begin{proof}
			Пусть $k$ - наибольший порядок элементов конечной абелевой группы $A$, $x$ - элемент порядка $k$, а $y$ - какой-нибудь другой элемент. Пусть порядок $y$ равен $n$ и  не делит $k$. Тогда существует такое простое $p$, что степень $p$ в разложении $n$ на простые множители больше, чем степень $p$ в разложении $k$ на простые множители, те $k = p^i \cdot x', n = p^j \cdot y'$, где $j > i$, а $x', y'$ не делятся на $p$. Тогда элементы $x^{p^i}$ и  $y^{y'}$ имеют взаимно простые порядки ($x'$ и $p^j$ соотв). Тк для любых двух элементов $x, y \in G$, имеющих порядки $a, b$ соотв, которые взаимно просты, выполяется, что порядок $xy$ равен $ab$ (тк если $(xy)^m = e$, то $(xy)^{mb}  = e$, $y^{mb} = e$, и тогда $x^{mb} = e$, а значит тк $a, b$ взаимно просты, то $a$ делит $m$ (аналогично $b$ делит $m$), и тогда $ab$ делит $m$, тк $a, b$ взаимно просты), то порядок $x^{p^i}y^{y'}$ равен $x'p^j > x'p^i = k$, что противоречит тому, что $k$ - наибольший порядок элементов конечной абелевой группы $A$. Таким образом, получили противоречие, а значит порядок любого элемента $A$ делит $k$.
		\end{proof}
	
	
	
	
	
\end{document}