\documentclass[a4paper, 16pt]{article}

\usepackage[utf8]{inputenc}

\usepackage[russian, english]{babel}
\usepackage{subfiles}
\usepackage[utf8]{inputenc}
\usepackage[T2A]{fontenc}
\usepackage{ucs}
\usepackage{textcomp}
\usepackage{array}
\usepackage{indentfirst}
\usepackage{amsmath}
\usepackage{amssymb}
\usepackage{enumerate}
\usepackage[margin=1.2cm]{geometry}
\usepackage{authblk}
\usepackage{tikz}
\usepackage{icomma}
\usepackage{gensymb}
\usepackage{graphicx}

\newcommand{\dropsign}[1]{\smash{\llap{\raisebox{-.5\normalbaselineskip}{$#1$\hspace{2\arraycolsep}}}}}%

\DeclareGraphicsExtensions{,.png,.jpg}

\DeclareMathOperator*\lowlim{\underline{lim}}
\DeclareMathOperator*\uplim{\overline{lim}}

\graphicspath{{pictures/}}

\renewcommand{\baselinestretch}{1.4}

\renewcommand{\C}{\mathbb{C}}
\newcommand{\N} {\mathbb{N}}
\newcommand{\Q} {\mathbb{Q}}
\newcommand{\Z} {\mathbb{Z}}
\newcommand{\R} {\mathbb{R}}
\newcommand{\ord} {\mathop{\rm ord}}
\newcommand{\Ima}{\mathop{\rm Im}}
\newcommand{\rk}{\mathop{\rm rk}}

\renewcommand{\r}{\right}
\renewcommand{\l}{\left}
\renewcommand{\inf}{\infty}
\newcommand{\Sum}[2]{\overset{#2}{\underset{#1}{\sum}}}
\newcommand{\Lim}[2]{\lim\limits_{#1 \rightarrow #2}}
\newcommand\tab[1][1cm]{\hspace*{#1}}

\newcommand{\task}[1] {\noindent \textbf{Задача #1.} \hfill}
\newcommand{\note}[1] {\noindent \textbf{Примечание #1.} \hfill}
\newenvironment{proof}[1][Доказательство]{%
	\begin{trivlist}
		\item[\hskip \labelsep {\bfseries #1:}]
		\item \hspace{14pt}
	}{
		$ \hfill\blacksquare $
	\end{trivlist}
	\hfill\break
}
\newenvironment{solution}[1][Решение]{%
	\begin{trivlist}
		\item[\hskip \labelsep {\bfseries #1:}]
		\item \hspace{15pt}
	}{
	\end{trivlist}
}

\newenvironment{answer}[1][Ответ]{%
	\begin{trivlist}
		\item[\hskip \labelsep {\bfseries #1:}] \hskip \labelsep
	}{
	\end{trivlist}
	\hfill
}

\title{Алгебра} 
\date{\today}
\author{Сидоров Дмитрий}
\affil{Группа БПМИ 219}


\begin{document}
	\maketitle
	
	\section*{№1}
	
		Какие значения может принимать длина убывающей (в лексикографическом порядке) цепочки одночленов от переменных $x_1, x_2, x_3$, начинающейся с одночлена $x_1x_2^3x_3^2$ и заканчивающейся одночленом $x_1x_2^2x_3^3$?
		
		\begin{solution}
			Тк цепочка начинается с одночлена $x_1x_2^3x_3^2$ и заканчивающейся одночленом $x_1x_2^2x_3^3$, то она состоит хотя бы из двух одночленов, а значит её длина не меньше 2 (например, если длина равна 2, то цепочка имеет вид $x_1x_2^3x_3^2 > x_1x_2^2x_3^3$). Теперь покажем, что длина цепочки может быть равна любому $n > 2, n \in N$. Если цепочка начинается с одночлена $x_1x_2^3x_3^2$ и заканчивающейся одночленом $x_1x_2^2x_3^3$, то цепочка вида $x_1x_2^3x_3^2 > x_1x_2^2x_3^{n + 1} > x_1x_2^2x_3^{n} > \dots > x_1x_2^2x_3^{4} > x_1x_2^2x_3^{3}$ удовлетворяет условию про начало и конец, является убывающей, а так же её длина равна $1 + ((n + 1) - 3 + 1) = n$. Таким образом, длина убывающей (в лексикографическом порядке) цепочки одночленов от переменных $x_1, x_2, x_3$, начинающейся с одночлена $x_1x_2^3x_3^2$ и заканчивающейся одночленом $x_1x_2^2x_3^3$ равна $n \geq 2, n \in N$.
		\end{solution}
	
		\begin{answer}
			$n \geq 2, n \in N$
		\end{answer}
	
	\section*{№2}
	
		Найдите остаток многочлена $g$ относительно системы $\{f\}$, где $g = x_2^4x_3^5 + 2x_1x_2^4x_3 + x_1^2x_2^2, \\ f = x_2^4x_3 - 2x_1x_2x_3^2 + x_1x_2^2$.
		
		\begin{solution}
			Известно, что множество $\{f\}$ является системой Грёбнера, тк единственный $S$-многочлен этой системы равен 0 (факт из лекции). Значит остаток многочлена $g$ относительно системы $\{f\}$ определён однозначно (те не зависит от порядка элементарных редукций). Заметим, что $L(f) = x_1x_2^2$. Теперь найдём остаток многочлена $g$ относительно системы $\{f\}$: \\
			$g = x_2^4x_3^5 + 2x_1x_2^4x_3 + x_1^2x_2^2 \stackrel{f \cdot x_1}{\to}  x_2^4x_3^5 + 2x_1x_2^4x_3 + x_1^2x_2^2 - 
			(x_1x_2^4x_3 - 2x_1^2x_2x_3^2 + x_1^2x_2^2) = 
			x_2^4x_3^5 + 2x_1x_2^4x_3 - x_1x_2^4x_3 + 2x_1^2x_2x_3^2 = x_2^4x_3^5 + x_1x_2^4x_3 + 2x_1^2x_2x_3^2 \stackrel{f \cdot x_2^2x_3}{\to} x_2^4x_3^5 + x_1x_2^4x_3 + 2x_1^2x_2x_3^2 - 
			(x_2^6x_3^2 - 2x_1x_2^3x_3^3 + x_1x_2^4x_3) = 
			x_2^4x_3^5 + 2x_1^2x_2x_3^3 - x_2^6x_3^2 + 2x_1x_2^3x_3^3 \stackrel{f \cdot 2x_2x_3^3}{\to} x_2^4x_3^5 + 2x_1^2x_2x_3^2 - x_2^6x_3^2 + 2x_1x_2^3x_3^3 - 
			(2x_2^5x_3^4 - 4x_1x_2^2x_3^5 + 2x_1x_2^3x_3^3) = 
			x_2^4x_3^5 + 2x_1^2x_2x_3^2 - x_2^6x_3^2 - 2x_2^5x_3^4 + 4x_1x_2^2x_3^5 \stackrel{f \cdot4x_3^5}{\to} = 
			x_2^4x_3^5 + 2x_1^2x_2x_3^2 - x_2^6x_3^2 - 2x_2^5x_3^4 + 4x_1x_2^2x_3^5 - 
			(4x_2^4x_3^6 - 8x_1x_2x_3^7 + 4x_1x_2^2x_3^5) = 
			x_2^4x_3^5 + 2x_1^2x_2x_3^2 - x_2^6x_3^2 - 2x_2^5x_3^4 - 4x_2^4x_3^6 + 8x_1x_2x_3^7$. Заметим, что в полученном многочлене каждый одночлен не делится на $L(f) = x_1x_2^2$, а значит этот многочлен нередуцируем относительно $\{f\}$, те является остатком многочлена $g$ относительно системы $\{f\}$.
		\end{solution}
	
		\begin{answer}
			$x_2^4x_3^5 + 2x_1^2x_2x_3^2 - x_2^6x_3^2 - 2x_2^5x_3^4 - 4x_2^4x_3^6 + 8x_1x_2x_3^7$
		\end{answer}
	
	\section*{№3}
	
		Выясните, является ли множество $\{f_1, f_2, f_3\}$ системой Грёбнера, где $f_1 = 2x_1x_2 + 4x_1x_3 + x_2x_3^2, \ \\ 
		f_2 = 4x_1x_3^2+x_2x_3^3-4, \ f_3 = x_2^2x_3^3 - 4x_2 - 8x_3$.
		
		\begin{solution}
			По критерию Бухбергера $F = \{f_1, f_2, f_3\}$ - система Грёбнера $\Leftrightarrow$ для любых $f'_1, f'_2 \in F$ многочлен $S(f'_1, f'_2)$ редуцируем к нулю относительно $F$, где $S(f'_1, f'_2) := m_1f'_1 - m_2f'_2, \ m = \text{НОК}(L(f'_1), L(f'_2)) = L(f'_1) \cdot m_1 = L(f'_2) \cdot m_2$.
			
			Так же заметим, что $L(f_1) = 2x_1x_2, \ L(f_2) = 4x_1x_3^2, \
			L(f_3) = x_2^2x_3^3$.
			
			Рассмотрим $S(f_1, f_2)$. $m = \text{НОК}(L(f_1), L(f_2)) = \text{НОК}(2x_1x_2, 4x_1x_3^2) = 4x_1x_2x_3^2 \Rightarrow m_1 = 2x_3^2, \ m_2 = x_2 \Rightarrow S(f_1, f_2) = 2x_3^2(2x_1x_2 + 4x_1x_3 + x_2x_3^2) - x_2(4x_1x_3^2+x_2x_3^3-4) = 4x_1x_2x_3^2 + 8x_1x_3^3 + 2x_2x_3^4 - (4x_1x_2x_3^2+x_2^2x_3^3-4x_2) = 8x_1x_3^3 + 2x_2x_3^4 - x_2^2x_3^3 + 4x_2 \stackrel{f_2 \cdot 2x_3}{\to} 8x_1x_3^3 + 2x_2x_3^4 - x_2^2x_3^3 + 4x_2 - (8x_1x_3^3+2x_2x_3^4-8x_3) = - x_2^2x_3^3 + 4x_2 + 8x_3 \stackrel{f_3 \cdot -1}{\to} 
			- x_2^2x_3^3 + 4x_2 + 8x_3 + (x_2^2x_3^3 - 4x_2 - 8x_3) = 
			0$. Таким образом, $S(f_1, f_2) \stackrel{F}{\rightsquigarrow
			} 0$.
			 
			Рассмотрим $S(f_1, f_3)$. $m = \text{НОК}(L(f_1), L(f_3)) = \text{НОК}(2x_1x_2, x_2^2x_3^3) = 2x_1x_2^2x_3^3 \Rightarrow m_1 = x_2x_3^3, \ m_3 = 2x_1 \Rightarrow S(f_1, f_3) = 
			x_2x_3^3(2x_1x_2 + 4x_1x_3 + x_2x_3^2) - 2x_1(x_2^2x_3^3 - 4x_2 - 8x_3) = 4x_1x_2x_3^4 + x_2^2x_3^5 + 8x_1x_2 + 16x_1x_3 \stackrel{f_2 \cdot x_2x_3^2}{\to} 4x_1x_2x_3^4 + x_2^2x_3^5 + 8x_1x_2 + 16x_1x_3 - 
			(4x_1x_2x_3^4+x_2^2x_3^5-4x_2x_3^2) = 
			8x_1x_2 + 16x_1x_3 + 4x_2x_3^2 \stackrel{f_1 \cdot 4}{\to} 
			8x_1x_2 + 16x_1x_3 + 4x_2x_3^2 - (8x_1x_2 + 16x_1x_3 + 4x_2x_3^2) = 0$. Таким образом, $S(f_1, f_3) \stackrel{F}{\rightsquigarrow
			} 0$.
		
			Рассмотрим $S(f_2, f_3)$. $m = \text{НОК}(L(f_2), L(f_3)) = \text{НОК}(4x_1x_3^2, x_2^2x_3^3) = 4x_1x_2^2x_3^3 \Rightarrow m_2 = x_2^2x_3, \ m_3 = 4x_1 \Rightarrow S(f_2, f_3)
			=  x_2^2x_3(4x_1x_3^2+x_2x_3^3-4) - 4x_1(x_2^2x_3^3 - 4x_2 - 8x_3) = x_2^3x_3^4-4x_2^2x_3 + 16x_1x_2 + 32x_1x_3 \stackrel{f_1 \cdot 8}{\to} x_2^3x_3^4-4x_2^2x_3 + 16x_1x_2 + 32x_1x_3 - 
			(16x_1x_2 + 32x_1x_3 + 8x_2x_3^2) = x_2^3x_3^4-4x_2^2x_3 - 8x_2x_3^2 \stackrel{f_3 \cdot x_2x_3}{\to} x_2^3x_3^4-4x_2^2x_3 - 8x_2x_3^2 - 
			(x_2^3x_3^4 - 4x_2^2x_3 - 8x_2x_3^2) = 0$. Таким образом, $S(f_2, f_3) \stackrel{F}{\rightsquigarrow
			} 0$.
		
			Таким образом, $S(f_1, f_2), S(f_1, f_3), S(f_2, f_3)$ редуцируются к нулю относительно $F$. Значит $S(f_2, f_1) = -S(f_1, f_2), \\ S(f_3, f_1) = -S(f_1, f_3),\  S(f_3, f_2) = -S(f_2, f_3)$ тоже редуцируются к нулю относительно $F$. Кроме того, известно, что $S(f_1, f_1) = S(f_2, f_2) = S(f_3, f_3) = 0 \stackrel{F}{\rightsquigarrow
			} 0$. Итого, для любых $f'_1, f'_2 \in F$ многочлен $S(f'_1, f'_2)$ редуцируем к нулю относительно $F$, а значит $F = \{f_1, f_2, f_3\}$ является системой Грёбнера.
		\end{solution}
	
		\begin{answer}
			является
		\end{answer}
	
	\section*{№4}
	
		Докажите, что множество $F \subseteq K[x] \backslash \{0\}$ является системой Грёбнера тогда и только тогда, когда существует такой многочлен $f \in F$, который делит любой многочлен из $F$.
		
		\begin{proof}
			Сначала докажем, что если множество $F \subseteq K[x] \backslash \{0\}$ является системой Грёбнера, то существует такой многочлен $f \in F$, который делит любой многочлен из $F$. Если множество $F \subseteq K[x] \backslash \{0\}$ является системой Грёбнера, то для всякого многочлена его остаток относительно $F$ определён однозначно, и по критерию Бухбергера \\ $\forall f, g \in F \ S(f, g) \stackrel{F}{\rightsquigarrow} 0$. Рассмотрим многочлен $f \in F$ с минимальной степенью и произвольный многочлен $g \in F$, степень которого равна степени $f$ (те $f = C \cdot g$, где $C$ - некоторая константа). Покажем, что многочлен минимальной степени делит все многочлены из $F$ такой же степени. Тк $f, g$ имеют одинаковую степень (обозначим её за $n$), и $S(f, g) = C_1 \cdot f - C_2 \cdot g$, где $C_1, C_2$ - некоторые константы, то найдутся такие $C_1, C_2$, что старшие члены при $f, g$ сократятся, а значит степень $S(f, g) < n \Rightarrow$ в $S(f, g)$ каждый одночлен не делится на старшие члены $f, g$, а тк они имеют минимальную степень в $F$, то $S(f, g)$ не делится на старший член любого многочлена из $F$. При этом известно, что $ S(f, g) \stackrel{F}{\rightsquigarrow} 0$, а значит $C_1 \cdot f - C_2 \cdot g = 0 \Rightarrow C_1 \cdot f  = C_2 \cdot g \Rightarrow f = C_3 \cdot g, \ C_3 = const \Rightarrow f$ делит $g$, те многочлен минимальной степени делит все многочлены из $F$ такой же степени. Теперь покажем, что $f$ делит все многочлены $g \in F$, степень которых больше степени $f$ (обозначим их  $n < m$ соотв). Тк $F \subseteq K[x] \backslash \{0\}$ \\ (те рассматриваемые многочлены являются многочленами от одной переменной), то НОК$(L(f), L(g)) = L(g) \Rightarrow S(f, g) = x^{m - n} \cdot f - g$. Тк $f$ имеет минимальную степень, то можно провести редукцию только с его помощью, те, тк  $ S(f, g) \stackrel{F}{\rightsquigarrow} 0$, получим, что $S(f, g) \ \vdots \ f$ (на каждом этапе из $S(f, g)$ вычитаем $m_i \cdot f$). Итого, получим, что $x^{m-n} \cdot f - g - m_1 \cdot f - m_2 \cdot f - \dots - m_k \cdot f = 0 \Rightarrow$ тк правая часть делится на $f$, то левая часть делится на $f$, а значит $g \ \vdots \ f$. Таким образом, любой многочлен из $F$ (тк в $F$ нет многочленов степени меньше степени $f$, тк его степень минимальна, а так же $f$ делит многочлены, степени которых больше или равна его степени) делится на $f$.
			
			Теперь докажем, что если существует такой многочлен $f \in F$, который делит любой многочлен из $F$, то множество $F \subseteq K[x] \backslash \{0\}$ является системой Грёбнера.
			По критерию Бухбергера $F$ - система Грёбнера $\Leftrightarrow$ для любых $f_1, f_2 \in F$ многочлен $S(f_1, f_2)$ редуцируем к нулю относительно $F$. Тогда для прозвольных $f_1, f_2 \in F$ известно, что они делятся на некоторый $f \in F$, и тогда $S(f_1, f_2) = m_1 \cdot f_1 - m_2 \cdot f_2$ делится на $f$ (тк $m_1 \cdot f_1, \ m_2 \cdot f_2$ делятся на $f$). Значит $S(f_1, f_2)$ редуцируется к нулю относительно $F$, тк $\exists g \in K[x] \backslash \{0\}: \ S(f_1, f_2) = g \cdot f$ (тк $S(f_1, f_2)$ делится на $f$). Таким образом, тк $f_1, f_2$ произвольные, то $F$ является системой Грёбнера.
		\end{proof}
	
	
\end{document}